\documentclass{article}

\usepackage{graphicx}
%\usepackage[active]{srcltx}
\usepackage{amsmath}
\usepackage{amssymb}




%%%%%%%\documentclass{article}
\usepackage{natbib}
\usepackage{rotating}
\usepackage{floatpag}
\rotfloatpagestyle{empty}
\usepackage{amsthm}
\usepackage{graphicx}
%%\usepackage{bbding}





%\usepackage{multind}\Providespackage{multind}



%%%%%%%%%\usepackage{makeidx}

%\nopagebreak

%\usepackage{txfonts}
%\usepackage{wasysym}
\usepackage[tone,extra]{tipa}
%\usepackage{flag,fleqn}
\usepackage{flagderiv}%,farray}

\usepackage{amsfonts}
\usepackage{amssymb}
\usepackage{amsmath}
%\usepackage{theorem}
\usepackage{epsfig}
%\input{flags.tex}
%\input{flag.sty}
\usepackage[english]{babel}

%%%%%%\usepackage{multind}\ProvidesPackage{multind}

%%\pagestyle{headings}
%%\renewcommand{\topmargin}{-0.1cm}
%%\renewcommand{\baselinestretch}{1.07}
%%\textheight=21cm
%%\headsep=1cm
%5\textwidth=13.3cm
%\mathindent=1.5cm
%%\evensidemargin=2.5cm
%%\oddsidemargin=2.5cm
%%\parindent=0.5cm
%%\normalmarginpar

%%%%%%%\makeindex{author}
%%%%%%%\makeindex{definition}
%%%%%%%\makeindex{technot}
%%%%%%%\makeindex{subject}


%\nopagebreak
%%%%%%%\nofiles

%\makeindex


\theoremstyle{plain}
\newtheorem{The}{Theorem}[section]
\newtheorem{Ste}[The]{Theorem}
\newtheorem{Cor}[The]{Corollary}
\newtheorem{Lem}[The]{Lemma}
\newtheorem{Not}[The]{Notation}
\newtheorem{Rem}[The]{Remark}
\newtheorem{Rems}[The]{Remarks}
\newtheorem{For}[The]{Formula}
\newtheorem{Rul}[The]{Rule}
\newtheorem{Con}[The]{Convention}
\newtheorem{Note}[The]{Note}
\newtheorem{Exp}[The]{Explanation}


\theoremstyle{definition}
\newtheorem{Def}[The]{Definition}
\newtheorem{Exa}[The]{Example}
\newtheorem{Exas}[The]{Examples}

%%\newtheorem{Def}[definition]{Definition}
%%\newtheorem{Exa}[Def]{Example}
%%\newtheorem{Exas}[Def]{Examples}

%\makeatletter
%\renewcommand\@biblabel[1]{}
%\makeatother


\begin{document}
\title{\bf {A new name-free system of lambda calculus with delayed updating}\\ {\small 5th draft version}}
%\marginpar{v4. 0--1}}
\author{Ferruccio Guidi and Rob Nederpelt}
\maketitle
\begin{abstract}
%As our basic system we take the {\em Calculus of Constructions\/} of \citealp{CoqHue}.
In this paper we focus on the updating of variables connected to beta-reduction in lambda calculus.
\end{abstract}


%\frontmatter
%\input{fairouz-frontmatter}

%%\vspace{-2cm}




%%%%%%\maketitle

%%\setcounter{page}{0}
%%\clearpage
%%\thispagestyle{empty}
%%\mbox{$$}
%\cleardoublepage
%\normalsize
%%%%%%%\cleardoublepage

%%%%%%%\pagenumbering{roman}
%%%%%%%\addtocounter{page}{4}


\def\res{\mathop{\vert\grave{}}}
\def\rres{\mathop{\vert\hbox
to0pt{$\grave{}$\hss}\lower2pt\hbox{$\grave{}$}}}
\def\rrightarrow{\rightarrow \hspace{-.8em}
\rightarrow}
\def\ssubset{\subset \hspace{-0.5 em} \subset}
\def\insem{[ \hspace{-.23em} |}
\def\outsem{] \hspace{-.33em} |}
\def\ovi{\overline{\imath}}
\def\ovj{\overline{\jmath}}
\def\Rar{\Rightarrow}
\def\rar{\rightarrow}
\def\Lar{\Leftarrow}
\def\lar{\leftarrow}
\def\LRar{\Leftrightarrow}
\def\lrar{\leftrightarrow}
\def\ttT{{\tt True}~}
\def\ttF{{\tt False}~}
\def\isval{\hspace{.4ex} \stackrel{\it  val}{= \hspace{-.5ex} =} \hspace{.4ex}}
\def\grval{\hspace{.4ex} \mid \hspace{-.5ex} \stackrel{\it  val}{= \hspace{-.5ex} =} \hspace{.4ex}}
\def\klval{\hspace{.4ex} \stackrel{\it  val}{= \hspace{-.5ex} =} \hspace{-.5ex} \mid \hspace{.4ex}}
\def\rrar{\twoheadrightarrow}
\def\red{\rightarrow_{\beta}}
\def\rred{\twoheadrightarrow_{\beta}}
\def\led{\leftarrow_{\beta}}
\def\lled{\twoheadleftarrow_{\beta}}
\def\ded{\stackrel{\Delta}{\rightarrow}}
\def\dded{\stackrel{\raisebox{-0.5ex}{\scriptsize{$\Delta$}}}{\twoheadrightarrow}}
\def\cded{\stackrel{\Delta~}{\twoheadrightarrow_{\beta}}}
\def\conv{=_{\beta}}
\def\donv{\stackrel{\Delta}{=}}
\def\cdonv{\stackrel{\Delta~}{=_{\beta}}}
\def\cond{{\tt con}_{\Delta}}
\def\rard{\rar_\delta}
\def\rrard{\rrar_\delta}
\def\rrarbd{\rrar_{\beta\delta}}

\def\crar{\stackrel{\bf c~~}{\rar_{\beta_1}}}
\def\crrar{\stackrel{\bf c~~}{\rrar_{\beta_1}}}


\def\frar{\stackrel{\bf f~~}{\rar_{\beta_1}}}
\def\frrar{\stackrel{\bf f~~}{\rrar_{\beta_1}}}

\def\lmea{||\,}
\def\rmea{\,||_{\Delta}}
\def\seml{[\![}
\def\semr{]\!]}

\def\bbN{\mathbb{N}}
\def\bbZ{\mathbb{Z}}
\def\bbR{\mathbb{R}}
\def\bbNp{\bbN^+}
\def\isN{=_{\bbN}}
\def\isNp{=_{\bbN^+}}
\def\nisN{\not=_{\bbN}}
\def\nisNp{\not=_{\bbN^+}}

\def\lobar{\lambda \underline{\omega}}
\def\lamC{\lambda {\rm C}}
\def\lamD{\lambda {\rm D}}
\def\lamDu{\lambda {\rm D}_{0}}
\def\lamDn{\lambda {\rm D}_{\it n}}
\def\lamDp{\lambda {\rm D}^+}
\def\lamCp{\lambda {\rm C}^+}
\def\lamDz{\lambda {\rm D}_{0}}
\def\lamP{\lambda {\rm P}}

\def\subZ{_{\bbZ}}

\def\isdef{\stackrel{d}{=}}
\def\ddef{\,:=\,}
\def\pplus{\,+\,}
\def\ccirc{~\circ~}
\def\vvdash{~\vdash~}
\def\bott{\bot \hspace{-1.4ex} \bot}
\def\smallbott{\bot \hspace{-1.2ex} \bot}
\def\mmin{\text -}
\def\noi{\noindent\hspace{1.2ex}}
\def\noin{\noindent--\hspace{1.2ex}}
\def\nin{\noindent}
\def\inn{\!<\hspace{-1em}-\,}
\def\ins{\!<\hspace{-1em}-}
\def\inR{\!<\hspace{-4.2mm}-\,}
\def\inr{\!<\hspace{-3.5mm}-\,}
\def\inst{\stackrel{\raisebox{-3em}{--}}{<}}
%\newcommand{\ina}{~\epsilon~}
\newcommand{\ina}{~\varepsilon~}


\def\inw{\in^\wedge}
\def\inv{\in_\vee}
\def\inwv{\in^\wedge_\vee}
\def\Tcar{{\cal T}^{\it car}}
\def\Tcarp{{\cal T}^{\it car+}}
\def\Tfre{{\cal T}^{\it fre}}
\def\Tfrep{{\cal T}^{\it fre+}}

\def\rmA{{\rm A}}
\def\rmL{{\rm L}}
\def\rmS{{\rm S}}
\def\rmP{{\rm P}}
\def\bft{{\bf t}}
\def\bfs{{\bf s}}
\def\bfu{{\bf u}}
\def\bfv{{\bf v}}


%%\makeatletter
%%\newcommand{\superimpose}[2]{%
%% {\ooalign{$#1\@firstoftwo#2$\cr\hfil$#1\@secondoftwo#2$\hfil\cr}}}
%%\makeatother
%%\newcommand{\ina}{\,\mathpalette\superimpose{{<}{-}}\;}


\def\sup{}

\newcommand{\be}{\beta}
\newcommand{\la}{\lambda}
\newcommand{\de}{\delta}
\newcommand{\si}{\sigma}
\newcommand{\ze}{\zeta}
\newcommand{\zei}{\zeta_{\star}}
\newcommand{\up}{\upsilon}
\newcommand{\eps}{\in}
\newcommand{\body}{{\tt body}}
\newcommand{\loc}{\tt loc}
\newcommand{\glo}{\tt glo}
\newcommand{\da}{\dagger}
\newcommand{\lamdot}{ \, . \,\, }
\newcommand{\ao}{\{}
\newcommand{\as}{\}}
\newcommand{\nat}{{\it nat}}
\newcommand{\bool}{{\it bool}}
\newcommand{\omegab}{{\underline{\omega}}}
\newcommand{\llabel}[1]{\label{#1}}
%\newcommand{\llabel}[1]{\label{#1}\marginpar{\tiny{#1}}}
%\newcommand{\iindex}[1]{\index{#1}}
%%\newcommand{\iindex}[1]{\index{#1}\marginpar{\tiny{#1}}}

\newcommand{\cindex}[2]{\dindex{${#1}$,
Fig.\ref{#2}}}

%\newcommand{\cindex}[2]{\iindex{${#1}$, Fig.\ref{#2}}}

%%%%%%\newcommand{\aindex}[2]{\nindex{#2,\;#1}}


\newcommand{\nindex}[1]{\index{author}{#1}}
\newcommand{\dindex}[1]{\index{definition}{#1}}
\newcommand{\tindex}[1]{\index{technot}{#1}}
\newcommand{\sindex}[1]{\index{subject}{#1}}

\newcommand{\textps}{\textprimstress}
\newcommand{\textss}{\textsecstress}
\newcommand{\calD}{{\cal D}}
\newcommand{\calR}{{\cal R}}
\newcommand{\calT}{{\cal T}}
\newcommand{\calE}{{\cal E}}
\newcommand{\calED}{{\calE_{\lambda D}}}
\newcommand{\calEDn}{{\calE_{\lamDn}}}
\newcommand{\calEDu}{{\calE_{\lamDu}}}
\newcommand{\calEDp}{\calE_{\lamDp}}
\newcommand{\calB}{{\cal B}}
\newcommand{\calV}{{\cal V}}


\newcommand{\calAn}{{\cal A}_0}
\newcommand{\calAo}{{\cal A}_1}
\newcommand{\calAi}{{\cal A}_i}
\newcommand{\ovx}{\overline x}
\newcommand{\ovy}{\overline y}


\newcommand{\bif}{{\sf if}}
\newcommand{\Zero}{{\sf Zero}}
\newcommand{\bthen}{{\sf then}}
\newcommand{\belse}{{\sf else}}
\newcommand{\Mult}{{\sf Mult}}
\newcommand{\Pred}{{\sf Pred}}

\newcommand{\List}{{\sf List}}
%\newcommand{\nat}{{\sf nat}}
\newcommand{\IN}{{\mathbb N}}

\newcommand{\sss}{\scriptstyle\;}


\newcommand{\ttp}{\tt p}
\newcommand{\ttTy}{\tt T}
\newcommand{\ttA}{\tt A}
\newcommand{\ttl}{\tt l}

\newcommand{\renlab}{\renewcommand{\labelitemi}{$-$}}%%%{$\!-\!$}}
\newcommand{\listbegin}{\vspace{-1ex}\begin{itemize}\renlab}
\newcommand{\listend}{\end{itemize}\vspace{-1ex}}



\pagenumbering{arabic}

\section{Introduction}
\label{SecInt}

It is well known that practical implementations of $\lambda$-calculus
turn $\alpha$-equivalence into syntactic equality
by representing bound variable occurrences
with numbers rather than names (\citealp{deB72}).\marginpar{v5. \ref{SecInt}--1}
Generally speaking, function application involves replacing the
occurrences of the independent variables in the function's body
with copies of the function's arguments
and, in this scenario, the indexes occurring in such copies
may need an update to prevent captures.
Experience shows that this update,
known as lift according to a well-established terminology,
is a time consuming operation \cite[Appendix A2]{Gui09}
that, precisely, computing machines strive to avoid (\citealp{Klu05}).

In a family of systems originating from \cite{deB78b}
%\marginpar{v4. \ref{SecInt}--1}
and $\lambda\sigma$ (\citealp{Aba91}),
$\beta$-reductions insert information on updates in the copied terms
with depth indexes or level indexes (\citealp{deB72}), \marginpar{v5. \ref{SecInt}--2} being positive numbers. Therefore, the resulting systems are termed {\em name-free\/} as opposed to {\em name-carrying\/}.
Thus a computation can delay updates at will
or apply them whenever is the case.

The name-free system we present in this article for the first time
lies in this family. Contrary to its predecessors,
it is based just on $\beta$-reduction at a distance,
an extension of $\beta$-reduction introduced in \cite{Ned73}. %\marginpar{\ref{SecInt}--2}
Such a reduction relation allows, for example, not only $\beta$-reduction $K \equiv (\lambda x y \lamdot L) M N \rar (\lambda x \lamdot L[y := M])N$, but also $K \rar (\lambda y \lamdot L[x := N])M$.


In deviation of the usual notation, lambda terms %\marginpar{\ref{SecInt}--3}
are presented as trees composed of branches. With this representation, one obtains a transparent view on matching pairs of abstraction and application, each of which pairs may generate a beta-reduction. This transparency facilitates our discussion considerably.

The name-free system of lambda-calculus with %\marginpar{\ref{SecInt}--4}
which the paper concludes has the following properties:

-- Name-free variables are positive integers, as usual, and the employed reduction is $\beta$-reduction at a distance.

-- None of the variables disappears by a $\beta$-reduction: name-free variables have a fixed position in the lambda term, not altered by the reduction of the term.

-- As a consequence, some variables become embedded {\em inside\/} the term by $\beta$-reduction; such variables retain information on the delayed update.

-- If a lambda term $\beta$-reduces to another one, then the first term is {\em included\/} in the second one, i.e., the tree of the first term is a {\em subtree\/} of the tree of its reduct.

-- Each sequence of one-step $\beta$-reductions in the traditional lambda calculus corresponds one-to-one to a similar sequence in the new system, and vice versa.

-- A kind of `garbage collection' reduces a lambda term in the new system to one in the usual lambda calculus.

-- The described $\beta$-reduction obeys confluence.

-- Weak normalization of a lambda term in the new system implies strong normalization.

\smallskip


We divide our exposition into six parts. In Section~\ref{SecShoHis} we give %\marginpar{\ref{SecInt}--5}
a concise historic overview of updating in name-free beta reduction.
We recall name-carrying $\lambda$-calculus in Section~\ref{AutFor} % replace 1 with \ref{}
and name-free $\lambda$-calculus in Section~\ref{SecNamFre}. The main part of the paper is Section~\ref{SecDelUpd}. Here we describe our new system with delayed updating, for the case of name-free variables, and we compare the system with traditional lambda calculus.
Our concluding remarks are in Section~\ref{SecConRem}. % replace 5 with \ref{}
 % replace 2 with \ref{}
In Appendix~\ref{SecEleVar} % replace 3 with \ref{}
we present a variant of $\beta$-reduction for our new system,
while in Appendix~\ref{SecDelRed} % replace 4 with \ref{}
we shortly describe our system for the name-{\em carrying\/} case.
%\marginpar{v4. \ref{SecInt}--2}

%\medskip

%The majority of the theorems has been checked %\marginpar{\ref{SecInt}--6}
%by one of us (F.\ Guidi) with the theorem prover Matita (see \citealp{Gui14}), which was a time-consuming but interesting process. For the theorems concerned, we add references to the checked Matita-code, which can be retrieved on-line.


\section{A short history of updating in name-free beta-reduction}\label{SecShoHis}
%\marginpar{v4. \ref{SecShoHis}--1}

In name-carrying systems of $\lambda$-calculus
a binder of a term $M$, say $\lambda_x$, and the variable occurrences
that refer to it carry the same name, say $x$.
On the contrary, name-free systems use unnamed binders, say $\lambda$,
and replace a bound variable occurrence $x$
with an index that is a non-negative integer denoting the position of
the corresponding $\lambda_x$ along the path connecting $x$ to the
root of $M$ in the representation of $M$ as an abstract syntax tree.
As we pointed out in the introduction,
the $\beta$-reduction step of the latter systems
requires updating the indexes occurring in a copied
argument, say $N$, to maintain the relationship between the bound variable
instances in $M$ and the respective binders.
Depending on the particular system,
if {\em immediate updating\/} is in effect, the update occurs by
applying a so-called {\em update function\/} to the indexes in $N$.\marginpar{v5. \ref{SecShoHis}--1} On the
contrary, if {\em delayed updating\/} is in effect, the update function
is just stored in the syntax \marginpar{v5. \ref{SecShoHis}--2} of the copied $N$.

The first name-free systems with immediate updating appear in \cite{deB72}
with the basic update functions $\tau_{d,h}$ of type $\bbNp \to \bbNp$,
where $d\in\bbN$ and $h\in\bbN$.
\[
\tau_{d,h} \equiv i \mapsto
\left\{\begin{tabular}{ll}
$i$&if $i \le d$\\
$i+h$&if $i > d$\\
\end{tabular}\right.
\]
The systems accompanying\marginpar{v5. \ref{SecShoHis}--3} \cite{deB78b}
-- for instance those in \cite{deB77,deB78a} -- \marginpar{v5. \ref{SecShoHis}--4}
are the first to allow delayed updating
by featuring the term node $\phi(f)$
where $f$ is an arbitrary function of type $\bbNp \to \bbNp$. The original purpose of $\phi(f)$ is to present substitution as a single operation defined by recursion on the structure of terms.


Other systems of the same family, like \cite{Ned79,Ned80,KN93},
feature the term node $\mu(d,h)$ or $\phi^{(d,h)}$
that holds the function $\tau_{d,h}$.
Moreover, the systems originating from \cite{Aba91}
-- for instance those in \citealp{CHL96}, 1996 -- \marginpar{v5. \ref{SecShoHis}--5}
feature the explicit substitution constructors $\mathit{id}$ and $\uparrow$
that essentially hold the functions $\tau_{0,0}$ (the identity)
and $\tau_{0,1}$ (the successor) respectively.

The system we are going to introduce takes a simpler approach.
Its syntax has a term node $p$ we call {\em inner numeric label\/}
where $p \in \bbNp$.
An active $p$ holds the function $\tau_{0,p}$
while a passive, i.e.\ present but ignored, $p$ holds the function $\tau_{0,0}$.
In some sense, we want to show that supporting the functions $\tau_{0,h}$
suffices to implement delayed updating in basic name-free $\lambda$-calculus.


\section{Name-carrying lambda trees}\label{AutFor}

\subsection{Lambda-trees}

We use a tree representation for lambda terms in (untyped or typed) lambda calculus. See Figure~\ref{TreLamTer}. The obtained tree is an undirected rooted tree with labels attached to the {\em edges\/} of the tree. We prefer labelling the edges instead of the vertices for reasons to be explained below.

Label ${\rm L}$ represents a `lambda' and ${\rm A}$ stands for an `application'. Moreover, we use label ${\rm S}$ for `subordination'; this extra label enables us to discriminate between main branches and subbranches, which is essential when considering different pathes in a tree.

\smallskip

We assume that an infinite set of {\em variables\/}, ${\calV}$, is available.

%{\bf In some typed calculi: also ${\rm P}_x$ is a label, for each variable $x$; it represents $\Pi$-binding of $x$. In such calculi, $\ast$ can be a label, as well.}\marginpar{v5. \ref{AutFor}--1}

%Instead of using variables ($x$) and binders ($\lambda$), we employ the name-free notation due to N.G.\ de Bruijn (\citealp{-}). The $v$ stands for a positive natural number representing a variable. (Explanation follows below.)

%Note that the positions of `function' $N$ and argument $M$ are reversed, so we write $M \, N$ instead of $N \, M$ or $N(M)$. So we follow the `argument-prior-to-function convention', also due to de Bruijn. We use $\langle t \rangle$ for denoting the tree of lambda term $t$.



%NOTE: When considering typed lambda calculi containing $\Pi$-terms: the tree representation of $\Pi x : M \lamdot N$ is similar to that of $\lambda x : M \lamdot N$, with ${\rm P}_x$ instead of ${\rm L}_x$. In some typed calculi, $\ast$ occurs as a constant.

%\medskip

%${\rm L}$ means `lambda', ${\rm A}$ means `application' and ${\rm R}$ means `restriction' ($x$ is restricted to type $M$ in $\lambda$-abstraction, function $M$ is restricted to argument $N$ in application).



%\subsection{Name-carrying lambda-trees}

%\medskip

\begin{Def}

{\em tree\/}: connected acyclic undirected graph, consisting of vertices and labelled edges.

{\em labels\/}: symbols ${\rm L}_x$ (for each variable $x \in V$), A, S, or a variable name.\marginpar{v5. \ref{AutFor}--1}
%{\bf (${\rm L}_x$ and A represent $\lambda$-binding of $x$, and application, respectively. Label S represents the beginning of a subexpression in $\lambda$-calculus.)}



%{\bf {\em binders\/}: ${\rm L}_x$ and ${\rm P}_x$ are binders. These symbols are intended to bind free variables (e.g., $x$), in the usual manner. We omit the details.}

%{\em edge-labelled tree\/}: a tree in which the leaf vertices are omitted and in which each edge has a label.

{\em path\/}: a connected string of labelled edges occurring in a tree.
\end{Def}

\begin{Def}

{\em Meta-variables for trees\/}: ${\bf t}$, ${\bf t}'$, ${\bf t}_1$, \ldots;
{\em for labels\/}: $\ell$, \ldots;
%{\bf
%{\em for binders\/}: $B_x$, $B'_x$, \ldots};
{\em for paths\/}: $p$, $q$, \ldots.

\end{Def}

\begin{Con}\label{ConPat}

%$(i)$ All trees and paths are edge-labelled.

%$(ii)$
 Each {\em path\/} in a tree will be {\em identified\/} with its string of labels.
\end{Con}

%\smallskip


\begin{Def} Let $p$ and $q$ be paths. Then $p \preceq q$ if there is a (possibly empty) path $p'$ obeying $q \equiv p \, p'$, and $p \prec q$ if there is a non-empty path $p'$ with that property.
\end{Def}

\begin{Def}

A-cell, $\rmL$-cell, var-cell: see Figure~\ref{TreLamTer}.
%; P-cell, $\ast$-cell


\end{Def}

\begin{figure}[h]

\begin{picture}(350,75)(-40,0)
\put(26,70){$\langle M  N \rangle \; =$}

\put(30,57){\circle*{4}}
\put(30,57){\line(1,-1){30}}
\put(30,57){\line(0,-1){30}}
%\put(215,32){\oval(30,30)[bl]}
%\put(214,17){\line(1,0){12}}
\put(53,36){${\rm S}$}
\put(19,36){${\rm A}$}
\put(22,17){$\langle M \rangle$}
\put(53,17){$\langle N \rangle$}

\put(32, 0){A-cell}

\put(100,70){$\langle\lambda x : M \lamdot N \rangle \; =$}

\put(120,57){\circle*{4}}
%\put(120,57){\line(1,-1){30}}
\multiput(120,57)(5,-5){8}{\circle*{1.3}}
%\multiput(118.5,58.5)(11,-11){3}{\line(1,-1){10}}
\put(120,57){\line(0,-1){30}}
%\put(100,32){\oval(30,30)[bl]}
%\put(99,17){\line(1,0){12}}
\put(143,36){${\rm S}$}
\put(107,36){${\rm L}_x$}
\put(112,17){$\langle N \rangle$}
\put(143,17){$\langle M \rangle$}

\put(122, 0){{$\rmL$}-cell}


%\put(220,70){$\langle\Pi x : M \lamdot N \rangle \; =$}

%\put(240,57){\circle*{4}}
%\put(240,57){\line(1,-1){30}}
%\put(240,57){\line(0,-1){30}}
%\put(263,36){${\rm S}$}
%\put(227,36){${\rm P}_x$}
%\put(232,17){$\langle N \rangle$}
%\put(263,17){$\langle M \rangle$}

%\put(242, 0){P-cell}

\put(197,70){$\langle x \rangle \;=$}

\put(210,55){\circle*{4}}
\put(210,55){\line(0,-1){30}}
\put(200,36){$x$}

\put(197, 0){var-cell}

%\put(33, -19){\bf {Cells for untyped $\lambda$-calculus}}
%\put(220, -19){\bf {Only in the typed case}}

%\put(205, -22){\line(0,1){97}}
%\put(207, -22){\line(0,1){97}}

%\put(304,70){$\langle \ast \rangle \;=$}

%put(316,55){\circle*{4}}
%put(316,55){\line(0,-1){30}}
%put(306,36){$\ast$}

%put(306,0){$\ast$-cell}
\end{picture}

\caption{cells of lambda trees}
\label{TreLamTer}
\end{figure}

\begin{Exp}

$(i)$ An {\em $\rmA$-cell\/} initializes an {\em application\/}, of {\em function\/} $M$\marginpar{v5. \ref{AutFor}--2} to {\em argument\/} $N$. Below the $\rmA$-edge the tree representation of function $M$ must be attached, below the $\rmS$-edge that of the argument $N$.
Clearly, an $\rmA$-cell is {\em binary\/}.
%Note that the positions of `function' $M$ and argument $N$ are reversed, according to de Bruijn's view to write $N \, M$ instead of $M \, N$ (the `argument-prior-to-function convention').

$(ii)$  Each {\em $\rmL$-cell\/} starts an {\em abstraction\/}, abstracting the {\em function body\/} $N$, possibly containing $x$ as a free variable, over that $x$. Below the $\rmL_x$-edge, the representation of the body $N$ is expected. This kind of cell depends on the name of the variable. That is to say, for another variable, say $y$, an $\rmL_y$-cell can be constructed in a similar manner.

As to the $\rmS$-edge in an $\rmL$-cell, and the {\em type\/} $M$ of $x$, there are two options.

-- In {\em untyped\/} $\lambda$-calculus, a variable $x$ has no type at all, so the type $M$ is absent. Consequently, the $\rmS$-edge should be suppressed completely (indicated by the dots in the drawn $\rmS$-edge).

-- In the {\em typed\/} $\lambda$-calculus, however, there must be an $\rmS$-edge, enabling one to attach the type $M$ of $x$.

Hence, an $\rmL$-cell is either unary (in the untyped case) or binary (in the typed case).

$(iii)$ A {\em var-cell\/} represents a single variable. Such a cell is always unary. A var-cell depends on the variable being the label. So there is an infinity of different var-cells.

\end{Exp}

Now we comment on how these three kinds of cells are {\em used\/} to construct so-called {\em $\lambda$-trees\/}, which are aimed to be structured representations of $\lambda$-terms.

We first note that the roots of cells are always vertices. But at the lower sides, $\rmA$-cells and $\rmL$-cells have so-called {\em open ends\/}. The idea is, that these open ends enable to connect these cells {\em downwards\/} to roots of other $\rmA$- or $\rmL$-cells or the root of a var-cell; and {\em upwards\/} to open ends of other $\rmA$- and $\rmL$-cells.

%{\em $\rmA$-, $\rmL$- and $\rmP$-cells\/} are binary; their roots are always a vertex, but they have no roots at the lower sides: these are

Note, however, that var-cells can only be {\em upwards\/} connected to A- or ${\rmL}$-cells, but they can not be {\em downwards\/} connected to any other cell: a var-cell does {\em not\/} have an open end.

%\begin{Note}\label{Lcel}
%In {\em untyped\/} lambda-calculus, $\rmL$-cells and $\rmP$-cells are {\em unary\/}, only having edges labelled $\rmL$ or $\rmP_x$. The edge labelled $\rmS$ is absent. These cells obviously have only {\em one\/} open end. This is because the types $M$ are absent in an untyped calculus.
%\end{Note}

\begin{Def}
A $\lambda$-{\em tree\/} is a non-empty, rooted and edge-labelled tree, built by an arbitrary connection of $\rmL$-cells, A-cells and var-cells, such that there are no open ends. (So every leaf is a var-cell and var-cells appear only at the leaves.)

%A-, L- and P-cells in $\lambda$-treed are downwards connected to other cells at both open ends; but neither var-cells nor $\ast$-cells are connected in the downward direction.

Each $\lambda$-tree apparently has a top-cell. The {\em root\/} of a $\lambda$-tree is the vertex of its top-cell.
\end{Def}

\begin{Note}\label{NamCarNoo}
In many {\em typed\/} lambda calculi, one has\marginpar{v5. \ref{AutFor}--3} -- separated from the variable set $\calV$ --  a set of {\em type variables\/}, say $\calT$.Often there also is a binder for these type variables, usually denoted by symbol $\Pi$.
Moreover, in these calculi one may have one or more {\em constants\/}, such as $\ast$, which denotes the type of type variables.

If we want to incorporate such kinds of  devices into our tree representation of $\lambda$-terms, we have to extend the set of labels with e.g.\ $\rmP_x$ (for each type variable $x$) to represent $\Pi$, and add names of constants to our labels, e.g.\ $\ast$.

Next, we have to add $\rmP$-cells and, e.g., $\ast$-cells, to our syntactic tools. It will be obvious how to implement these new cells into our system.

%$(i)$ For terms of the form $\Pi x : M \lamdot N$, we add $\rmP$-cells, being quite similar to $\rmL$-cells, with the only difference that $\rmP_x$ replaces $\rmL_x$. Also for $\rmP$-cells, the $\rmS$-edge is absent in case type variables have no types themselves.

%$(ii)$ For each constant we add a separate cell, having the same format as var-cells have. For example, for the constant $\ast$ we may add a $\ast$-cell, similar to a var-cell, but for the label $\ast$ replacing the label $x$.
\end{Note}

The following example shows how a term from {\it typed\/} lambda calculus can be pictured by means of the mentioned cells. Note that we also need a $\rmP$-cell and $\ast$-cells (cf. Note~\ref{NamCarNoo}).

\begin{Exa}\label{NamCarExa}
In Figure~\ref{ExaLamTre} (1) we picture the $\lambda$-tree of the following term from typed lambda calculus:

$\lambda \alpha : \ast \lamdot \lambda \beta : \ast \lamdot \lambda x : \alpha \lamdot \lambda y : \alpha \rar \beta \lamdot y \, x$.

The tree marked (1) is written in the {\em name-carrying\/} notation, using the {\em variable names\/} $x$, $y$, $\alpha$ and $\beta$. The notation $\alpha \rar \beta$ is treated as an abbreviation of $\Pi u : \alpha \lamdot \beta$.
%(The symbol $\ast$ is a {\em constant}\/ of typed $\lambda$-calculus.)

(The tree in Figure~\ref{ExaLamTre} (2) will be discussed later.)\marginpar{v5. \ref{AutFor}--4}

\end{Exa}





%\newpage

\begin{figure}[h]

\begin{picture}(250,230)(-30,25)
\put(0,250){(1) name-carrying}

\put(20,40){\circle*{4}}
\put(20,60){\circle*{4}}
\put(20,120){\circle*{4}}
\put(20,160){\circle*{4}}
\put(20,200){\circle*{4}}
\put(20,240){\circle*{4}}
\put(40,40){\circle*{4}}
\put(40,80){\circle*{4}}
\put(40,100){\circle*{4}}
\put(40,140){\circle*{4}}
\put(40,180){\circle*{4}}
\put(40,220){\circle*{4}}
\put(60,80){\circle*{4}}

\put(20,60){\line(1,-1){20}}
\put(20,120){\line(1,-1){40}}
\put(20,160){\line(1,-1){20}}
\put(20,200){\line(1,-1){20}}
\put(20,240){\line(1,-1){20}}
\put(20,60){\line(1,-1){20}}

\put(20,20){\line(0,1){220}}
\put(40,60){\line(0,1){40}}
\put(40,20){\line(0,1){20}}
\put(40,120){\line(0,1){20}}
\put(40,160){\line(0,1){20}}
\put(40,200){\line(0,1){20}}
\put(60,60){\line(0,1){20}}

\put(31,51){${\rm S}$}
\put(31,111){${\rm S}$}
\put(31,151){${\rm S}$}
\put(31,191){${\rm S}$}
\put(31,231){${\rm S}$}

\put(10,47){${\rm A}$}
\put(8,100){${\rm L}_y$}
\put(8,140){${\rm L}_x$}
\put(8,180){${\rm L}_\beta$}
\put(8,220){${\rm L}_\alpha$}

\put(12.5,28){$y$}
\put(32.5,28){$x$}
\put(32.5,68){$\beta$}
\put(52.5,68){$\alpha$}
\put(32.5,128){$\alpha$}
\put(32.5,168){$\ast$}
\put(32.5,208){$\ast$}

\put(51,91){${\rm S}$}
\put(27,87){${\rm P}_{u}$}


\put(105,250){(2) name-free}

\put(120,40){\circle*{4}}
\put(120,60){\circle*{4}}
\put(120,120){\circle*{4}}
\put(120,160){\circle*{4}}
\put(120,200){\circle*{4}}
\put(120,240){\circle*{4}}
\put(140,40){\circle*{4}}
\put(140,80){\circle*{4}}
\put(140,100){\circle*{4}}
\put(140,140){\circle*{4}}
\put(140,180){\circle*{4}}
\put(140,220){\circle*{4}}
\put(160,80){\circle*{4}}

\put(120,60){\line(1,-1){20}}
\put(120,120){\line(1,-1){40}}
\put(120,160){\line(1,-1){20}}
\put(120,200){\line(1,-1){20}}
\put(120,240){\line(1,-1){20}}
\put(120,60){\line(1,-1){20}}

\put(120,20){\line(0,1){220}}
\put(140,60){\line(0,1){40}}
\put(140,20){\line(0,1){20}}
\put(140,120){\line(0,1){20}}
\put(140,160){\line(0,1){20}}
\put(140,200){\line(0,1){20}}
\put(160,60){\line(0,1){20}}

\put(131,51){${\rm S}$}
\put(131,111){${\rm S}$}
\put(131,151){${\rm S}$}
\put(131,191){${\rm S}$}
\put(131,231){${\rm S}$}

\put(110,47){${\rm A}$}
\put(110,100){${\rm L}$}
\put(110,140){${\rm L}$}
\put(110,180){${\rm L}$}
\put(110,220){${\rm L}$}

\put(112.5,28){$1$}
\put(132.5,28){$2$}
\put(132.5,68){$3$}
\put(152.5,68){$3$}
\put(132.5,128){$2$}
\put(132.5,168){$\ast$}
\put(132.5,208){$\ast$}

\put(151,91){${\rm S}$}
\put(130,87){${\rm P}$}

%\put(205,250){(3) skeleton}

%\put(220,40){\circle*{4}}
%\put(220,60){\circle*{4}}
%\put(220,120){\circle*{4}}
%\put(220,160){\circle*{4}}
%\put(220,200){\circle*{4}}
%\put(220,240){\circle*{4}}
%\put(240,40){\circle*{4}}
%\put(240,80){\circle*{4}}
%\put(240,100){\circle*{4}}
%\put(240,140){\circle*{4}}
%\put(240,180){\circle*{4}}
%\put(240,220){\circle*{4}}
%\put(260,80){\circle*{4}}

%\put(220,60){\line(1,-1){20}}
%\put(220,120){\line(1,-1){40}}
%\put(220,160){\line(1,-1){20}}
%\put(220,200){\line(1,-1){20}}
%\put(220,240){\line(1,-1){20}}
%\put(220,60){\line(1,-1){20}}

%\put(220,40){\line(0,1){200}}
%\put(140,20){\line(0,1){20}}
%\put(240,80){\line(0,1){20}}
%\put(140,120){\line(0,1){20}}
%\put(140,160){\line(0,1){20}}
%\put(140,200){\line(0,1){20}}
%\put(160,60){\line(0,1){20}}

%\put(231,51){${\rm S}$}
%\put(231,111){${\rm S}$}
%\put(231,151){${\rm S}$}
%\put(231,191){${\rm S}$}
%\put(231,231){${\rm S}$}

%\put(210,47){${\rm A}$}
%\put(210,100){${\rm L}$}
%\put(210,140){${\rm L}$}
%\put(210,180){${\rm L}$}
%\put(210,220){${\rm L}$}

%\put(117.5,32){$2$}
%\put(137.5,32){$1$}
%\put(137.5,72){$4$}
%\put(157.5,72){$2$}
%\put(137.5,132){$2$}
%\put(137.5,172.3){$\ast$}
%\put(137.5,212.3){$\ast$}

%\put(251,91){${\rm S}$}
%\put(230,87){${\rm P}$}
\end{picture}

\caption{Lambda trees}
\label{ExaLamTre}
\end{figure}

Paths in $\lambda$-trees are essential elements of our forthcoming discussion, in particular regarding several kinds of {\em reduction\/}. We give names to particular paths in the following definition.


\begin{Def}
Let ${\bf t}$ be a $\lambda$-tree.

We write $p \in {\bf t}$ if path $p$ is a (coherent) part of ${\bf t}$.

A {\em root path\/} $p$ in ${\bf t}$ is a (non-empty) path starting in the root. Notation: $p \in^\wedge {\bf t}$.

A {\em leaf path\/} in ${\bf t}$ is a (non-empty) path with a leaf as final label. Notation: $p \in_\vee {\bf t}$.

A {\em complete path\/} in ${\bf t}$ is a path that is both a root path and a leaf path. Notation: $p \in^\wedge_\vee {\bf t}$.

\end{Def}

\begin{Exa}\label{ExaComPat}
In the $\lambda$-tree of Example~\ref{ExaLamTre}\,(1), the path $\rmL_\alpha \, \rmL_\beta \, \rmL_x \, \rmS \, \rmP_u \, \beta$ is a complete path.
\end{Exa}


For the sake of convenience, we assume that the binding variables in a $\lambda$-tree always differ from each other. Let's assume, for example, that we consider typed lambda calculus and that the set $\calB$ of all binders is $\{\rmL_x | x \in \calV\} \cup \{\rmP_x | x \in \calT\}$. Then we obey the following convention.

\begin{Con}
Let ${\bf t}$ be a $\lambda$-tree, let $B_x$ and $B'_y \in \calB$ and let the paths $p \, B_x$ and $q \, B'_y \in^\wedge {\bf t}$. Then $ x \equiv y ~\Rar~ (p \equiv q \wedge B \equiv B')$.
\end{Con}

Hence, in particular: If $B_x$ is a binder in ${\bf t}$, then there is no other binder with subscript $x$ in ${\bf t}$.

\begin{Def}
$(i)$ Let $p_1$, \ldots, $p_n$ be all complete paths in a $\lambda$-tree ${\bf t}$. Then we  identify ${\bf t}$ with the set $\{p_1, \ldots, p_n\}$.

$(ii)$ Let $\bft \equiv \{p_1,\ldots,p_m\}$ and $\bft' \equiv \{q_1,\ldots,q_n \}$ be $\lambda$-trees. Then $\bft$ is a {\em subtree\/} of $\bft'$, or $\bft \subseteq \bft'$, if for each $p_i \inwv \bft$ there exists a path $q_j \inwv \bft'$ such that $p_i \preceq q_j$.
\end{Def}


\begin{Rem}
From now on, we focus on the {\bf untyped} version of $\lambda$-calculus, since we concentrate on various forms of $\beta$-reduction and types play no role in $\beta$-reduction. Most of the material presented here below, however, also applies to the {\em typed}\/ version of $\lambda$-calculus, with some adaptations. %For example, in the untyped case the righthand branching of an L-cell should be left out, just as all the P-cells and $\ast$-cells.
\end{Rem}

%\begin{Def} The labels $\rmA$, $\rmL_x$ and $\rmP_x$, for any variable $x$, are {\em binary labels\/}. The labels $x$, for any variable $x$, and $\ast$, are {\em unary labels\/}.
%\end{Def}

%\begin{Def} Let $\bft$ be a $\lambda$-tree.

%Let $p \inw \bft$ and assume that also $p \, \ell \inw \bft$, for some label $\ell$. We say that $p$ {\em is binary\/} in $\bft$ if $\ell$ is a binary label and {\em unary\/} in $\bft$ if $\ell$ is a unary label.

%$(ii)$ The path $p \inw \bft$ is called {\em nullary\/}, if $p$ ends in a variable or $\ast$.


%\end{Def}

%A $\lambda$-tree can be characterized as follows.

%\begin{Lem} Let ${\bf s} \equiv p_1,\ldots,p_n$ be a non-empty set of non-empty, distinct  strings of labels.

%${\bf s}$ is a $\lambda$-tree iff the following condition holds:

%For all $p_i$ and all (possibly empty) paths $p \prec p_i$, either $(i)$ or $(ii)$ holds:

%\noindent $(i)$ $p \, \rmS \preceq p_i$ and $p \, \ell \prec p_i$ for a {\em unique\/} label $\ell$, where $\ell$ is not $\rmS$, $\ast$ or some variable $x$ (i.e., if also $p \, \ell' \preceq p_i$ for $\ell' \not\equiv \rmS$, then $\ell' \equiv \ell$).

%\noindent $(ii)$ $p_i \equiv p \, x$ or $p_i \equiv p \, \ast$.

%$(ii)$ Either $(ii)'$ or $(ii)''$ holds:

%\noindent $(ii)'$ A path $p_i$ begin with either the label $\rmS$ or a {\em unique\/} label $\ell \not\equiv \rmS$ (i.e., if $p_i$ begins with $\ell \not\equiv \rmS$ and $p_j$ with $\ell' \not\equiv \rmS$, then $\ell \equiv \ell'$).

%\noindent $(ii)''$ $\bfs \equiv \{x\}$ for some $x$ or $\bfs \equiv \{\ast\}$.
%\end{Lem}







\subsection{$\beta$-reduction on $\lambda$-trees}

The relation $\beta$-reduction between $\lambda$-terms in typed lambda calculi is defined as the compatible relation generated by

\smallskip

$(\lambda x : K \lamdot M)P \red M[x := P]$.

\smallskip

Here $M [x := P]$ is the result of substituting $P$ for all free $x$'s in $M$. (We do not give the necessary conditions on the names of variables which prevent undesired bindings in the `process' of $\beta$-reduction.)

\smallskip

We give another description of $\beta$-reduction, suited to the {\em tree format\/} sketched previously and facilitating the alternative $\beta$-reductions described below. We first give some definitions, and a lemma on binding.

\begin{Def}\label{DefGraTre}
Let ${\bf t}$ be a $\lambda$-tree and $p \in^\wedge {\bf t}$.

Consider the set $S \subseteq \bft$ of all complete paths $p \, q \, x \in^\wedge_\vee{\bf t}$, so the paths beginning with $p$, ending in some leaf $x$ and having some path $q$ in between. The set of all these $q \, x$ is itself a tree, called {\it tree}$(p)$.



We call the set $S$ the {\em grafted tree\/} of $p$ in ${\bf t}$.
We denote this grafted tree by $p \, {\bf t}'$, if ${\bf t}'$ is {\it tree}$(p)$.

%We use the notation $p \, {\bf t}' \in^\wedge_\vee {\bf t}$ for this situation.

%{\em Metavariable for grafted trees\/}: ${\bf g}$.

\end{Def}

\begin{Def}\label{DefSubTre}
Let ${\bf t}$ and ${\bf t}'$ be $\lambda$-trees.

(1) Let $p \, y \in^{\wedge}_{\vee} {\bf t}$.
Then $(p \, y)[x := {\bf t}']$ is defined as the grafted tree $p \, {\bf t}'$ if $x \equiv y$, and as $p \, y$ if $x \not\equiv y$.

(2) We define ${\bf t} [x := {\bf t}']$ as $\{q [x := {\bf t}'] ~|~ q \in^\wedge_\vee {\bf t} \}$.

(3) Let $p \inw {\bf t}$ but $p \not\inwv {\bf t}$. Then ${\bf t} [{\it tree}(p) := {\bf t}']$ is the tree obtained from ${\bf t}$ by replacing the grafted tree $p \, {\it tree}(p)$ by the grafted tree $p \, {\bf t}'$.
\end{Def}

\begin{Lem}\label{LemTreBou} Let ${\bf t}$ be a $\lambda$-tree.

$(i)$  Assume that $p\, x \in^\wedge_\vee {\bf t}$. If $x$ is a variable in ${\bf t}$ that is bound in the original $\lambda$-term by a $\lambda$, then there is exactly one ${\rm L}_x \in {\bf t}$ binding this $x$, and $p \equiv p_1 \, {\rm L}_x \, p_2$ for some paths $p_1$ and $p_2$.

$(ii)$ Assume that $p \, {\rm L}_x \in^\wedge {\bf t}$. If $p \, {\rm L}_x \, q \, x \in^\wedge_\vee {\bf t}$, then $x$ is bound by ${\rm L}_x$. In a {\em bound\/} term, all $x$'s are bound by this ${\rm L}_x$, so there are no $x$'s `outside' ${\it tree}(p \, {\rm L}_x)$.
\end{Lem}



Hence, the binder of a variable $x$ in ${\bf t}$ can be found on the path leading backwards from $x$ to the root of ${\bf t}$. (A similar lemma holds for variables bound by a $\Pi$ in the original $\lambda$-term.)\marginpar{v5. \ref{AutFor}--5}
Moreover, an ${\rm L}_x$ in ${\bf t}$ binds all $x$'s that occur in the tree `below' it.

\begin{Def}\label{DefLBlo}
$(i)$ Let ${\bf t}$ be a $\lambda$-tree and let $p \, {\rm L}_x \, q \, x \inwv {\bf t}$, such that ${\rm L}_x$ binds~$x$. Then the path ${\rm L}_x \, q \, x$ is called the {\bf {\rm L}-block} {\em of (this occurrence of) $x$}.

$(ii)$ A $\lambda$-tree is called {\it closed\/} if all leaf variables $x$ are bound by an ${\rm L}_x$.
\end{Def}

Obviously, in a closed $\lambda$-tree, every leaf variable $x$ corresponds to exactly one ${\rm L}$-block ending in that $x$.

\medskip

We can describe {\em $\beta$-reduction\/} of a $\lambda$-tree ${\bf t}$, with relation symbol $\red$, as follows.

\begin{Def}\label{DefBetTre}

Let ${\bf t}$ be a $\lambda$-tree and assume that $p \, \rmA \, \rmL_x \in^\wedge {\bf t}$.

Now ${\bf t} ~\red~ {\bf t}[{\it tree}(p) := {\it tree}(p \, \rmA \, \rmL_x)[x := {\it tree}(p \, \rmS)]]$.
\end{Def}

Again, we disregard some necessary conditions on the variable names. Note: if $p \, \rmA \, \rmL_x \in^\wedge {\bf t}$, then also $p \, \rmS \in^\wedge {\bf t}$.

%\smallskip

%We conclude this section with some remarks about the `natural' order between paths in a tree ${\bf t}$.

%\begin{Def}\label{DefPatOrd} $(i)$ The labels are ordered according to the order relation generated by ${\rm A} < {\rm S}$, ${\rm L}_x < {\rm S}$ and ${\rm P}_x < {\rm S}$, for all $x$.

%$(ii)$ Let $p, \, q \inwv {\bf t}$. Then $p$ and $q$ are ordered by the lexicographic ordering generated by $<$.

%$(iii)$ The smallest complete path in a $\lambda$-tree ${\bf t}$, with respect to the order $<$, is called the {\em spine\/} of $\bf t$.
%\end{Def}

%It will be clear that the spine of a $\lambda$-tree does not contain the label $\rmS$.

%\begin{Exa} In Figure~\ref{ExaLamTre}, (1), we have $\rmL_\alpha \, \rmL_\beta \, \rmL_x \, \rmS \, \rmP_u \, \alpha \, < \, \rmL_\alpha \, \rmL_\beta \, \rmS \, \alpha$.

%The spine of the $\lambda$-tree in that example is $\rmL_\alpha \, \rmL_\beta \, \rmL_x \, \rmL_y \, \rmA  \, x$.
%\end{Exa}

\subsection{Some variants of beta-reduction}\label{SecVarBet}

\subsubsection{Balanced beta-reduction}

There is a variant of $\beta$-reduction that is interesting for certain purposes. We  call it {\em balanced $\beta$-reduction\/}. In the literature, it originally appeared under the name $\beta_1$ (\citealp{Ned73}). For details, see the more recent literature about the Linear Substitution Calculus (cf., \citealp{AccKes}) and \citealp{AccKes12}, in which it is called {\em distant beta\/}, symbol $\rar_{dB}$. See also \citealp{BarBon}.
%$\{$To be added: more references plus discussion about the literature$\}$

\smallskip

Firstly, we give the following definition.

\begin{Def}\label{DefBalPat}
A path $p$ in a $\lambda$-tree ${\bf t}$ is called {\em balanced\/}, denoted ${\it bal}(p)$, if it is constructed by means of the following inductive rules:

$(i)$ ${\it bal}(\varepsilon)$, i.e., the empty string is balanced;

$(ii)$ if ${\it bal}(p)$, then ${\it bal}(\rmA \, p \, \rmL_x)$, for every variable $x$;

$(iii)$ if ${\it bal}(p)$ and ${\it bal}(q)$, then ${\it bal}(p \, q)$.

\noindent In case~$(ii)$, we say that the mentioned $\rmA$ {\it matches\/} the mentioned ${\rmL}$.
\end{Def}

Examples of balanced paths: $\varepsilon$, $\rmA \, \rmL$, $\rmA \, \rmA \, \rmL \, \rmL$, $\rmA \, \rmL \, \rmA \, \rmL$, $\rmA \, \rmA \, \rmL \, \rmA \, \rmA \, \rmL \, \rmL \, \rmL$.

Note the close correspondence between nested paths and (consecutive) nested pairs of parentheses. Only $\rmA$- and $\rmL$-labels occur in balanced paths, so there is no other label involved, such as $\rmS$.

\smallskip

We define {\it balanced\/} $\beta$-reduction, with symbol $\rar_b$, as follows.

\begin{Def}\label{DefBalRed}
Let ${\bf t}$ be a $\lambda$-tree, let $b \in {\bf t}$ be a balanced path and assume that $p \, \rmA \, b \, \rmL_x \in^\wedge {\bf t}$ and there is at least one path $p \, \rmA \, b \, \rmL_x \, q \, x \inwv {\bf t}$.

Then ${\bf t} ~\rar_b~ {\bf t}[{\it tree}(p \, \rmA \, b \, \rmL_x) := {\it tree}(p \, \rmA \, b \, \rmL_x)[x := {\it tree}(p \, S)]]$.
\end{Def}

%So, in particular: ${\bf t} ~\rar_b~ {\bf t}[{\it tree}(p \, A  \, L_x) := {\it tree}(p \, A \, L_x)[x := {\it tree}(p \, S)]]$.

Compare this with Definition~\ref{DefBetTre}.

Consider two $\lambda$-trees ${\bf t}$ and ${\bf t}'$ such that ${\bf t} \rar_b {\bf t}'$ as described in Definition~\ref{DefBalRed}, so each $x$ has been replaced by ${\it tree}(p \, S)$ in ${\it tree}(p \, \rmA \, b \, \rmL_x)$. Now we have that $\bft$ is a subtree of ${\bft}'$, provided that we omit all var-labels $x$ in ${\bf t}$. So, balanced $\beta$-reduction has the property that it {\it extends\/} the original underlying tree $\bft$ if all instances of variable $x$ are skipped in $\bft$. (In the `tree-like' image of $\bft$, we have to skip the edges of the $x$-labels, as well.)

\smallskip

We now give the usual definition of {\em redex\/} (i.e., {\em red\/}ucable {\em ex\/}pression) and some notions connected with that. Compare this definition with the condition stated in Definition~\ref{DefBalRed}.

\begin{Def}\label{DefRedExp}
Let ${\bf t}$ be a $\lambda$-tree, let $b \in {\bf t}$ be a balanced path and assume that $p \, \rmA \, b \, \rmL_x \in^\wedge {\bf t}$ and there is at least one path $p \, \rmA \, b \, \rmL_x \, q \, x \inwv {\bf t}$.

Then ${\it tree}(p)$ is a {\em redex\/}, ${\it tree}(p \, \rmA \, b \, \rmL_x)$ is the {\em body\/} of the redex and ${\it tree}(p \, S)$ is the {\em argument\/} of the redex.
\end{Def}

Apart from the ${\rm L}$-block as described in Definition~\ref{DefLBlo}, there are two other kinds of paths that we will call {\em blocks}: {\em ${\rm A}$-blocks\/} and {\em ${\rm r}$-blocks}. See the following definition.

\begin{Def}\label{DefABlo}
Let ${\bf t}$ be a $\lambda$-tree, $b \in {\bf t}$ a balanced path and assume that $\rmA \, b \, \rmL_x \in {\bf t}$. Let $\rmA \, b \, \rmL_x \, q \, x$ be a path in {\bf t}.

$(i)$ The path $\rmA \, b \, \rmL_x \, q \, x \inv {\bf t}$ (where ${\rm L}_x$ binds $x$) is called the {\bf ${\rm A}$-block} {\em of (this occurrence of) $x$}.

$(ii)$ An ${\rm A}$-block $\rmA \, b \, \rmL_x \, q \, x \inv {\bf t}$ is a front extension of the ${\rm L}$-block $\rmL_x \, q \, x$; the mentioned ${\rm A}$-block and the ${\rm L}$-block are called {\em corresponding\/}.

$(iii)$ The path $\rmA \, b \, \rmL_x$ is called the {\em redex block\/} or ${\rm r}$-block of $x$.
\end{Def}

We now focus on the variable $x$ bound by the root $\rmL_x$ of the body of a redex.


\begin{Lem}\label{LemALblo} $(i)$ The ${\rm A}$-block of a certain $x \in \bft$, if it exists, is unique, just as the $\rmL$-block of $x$ and the ${\rm r}$-block of $x$.

$(ii)$ In a closed term, each var-label $x$ corresponds to exactly one ${\rm L}$-block; but even when the term is closed, not every ${\rm L}$-block has a corresponding ${\rm A}$-block.

$(iii)$ An $\rmA$-block of a certain $x \in \bft$ is a join of exactly one ${\rm r}$-block and exactly one $\rmL$-block, {\it overlapping\/} at the $\rmL_x$ binding $x$.

\end{Lem}



\subsubsection{Focused beta-reduction}

Sometimes there is a need for another form of $\beta$-reduction. One occasion is when $\beta$-reduction is invoked to model {\em definition unfolding\/}: then a defined notion occurring in $M$, say $x$, is replaced by the definiens, say $P$. Such an action generally occurs for only one instance of the definiendum $x$. So instead of replacing {\em all\/} occurrences of $x$ in $M$, one aims at {\em precisely one\/} occurrence.

When adapting $\beta$-reduction to this situation, there are several things to be considered:

\smallskip

$(i)$ The substitution $[x := P]$ should act on exactly {\em one\/} free $x$ in $M$.

$(ii)$ Hence, $x$ must occur free in $M$.

$(iii)$ The redex $(\lambda x : K \lamdot M)P$ should remain active after the intended reduction, since there may be other free $x$'s in $M$ which still need a type annotation (i.e., $K$); moreover, there must remain a possibility to substitute $P$ for one or more of these $x$'s in a later stage of the process.

%$(iv)$ If we keep the redex active, then it may stand in the way of an `embracing' redex.

\smallskip

All this is covered in the following definition of {\em focused\/} $\beta$-reduction, for which we use the symbol $\rar_f$.


\begin{Def}
Let ${\bf t}$ be a $\lambda$-tree, let $b \in {\bf t}$ be a balanced path and assume that $r \, x \in^\wedge_\vee {\bf t}$ is a complete path in ${\bf t}$ with $r \equiv p \, \rmA \, b \, {\rm L}_x \, q$. We call the balanced reduction identified by $p$ and focussed on the path $q \, x$, a {\em $(p,q)$-reduction\/}. It is defined by

\smallskip

${\bf t} ~\rar_f~ {\bf t}[r \, x := r \, {\it tree\/}(p \, \rmS)]$.
\end{Def}

The possibility of having {\em balanced\/} $\beta$-reduction is necessary to be able to deal with redexes which otherwise would be forbidden by the maintenance of the redex $(\lambda x : K \lamdot M)P$ after $\beta$-reduction, as described in requirement $(iii)$. See the following example.

\begin{Exa}
We have, in $\lambda$-calculus with normal untyped $\beta$-reduction:

$(\lambda x  \lamdot ((\lambda y  \lamdot M) Q)) P \red (\lambda x  \lamdot (M[y := Q])) P \red M[y := Q][x := P]$.

In {\em focused\/} $\beta$-reduction, this becomes:

$(\lambda x \lamdot ((\lambda y  \lamdot M) Q)) P \rar_f (\lambda x  \lamdot (\lambda y  \lamdot M[y_0 := Q])Q) P \rar_f$

\mbox{$(\lambda x  \lamdot ((\lambda y  \lamdot M[y_0 := Q])Q)[x_0 := P])P$}.

Here $y_0$ and $x_0$ are selected instances of the free $y$'s and $x$'s in $M$, respectively.

The second of the two one-step, focused, reductions could not be executed without the possibility to have a balanced $\lambda$-term $(\lambda y  \lamdot M[y_0 := Q])Q$ between the $\lambda x$ and the $P$.

\end{Exa}


\subsubsection{Erasing reduction}

After having applied balanced or focused $\beta$-reduction, one also desires a reduction that gets rid of the `remains', i.e., the $\rmA$ and the ${\rm L}_x$ in grafted trees $p \, \rmA \, b \, {\rm L}_x \, {\bf t'}$ where $x \not \in {\it FV}({\bf t'})$. Such pairs $\rmA \ldots {\rm L}_x$ are superfluous, since ${\bf L}_x$ has no $x$ that is bound to it. We call the corresponding reduction {\em erasing\/} $\beta$-reduction and use symbol $\rar_e$ for it. (This reduction is also referred to as `garbage collection' in the literature.)

\begin{Def}
Let ${\bf t}$ be a $\lambda$-tree, let $b \in {\bf t}$ be a balanced path and assume that $p \, \rmA \, b \, {\rm L}_x \, {\bf t'} \in^\wedge_\vee {\bf t}$. Assume moreover that $x \not\in {\it FV}({\bf t'})$.

Then ${\bf t} ~\rar_e~ {\bf t}[{\it tree}(p) := {\it tree}(p \, \rmA)[{\it tree} (p \, \rmA \, b) := {\it tree} (p \, \rmA \, b \, {\rm L}_x)]]$.
\end{Def}

We denote the transitive closure of a reduction $\rar_i$ by $\rrar_i$. An arbitrary sequence of reductions $\rar_i$ and $\rar_j$ is denoted $\rrar_{i,j}$.

\begin{The}
Let ${\bf t}$ and ${\bf t'}$ be $\lambda$-trees.

$(i)$ ${\bf t} \rrar_\beta {\bf t}' \Rar {\bf t} \rrar_{b,e} {\bf t}'$.

$(ii)$ ${\bf t} \rar_b {\bf t}' \Rar {\bf t} \rrar_f {\bf t}'$

$(iii)$ {\em (Postponement of $\rar_e$ after $\rar_b$)}  If ${\bf t} \rrar_{b,e} {\bf t'}$, then there is ${\bf t''}$ such that ${\bf t} \rrar_{b} {\bf t''} \rrar_e {\bf t'}$.

$(iv)$ {\em (Postponement of $\rar_e$ after $\rar_f$)} If ${\bf t} \rrar_{f,e} {\bf t'}$, then there is ${\bf t''}$ such that ${\bf t} \rrar_{f} {\bf t''} \rrar_e {\bf t'}$.
\end{The}
{\it Proof\/} $(iii)$ \citealp{Ned73}, p.\ 48, Theorem 6.19.

$(iv)$ Similarly. $\Box$

\begin{The}\label{TheConRed}
$\rar_b$, $\rar_f$ and $\rar_e$ are confluent.
\end{The}
{\it Proof\/} For $\rar_b$ and $\rar_e$, see \citealp{Ned73}, Theorems 6.38 and 6.42. For $\rar_f$, see \citealp{AccKes12}.

%\newpage

\section{Name-free lambda-terms}\label{SecNamFre}

\subsection{The binding of variables}

A recurrent nuisance in the formalization of lambda calculus is the {\it naming\/} of variables, which plays a dominant role in the establishment of binding. Let's consider some $\lambda$-term in which $\lambda x$ binds variable $x$. Then one may replace both mentioned $x$'s by a $y$, on the condition that this is done consistently through the term, and that one prevents that the variable renaming does lead to a name-clash. (This relation is called $\alpha$-reduction.) For example, the mentioned renaming is forbidden in the term $\lambda x \lamdot \lambda y \lamdot x$, for obvious reasons.

Another cause of worry is that $beta$-reduction has a spreading effect on all kinds of variables, so that it is sometimes a precarious matter to ensure that no `undesired' binding between a $\lambda$ and a variable arises.

\smallskip

To prevent these matters, N.G.\ de Bruijn invented a {\it name-free\/} version for terms in $\lambda$-calculus (\citealp{deB72}). Instead of using names such as $x$ to record bindings in terms, he employed {\it natural numbers\/}. The idea is to see the $\lambda$-term as a tree, comparable to the way we introduced $\lambda$-trees in the previous chapter. The principle is, that a variable with number $n$ is bound to the $\lambda$ which can be found by following the root path ending in this $n$, and choosing the $n$-th $\lambda$ along this path as the binder. In a $\lambda$-tree, we call such an end label $n$ a {\em numerical variable\/} or a {\em num-label\/}.

This approach is known as: 'bound variables references by depth' (\citealp{deB78a}). Another way of identifying the binder, also discussed in that paper, is to count the number of $\lambda$'s from the root to the intended one (`reference by level').

In Example~\ref{ExaLamTre}\,(2), the name-carrying $\lambda$-tree of  Example~\ref{ExaLamTre}\,(1) has been exhibited, as an example of a name-free tree.

\smallskip

This {\it name-free\/} representation of $\lambda$-calculus, straightforward as it seems, is not so simple as it appears. A pleasant feature of it is that $\alpha$-reduction is no longer required. But on the other hand, when applying $\beta$-reduction, a lot of updating is necessary. This updating is not very easy and it may require extra calculations that complicate matters.

\begin{Exa}
Consider the term $t_1 \equiv (\lambda x \lamdot \lambda y \lamdot (\lambda z \lamdot y)x)$ in untyped lambda calculus. This term $\beta$-reduces to $t_2 \equiv \lambda x \lamdot \lambda y \lamdot y$.

In name-free notation, $t_1 \equiv \lambda \,\lambda \,(\lambda \,2)\,2$. (Note that the third $\lambda$ is not on the root path of the free $x$; so, the $x$ in $t_1$ becomes not $3$, but $2$ in the name-free version.)

The name-free version of $t_2$ is $\lambda \, \lambda \, 1$: the first number $2$ in $t_1$ must be updated after the $\beta$-reduction: it returns as the number $1$ in $t_2$. (The second $2$ in $t_1$, together with the third $\lambda$, vanish by the $\beta$-reduction.)
\end{Exa}



\begin{Note}
There is another version of name-free trees for $\lambda$-terms, in which the labels are situated not at the edges -- as in our proposal -- but at the {\it nodes\/} (see \citealp{deB78a}). Moreover, the labels $\rmS$ that we use, are omitted. This works just as well, but for two details. We show this with the same Example~\ref{ExaLamTre}\,(2).

Imagine the same tree, but with all labels `raised' to the closest node. So, for example, the root vertex is now labelled ${\rm L}$ and the $\rmS$'s have vanished. %(One may erase now the edges hanging on the leafs.)

(1) When we consider the tree as not being embedded in the plane, then it is unclear in this case which is the main term and which is the subterm for a given branching. But this can be easily solved by defining trees as planar.

(2) A more serious matter is that an extra provision is necessary for determining the binder of a variable. For example, in the tree of Example~\ref{ExaLamTre}\,(2) with all labels raised, the root path of the variable numbered 4 becomes one ${\rm L}$ longer, viz.\ ${\rm L} \, {\rm L} \, {\rm L} \, {\rm L} \, {\rm P}$. Now observe that the fourth ${\rm L}$ is never meant to bind a variable on this path. So one has to neglect that fourth ${\rm L}$ when counting backwards from 4 to 0 in the process of finding its binder. (Simply changing variable number 4 into 5 is not the proper way to solve this problem; for example, this makes Lemma~\ref{LemTreBou}, (2), untrue, thus undermining the intended binding structure.)
\end{Note}

\begin{Def}
We use the symbol ${\cal T}^{\it car}$ for the set of {\it name-carrying\/}, closed $\lambda$-trees. The symbol ${\cal T}^{\it fre}$ means the set of {\it name-free\/}, closed $\lambda$-trees.
\end{Def}

The following sections discuss $\beta$-reduction in the name-free case. Since reduction itself is independent of typing, we do not distinguish between typed or untyped version of $\lambda$-calculus. So when discussing the sets ${\cal T}^{\it car}$ and ${\cal T}^{\it fre}$ we do not bother whether the terms are typed or not.


%\put(0,250){(1) name-carrying}
%\put(20,40){\circle*{4}}
%\put(20,60){\line(1,-1){20}}
%\put(30,51){${\rm S}$}


\subsection{Name-free reductions with instant updating}

In the present section we consider name-free $\beta$-reduction and some of its variants and each time we describe the updating that is required. We consider $\beta$-reduction and its variants, all with {\it instant updating\/}. That is, each one-step reduction ends in a $\lambda$-tree in which the num-labels have immediately been updated. In Section~\ref{SecDelUpd}, we try to simplify the name-free $\beta$-reduction of $\lambda$-terms, by postponing the updates ({\it delayed
updating\/}).

\begin{Note}
Below, we use the same symbols for the various name-free reductions, as in the {\em named\/} case of Section~\ref{SecVarBet} (viz., $\rar_{\beta}$, $\rar_b$ and $\rar_f$). There will be no confusion, since we use letters like ${\bf t}$ for paths in the name-carrying cases, and ${\bf u}$ in the name-free cases.
\end{Note}

\subsubsection{Beta-reduction}\label{FreBetRed}

We give a description of $\beta$-reduction with instant updating in Definition~\ref{DefBetFre}.

\begin{Not}
$(i)$ We use the notation $[x := A; y := B]$ for {\it simultaneous substitution\/}.

$(ii)$ The {\em lenght\/} $| p |$ of a path $p$ is the number of labels (including num-labels) in $p$.

$(iii)$ The {\em {\rm L}-length\/} $\lVert p \lVert$ of a path $p$ is the number of binders (i.e., ${\rm P}$ or ${\rm L}$) occurring in~$p$.
\end{Not}

\begin{Def}\label{DefBetFre}

Let ${\bf u} \in {\cal T}^{\it fre}$ and assume that $p \, \rmA \, \rmL \in^\wedge {\bf u}$.
Then the $\beta$-reduction based on the redex identified by $p$, also called {\em $p$-reduction\/}, is

\smallskip

 ${\bf u} ~\red~ {\bf u} \, [{\it tree(p)} := \{q \, n \in^\wedge_\vee {\it tree}(p \, \rmA \, \rmL) \,[{\it upd}_1 ~;~ {\it upd\/}_2] \}]$, where

%\smallskip

\[ \left\{ \begin{array}{l}
{\it upd\/}_1 \equiv \, n := n - 1~{\it if\/}~n > \lVert q \lVert + 1, \\

{\it upd\/}_2 \equiv \, n := \{r \, l \in^\wedge_\vee {\it tree}(p \, \rmS) \, [{\it upd\/}_3] \} ~{\it if\/}~ n =  \lVert q \lVert  + 1
\end{array}
\right. \]
\hspace{3cm} and ${\it upd\/}_3 \equiv \, l := l +  \lVert q \lVert ~{\it if\/}~l >  \lVert r \lVert $

\end{Def}

We now explain the contents of this definition.

\smallskip

Just as in Definition~\ref{DefBetTre}, we consider two subtrees of the original $\lambda$-tree ${\bf u}$: ${\it tree\/}(p \, \rmA \, \rmL)$ and ${\it tree\/}(p \, \rmS)$. Both trees need updating because of the performed $\beta$-reduction. In the first tree, we have two simultaneous updatings, ${\it upd}_1$ and ${\it upd}_2$. They act on {\it all\/} complete paths $q \, n$ in ${\it tree\/}(p \, \rmA \, \rmL)$, and the choice depends on the value of the leaf $n$ in the path considered. We discern two cases: $n > \lVert q \lVert  + 1$ and $n =  \lVert q \lVert  + 1$. (It is understood that in the case not mentioned, viz. $n <  \lVert q \lVert  + 1$, the {\it identical\/} update is meant, so $n := n$.) See Figure~\ref{PicBetRed} for a visual explanation.

Note that, in the case $n =  \lVert q \lVert  + 1$, the $n$ is replaced by a copy of the full ${\it tree}(p \, S)$, updated by ${\it upd_3\/}$ if $l >  \lVert r \lVert $. (Also here, the intention is that an identical update $l := l$ is applied on $r \, l \in^\wedge_\vee {\it tree}(p \, S)$,  in the missing case $l \leq  \lVert r \lVert $.)
If $n > \lVert q \rVert + 1$, one has to compensate ($n$ becoming $n-1$) for the missing $\rmL$.

In all three cases, not only the relevant $\rmA$-$\rmL$-pair, but also the grafted tree $\rmS \, {\it tree}(p \, \rmS)$ has vanished.


\begin{figure}[h]

\begin{picture}(250,230)(0,45)
\put(50,70){\line(0,1){200}}
\put(50,270){\line(-1,-1){30}}
\put(50,270){\line(1,-1){30}}
\multiput(17,237.3)(-3,-3){5}{\circle*{1}}
\multiput(83,237.3)(3,-3){5}{\circle*{1}}

\put(50,190){\line(2,-1){40}}
\put(90,170){\line(0,-1){60}}

\put(90,170){\line(-1,-3){13,3}}
\put(90,170){\line(1,-3){13,3}}
%\put(76.8,130){\line(1,0){8.2}}
%\put(103.2,130){\line(-1,0){8.2}}

\put(50,150){\line(-1,-3){20}}
\put(50,150){\line(1,-3){20}}
%\put(30,90){\line(1,0){15}}
%\put(70,90){\line(-1,0){15}}

\put(50,90){\circle*{4}}
\put(50,150){\circle*{4}}
\put(50,170){\circle*{4}}
\put(50,190){\circle*{4}}
\put(50,270){\circle*{4}}
\put(90,130){\circle*{4}}


\put(42,78){$n$}
\put(42,110){$q$}
\put(40,157){$\rmL$}
\put(40,177){$\rmA$}
\put(42,227){$p$}
\put(68,184){$\rmS$}
\put(83,140){$r$}
\put(82,118){$l$}

\put(106,107){${\it tree}(p \, \rmS)$}
\put(80,70){${\it tree}(p \, \rmA \,  \rmL)$}

\put(114,117){\vector(-2,1){10}}
\put(88,80){\vector(-2,1){10}}

\put(125,187){$\red$}


\put(250,70){\line(0,1){200}}
\put(250,270){\line(-1,-1){30}}
\put(250,270){\line(1,-1){30}}
\multiput(217,237.3)(-3,-3){5}{\circle*{1}}
\multiput(283,237.3)(3,-3){5}{\circle*{1}}

\put(250,190){\line(-1,-3){20}}
\put(250,190){\line(1,-3){20}}
%\put(230,130){\line(1,0){40}}
%\put(270,110){\line(-1,0){15}}

\put(250,130){\line(-1,-3){13.3}}
\put(250,130){\line(1,-3){13.3}}
%\put(236.8,90){\line(1,0){8.2}}
%\put(263.2,90){\line(-1,0){8.2}}

\put(250,90){\circle*{4}}
\put(250,130){\circle*{4}}
\put(250,190){\circle*{4}}
\put(250,270){\circle*{4}}

%\put(242,117){$n$}
\put(242,150){$q$}
\put(242,227){$p$}
\put(243,100){$r$}
\put(242,78){$l$}

\put(190,190){\line(-1,-3){20}}
\put(190,190){\line(1,-3){20}}
%\put(170,130){\line(1,0){15}}
%\put(210,130){\line(-1,0){15}}

\put(310,190){\line(-1,-3){20}}
\put(310,190){\line(1,-3){20}}
%\put(290,130){\line(1,0){15}}
%\put(330,130){\line(-1,0){15}}

\put(190,110){\line(0,1){80}}
\put(310,110){\line(0,1){80}}

\put(182,118){$n$}
\put(182,150){$q$}

\put(290,118){$n \! - \! 1$}
%\put(333,138){\vector(-4,-1){16}}
\put(302,150){$q$}

\put(190,130){\circle*{4}}
\put(190,190){\circle*{4}}
\put(310,130){\circle*{4}}
\put(310,190){\circle*{4}}

\multiput(195,190)(8,0){7}{\circle*{1}}
\multiput(305,190)(-8,0){7}{\circle*{1}}

\put(266,70){${\it tree}(p \, \rmS)$}
\put(274,80){\vector(-2,1){10}}

\put(162,93){${\rm if}~n <  \lVert q \lVert  + 1$}
\put(282,93){${\rm if}~n >  \lVert q \lVert  + 1$}
\put(222,53){${\rm if}~n =  \lVert q \lVert  + 1$}

\end{picture}

\caption{A picture of name-free $\beta$-reduction with updating}
\label{PicBetRed}
\end{figure}



\subsubsection{Balanced beta-reduction}\label{FreBalRed}

For balanced $\beta$-reduction, the updates are somewhat different, due to the non-vanishing of the ${\rm A}$-${\rm L}$-couple involved, and the remaining of the original `argument' ${\it tree}(p \, S)$.


\begin{Def}\label{DefBetFreBal}

Let ${\bf u}  \in {\cal T}^{\it fre}$, let $b \in {\bf u}$ be a balanced path and assume that $p \, \rmA \, b \, \rmL \in^\wedge {\bf u}$.
Then the balanced reduction based on the  redex identified by $p$ (again called $p$-reduction) is

\smallskip

${\bf u} \rar_b {\bf u} \, [{\it tree}(p \, \rmA \, b \, \rmL) := \{q \, n \in^\wedge_\vee {\it tree}(p \, \rmA \, b \, \rmL) \,[{\it upd}_1]\}]$,
where

\smallskip

%\[ \left\{ \begin{array}{l}

\hspace{1.5cm} ${\it upd}_1 \equiv n := \{r \, l \in^\wedge_\vee {\it tree}(p \, \rmS)[{\it upd_2}] \} ~{\it if}~ n =  \lVert q \lVert  + 1$

\smallskip

\hspace{3cm} and ${\it upd}_2 \equiv \, l := l +  \lVert q \lVert  + 1 +  \lVert b \lVert  ~{\it if\/}~l >  \lVert r \lVert $
%\end{array}
%\right. \]
\end{Def}

As in Section~\ref{FreBetRed}, an identical substitution (i.e., nothing changes) applies in the missing cases for $n$ and $l$.

\subsubsection{Focused beta-reduction}\label{FreBalFoc}

In the case of focused $\beta$-reduction, a simple adaptation of Section~\ref{FreBalRed} is required. This leads to the following.

\begin{Def}\label{DefBetFreFoc}

Let ${\bf u} \in {\cal T}^{\it fre}$, let $b \in {\bf u}$ be a balanced path, assume that $p \, \rmA \, b \, \rmL \in^\wedge {\bf u}$ and that $q \, n$ is a fixed, complete path in ${\it tree}(p \, \rmA \, b \, \rmL)$.

Consider a $\beta$-reduction identified by $p$. When focusing on $q$ as the {\em focus path\/}, we call this a {\em $(p,q)$-reduction\/} defined by

\smallskip

${\bf u} \rar_{f} {\bf u} \, [{\it tree}(p \, \rmA \, b \, \rmL) := {\it tree}(p \, \rmA \, b \, \rmL)[{\it upd_1}]]$,
where

\smallskip

%\[ \left\{ \begin{array}{l}

\hspace{1.5cm} ${\it upd_1} \equiv q \, n := q \, \{r \, l \in^\wedge_\vee {\it tree}(p \, \rmS)[{\it upd_2}] \} ~{\it if}~ n =  \lVert q \lVert  + 1$

\smallskip

\hspace{3cm} and ${\it upd_2} \equiv \, l := l +  \lVert q \lVert  + 1 +  \lVert b \lVert  ~{\it if\/}~l >  \lVert r \lVert $
%\end{array}
%\right. \]
\end{Def}

\subsubsection{Erasing reduction}\label{FreBalEra}

For erasing reduction, the following definition applies.

\begin{Def}
Let ${\bf u} \in {\cal T}^{\it fre}$, let $b \in {\bf u}$ be a balanced path, assume that $p \, \rmA \, b \, \rmL \in^\wedge {\bf u}$ and that no numerical variable in ${\it tree}(p \, \rmA \, b \, \rmL)$ is bound by the mentioned $\rmL$. (Otherwise said: for no $q \, n \in^\wedge_\vee {\it tree}(p \, \rmA \, b \, \rmL)$ we have that $n =  \lVert q \lVert  + 1$.)

\smallskip

Then ${\bf u} \rar_e {\bf u} [{\it tree}(p) := {\it tree}(p \, \rmA)[{\it tree}(p \, \rmA \, b) := \{q \, n \in^\wedge_\vee {\it tree}(p \, \rmA \, b \, \rmL)\}[{\it upd}]]$, where

\smallskip

\hspace{1.5cm} ${\it upd} \equiv n := n - 1 ~{\it if\/}~n >  \lVert q \lVert $
\end{Def}


\section{Delayed updating}\label{SecDelUpd}

Another possibility is to make reduction easy, by not considering the updates of the numerical variables, until required. In this case we {\it delay all updates\/}. In the text below, we restrict this case to the {\it focused\/} balanced $\beta$-reduction described above (Section~\ref{FreBalFoc}), but it can easily be extended to the more general balanced version of $\beta$-reduction discussed in Section~\ref{FreBalRed}.

There have been many proposals for describing the necessary updating in a formal way. The fist one has been described in \citealp{deB78a}: an update function $\theta$ is added to a term constructor $\varphi$ in order to update the num-labels. See also, e.g., \citealp{Ven}.

\subsection{Focused beta-reduction with delayed updating}\label{SubSecDel}

We use the symbol `$\rar_{\it df}$' for the delayed, focused $\beta$-reduction.

\begin{Def}\label{DefBetDel}
Let ${\bf u} \in {\cal T}^{\it fre}$, let $b \in {\bf u}$ be a balanced path, assume that $p \, \rmA \, b \, \rmL \in^\wedge {\bf u}$ and that $q \, n$ is a fixed, complete path in ${\it tree}(p \, \rmA \, b \, \rmL)$, where $n$ is bound by $\rmL$.

\smallskip

Then ${\bf u} \rar_{\it df} {\bf u} \, [{\it tree}(p \, \rmA \, b \, \rmL) := {\it tree}(p \, \rmA \, b \, \rmL)[q \, n := q \, n \, {\it tree}(p \, \rmS)]]$.
\end{Def}

Note that the edge labelled $n$ stays where it is, and ${\it tree}(p \, \rmS) $ is simply attached to it in $\rar_{\it df}$-reduction, whereas this edge is {\it replaced\/} by an updated ${\it tree}(p \, \rmS)$ in the original focused reduction. The remaining presence of the label $n$ is to enable updating at a later stage.

For a pictorial representation, see Figure~\ref{PicDelRed}. Note: if ${\it tree}(p \, \rmS) $ consists of a single edge only, labelled with a num-variable, then this edge is just attached to the open end of the edge labelled $n$, {see Section~\ref{ExaBetDel} for an example.

\smallskip

The above definition implies that we have to revise our definition of $\lambda$-trees, since {\it var-cells now have open ends\/}: they may have connections to other cells at their bottom end. So num-variables no longer need to be end-labels in a path.

\begin{Def}\label{DefInnOut}
$(i)$ Num-variables not being end-labels, we call {\it inner num-labels\/}.

$(ii)$ To distinguish them from the (still present) num-labels that {\it are\/} end-labels, we call the latter {\it outer\/} num-labels (or leafs).

$(iii)$ A $\lambda$-tree in which inner variables are allowed, we call an {\em extended\/} $\lambda$-tree.
\end{Def}

Consequently, other definitions should be extended, as well. For example, the definition of a {\it balanced path\/} (Definition~\ref{DefBalPat}) must be adapted such that it allows inner num-labels {\em inside\/} the string of ${\rm A}$'s and $\rmL$'s.

\smallskip

{\it In the remainder of this chapter, we assume that these definition revisions have been made. Moreover, when speaking about $\lambda$-trees without further restrictions, we mean {\em extended} ones.}

\smallskip

\begin{Def}\label{DefTreFre}
The symbol ${\cal T}^{\it fre}$ concerns the set of name-free, {\em closed\/} trees {\it without\/} inner variables. The set of these trees where also inner variables are permitted, we denote by ${\cal T}^{\it fre+}$.

Hence, we must read $\bfu \in \Tfrep$ for $\bfu \in \Tfre$, in Definition~\ref{DefBetDel}.
\end{Def}

We define what {\em inclusion\/} of (extended) $\lambda$-trees means.

\begin{Def}\label{DefIncTre}
Let ${\bf u}$ and ${\bf u}'$ be $\lambda$-trees. Then ${\bf u} \subseteq {\bf u'}$ iff $p \inwv {\bf u} \Rar p \inw {\bf u}'$.
\end{Def}






\begin{figure}[h]

\begin{picture}(250,270)(0,10)

\put(50,70){\line(0,1){200}}
\put(50,270){\line(-1,-1){30}}
\put(50,270){\line(1,-1){30}}
\multiput(17,237.3)(-3,-3){5}{\circle*{1}}
\multiput(83,237.3)(3,-3){5}{\circle*{1}}

\put(50,220){\line(2,-1){40}}
\put(90,200){\line(0,-1){60}}

\put(90,200){\line(-1,-3){13.3}}
\put(90,200){\line(1,-3){13.3}}
%\put(76.8,160){\line(1,0){8.2}}
%\put(103.2,160){\line(-1,0){8.2}}

\put(50,150){\line(-1,-3){20}}
\put(50,150){\line(1,-3){20}}
%\put(30,90){\line(1,0){15}}
%\put(70,90){\line(-1,0){15}}

\put(50,90){\circle*{4}}
\put(50,150){\circle*{4}}
\put(50,170){\circle*{4}}
\put(50,200){\circle*{4}}
\put(50,220){\circle*{4}}
\put(50,270){\circle*{4}}
\put(90,200){\circle*{4}}
\put(90,160){\circle*{4}}

\put(42,77){$n$}
\put(42,110){$q$}
\put(40,157){$\rmL$}
\put(42,182){$b$}
\put(40,207){$\rmA$}
\put(42,242){$p$}
\put(68,214){$\rmS$}
\put(83,170){$r$}
\put(82,147){$l$}

%\put(72,130){${\it tree}(p \, \rmS)$}
%\put(24,60){${\it tree}(p \, \rmA \, b \, \rmL)$}


\put(144,187){$\rar_{\it df}$}


%\put(250,95){\line(0,1){50}}
%\put(250,95){\line(0,1){175}}
\put(250,270){\line(-1,-1){30}}
\put(250,270){\line(1,-1){30}}
\multiput(217,237.3)(-3,-3){5}{\circle*{1}}
\multiput(283,237.3)(3,-3){5}{\circle*{1}}

\put(250,220){\line(2,-1){40}}
\put(290,200){\line(0,-1){60}}

\put(290,200){\line(-1,-3){13.3}}
\put(290,200){\line(1,-3){13.3}}
%\put(276.8,160){\line(1,0){8.2}}
%\put(303.2,160){\line(-1,0){8.2}}

\put(250,150){\line(-1,-3){20}}
\put(250,150){\line(1,-3){20}}
%\put(230,90){\line(1,0){15}}
%\put(270,90){\line(-1,0){15}}

\put(250,90){\circle*{4}}
\put(250,150){\circle*{4}}
\put(250,170){\circle*{4}}
\put(250,200){\circle*{4}}
\put(250,220){\circle*{4}}
\put(250,270){\circle*{4}}
\put(290,200){\circle*{4}}
\put(290,160){\circle*{4}}

\put(242,77){$n$}
\put(242,110){$q$}
\put(240,157){$\rmL$}
\put(242,182){$b$}
\put(240,207){$\rmA$}
\put(242,242){$p$}
\put(268,214){$\rmS$}
\put(283,170){$r$}
\put(282,147){$l$}

%\put(250,70){\line(-1,-3){20}}
%\put(250,70){\line(1,-3){20}}
\put(250,10){\line(0,1){260}}
%\put(250,70){\line(0,-1){60}}
%\put(250,70){\circle*{4}}
\put(250,70){\circle*{4}}
\put(250,30){\circle*{4}}
\put(243,40){$r$}
\put(242,17){$l$}

\put(250,70){\line(-1,-3){11.7}}
\put(250,70){\line(1,-3){11.7}}
%\put(236.8,50){\line(1,0){8.2}}
%\put(263.2,50){\line(-1,0){8.2}}

\end{picture}

\caption{A picture of name-free $\beta$-reduction with delayed updating}
\label{PicDelRed}
\end{figure}

In Definition~\ref{DefBetDel}, right hand side, the copy of ${\it tree}(p \, \rmS)$ attached to the edge labelled $n$, is not updated. Moreover, the complete tree ${\bf u}$ (including the edge labelled $n$) remains intact as integral part of ${\bf u'}$. Hence, we have the following theorem.


\begin{The}\label{TheImpInc}
Let ${\bf u}, {\bf u'} \in \Tfrep$. Then ${\bf u} \rar_{\it df} {\bf u'}$ implies ${\bf u} \subset {\bf u'}$.
\end{The}


The latter fact is clearly an advantage. But it comes with a price: when we wish to establish the relation between a leaf-variable of the copied ${\it tree}(p \, \rmS)$ and its binder, we have to do more work. We discuss this in the following subsection.



\subsection{Tracing the binder in reduction with delayed updating}

Let ${\bf u}_0 \in {\cal T}^{\it fre}$ and ${\bf u}_0 \rrar_{\it df} {\bf u}$, so ${\bf u} \in \Tfrep$ is the result of a series of delayed, focused reductions. These reductions may introduce inner variables, so it is not immediately clear what the binders are for (inner or outer) variables. In this section we investigate how one can determine the binder of a variable in ${\bf u}$.

\smallskip

Let $p \, n \in^\wedge {\bf u}$, so $p \, n $ is a root path. Here $n$ can be an inner or an outer num-label. We describe a pushdown automaton ${\cal P}_{fre}$ to find:

\smallskip

$(i)$ the {\bf ${\rm L}$-binder} of $n$, i.e., the label ${\rm L} \in p$ that binds $n$ (this label always exists, since ${\cal T}^{\it fre}$ only contains closed terms),

$(ii)$ or the {\bf ${\rm A}$-binder} of $n$, i.e., the label ${\rm A} \in p$ that {\em matches\/} the ${\rm L}$-binder of $n$, {\em if such an ${\rm A}$ exists\/} (this needs not to be the case).

\smallskip

%In case $(i)$, the initial pair is $(n,0)$; in case $(ii)$ it is $(n,1)$.

%\begin{Note}

%In the previous section, Definitions~\ref{DefLBlo} and \ref{DefABlo}, we have defined ${\rm L}$-blocks and ${\rm A}$-blocks of {\it named\/} variables in $\Tcar$. But we have not yet done this for (intern or extern) num-labels in $\Tfrep$. In the description above, we use the notions ${\rm L}$-block and ${\rm A}$-block only informally, to facilitate the explanation of the algorithm $\cal P$. A formal definition will be given in Definition~\ref{DefALBlo}.

%\end{Note}

\medskip

The {\bf use of the algorithm} is as follows. Let ${\bf u} \in \Tfrep$ and $p \, n \inw {\bf u}$. Assume that we desire to apply algorithm ${\cal P}_{fre}$ to find the ${\rm L}$-binder or the ${\rm A}$-binder of $n$.

In algorithm ${\cal P}_{fre}$, we employ {\it states\/} that are pairs of natural numbers: $(k, l)$. We start with the insertion of a pair $(n,i)$  in the tail of the string $p \, n$, {\it between\/} $p$ and $n$. The automaton moves the pair {\it to the left\/} through $p$, one step at a time, successively passing the labels in $p$ and meanwhile adapting the numbers in the pair. The automaton stops when the desired label (either the binding ${\rm L}$ or the ${\rm A}$ matching that ${\rmL}$) has been found.





The automaton has an outside {\it stack\/} that will contain certain states that are {\it pushed\/} on the top of the stack; a state on top of the stack can also be {\it popped back\/}, i.e., inserted into the path $p$, again.

The {\it transitions\/} are described in Definition~\ref{DefAlgBin}. A possible one-step transition is denoted by the symbol $\rar$ (the reflexive, transitive closure of this relation is denoted $\rrar$). The procedure may be complicated by several recursive calls (see Example~\ref{ExaProPat}).

The {\it action\/} of ${\cal P}_{fre}$ is described in Note~\ref{ExpActPro}.


%We now discuss the transition rules for determining the label ${\rm L}$ binding $n$ in a root path $p \, n$ of $\lambda$-tree ${\bf u}$.

\smallskip

The formal description of ${\cal P}_{fre}$ is the following.

\smallskip

{\bf Preparation:} Transform $p \, n$ into $p \, (n,i) \, n$, where $i$ is $0$ to find the ${\rm L}$-binder, and $1$ for the ${\rm A}$-binder.

Now {\bf start} ${\cal P}_{fre}$ employing the transition rules of Definition~\ref{DefAlgBin}.



%\smallskip




\begin{Def}\label{DefAlgBin}
The algorithm ${\cal P}_{fre}$ is specified by the following rules:

$(1)$ ~ {\it first step:} ${\it stack} = \emptyset$

%$ \left\{ \begin{array}{l}
%{\rm in~search~of~\mbox{L-block}:} ~p \, m \rar p \, (m,0) \, m \\
%{\rm in~search~of~\mbox{A-block}:} ~p \, m \rar p \, (m,1) \, m
%\end{array} \right. $

$(2)$ ~ $p ~ {\rm L} ~ (m,k) ~ q ~ \rar ~ p ~ (m \makebox{--} 1,k) ~ {\rm L} ~ q$, if $m > 0$

$(3)$ ~ $p ~ {\rm A} ~ (m,k) ~ q ~ \rar ~ p ~ (m,k) ~ {\rm A} ~ q$, if $m > 0$

$(4)$ ~ $p ~ {\rm S} ~ (m,k) ~ q ~ \rar ~ p ~ (m,k) ~ {\rm S} ~ q$, if $m > 0$

$(5)$ ~ $p ~ j ~ (m,k) ~ q ~ \rar ~ p ~ (j,1) ~ j ~ q$, if $m > 0$; {\it push} $(m,k)$

$(6)$ ~ $p ~ {\rm L} ~ (0,l) ~ q ~ \rar ~ p ~ (0,l \makebox{+} 1) ~ {\rm L} ~ q$, if $l > 0$

$(7)$ ~ $p ~ {\rm A} ~ (0,l) ~ q ~ \rar ~ p ~ (0,l \makebox{--} 1) ~ {\rm A} ~ q$, if $l > 0$

$(8)$ ~ $p ~ j ~ (0,l) ~ q ~ \rar ~ p ~ (0,l) ~ j ~ q$, if $l > 0$

$(9a)$ \, $p ~ (0,0)  ~ q ~ \rar ~ p ~ {\it pop}  ~ q$, if  ${\it stack} \not = \emptyset$

$(9b)$ \, $p ~ (0,0) ~ q ~ \rar {\it stop}$, if ${\it stack} = \emptyset$.

\end{Def}

%The result of the algorithm is decided by the beginning, as we see in Definition~\ref{DefALBlo}.



\begin{Def}\label{DefALBlo}
Let ${\bf u} \in \Tfrep$ and $p \, n \inw {\bf u}$.

$(i)$ If ${\cal P}_{fre}$ is applied to $p \, (n,0) \, n$ and stops in $p' \, (0,0) \, q \, n$, then $q \, n$ is called the ${\rm L}$-{\it block\/} of $n$.

$(ii)$ If ${\cal P}_{fre}$ is applied to $p \, (n,1) \, n$ and stops in $p' \, (0,0) \, q \, n$, then $q \, n$ is called the ${\rm A}$-{\it block\/} of $n$.
\end{Def}

\begin{Lem}\label{LemLblAbl}
$(i)$: If $q \, n$ is the {\rm L}-block of $n$, then $q \, n \equiv {\rm L} \, q' \, n$ and the mentioned {\rm L} is the ${\rm L}$-binder of $n$.

$(ii)$: If $q \, n$ is the {\rm A}-block of $n$, then $q \, n \equiv {\rm A} \, q' \, n$ and the mentioned {\rm A} is the ${\rm A}$-binder of $n$.
\end{Lem}

After the {\em preparation\/} and {\em step (1)}, the procedure ${\cal P}_{fre}$ has two possibly recursive {\it rounds\/}, {\bf I} and {\bf II}, with different actions. We explain this below.

\begin{Note}\label{ExpActPro}

$(i)$: Round~{\bf I} (steps (2) to (4)), gives a {\it count-down\/} of the ${\rm L}$-labels in the path leftwards from an (inner or outer) num-label $n$; in this round,
${\rm A}$-labels and ${\rm S}$-labels have no effect. When the count-down results in $m = 0$, then we just passed the ${\rm L}$ binding the original $n$. So we found the ${\rm L}$-block of $n$. If the stack is empty, so ${\cal P}_{fre}$ is not in a recursion, then the ${\rm L}$-block of the original variable has been found (step~(9b)).

$(ii)$: When an inner variable $j$ is met in Round~{\bf I} (step~(5)), then ${\cal P}_{fre}$ starts a recursive round, in which the ${\rm A}$-block of that $j$ is determined. The count-down of Round~{\bf I} is temporarily stopped and resumes immediately left of that ${\rm A}$-block (step~(9a)).\marginpar{v5. \ref{SecDelUpd}--1}

$(iii)$ The ${\rm A}$-block of an inner variable $j$ is found by executing Round~{\bf I}, giving the ${\rm L}$-block of $j$, immediately followed by Round~{\bf II} (steps (6) to (8)) determining the ${\rm r}$-block connected with $j$. The latter is done by counting upwards for ${\rm L}$'s and downwards for ${\rm A}$'s. The ${\rm L}$-block and the ${\rm r}$-block combine in the desired ${\rm A}$-block of $j$.

%$(iv)$ When searching an ${\rm A}$-block of a certain variable, ${\cal P}_{fre}$ may encounter an inner variable and consequently, the recursion goes deeper (step~(5)).

$(iv)$ Inner variables met in the search of an ${\rm r}$-block, are skipped (step~(8)), because inner variables do not affect this Round~{\bf II}-search.\marginpar{v5. \ref{SecDelUpd}--2} So no recursion is started inside an ${\rm r}$-block.

$(v)$ When determining an ${\rm r}$-block, no label $S$ may be encountered, as can be seen in steps~(6) to (8), where $S$ is missing. (See also Definition~\ref{DefBalPat}.) This is due to the fact that for each ${\rm A}$, a possibly matching ${\rm L}$ is positioned on the same `branch', and not on a `sub-branch'.

\end{Note}



\subsection{An example of the action of algorithm ${{\cal P}_{fre}}$}


We give an example of the execution of algorithm ${{\cal P}_{fre}}$.

%\newpage


\begin{Exa}\label{ExaProPat}
Firstly, we look for the ${\rm L}$-binder of the final num-label -- i.e., $3$ -- in the path ${\rm A \, A \, L \, L \, A \, A \, L \, L \, L \, A \, A \, L \, A \, L \, L \, L \, A \, 2 \, S \, L \, 4 \, L \, L \, S \, 3} $.

So we take $i = 0$,  and start ${\cal P}_{fre}$:

\smallskip

 ${\rm A \, A \, L \, L \, A \, A \, L \, L \, L \, A \, A \, L \, A \, L \, L \, L \, A \, 2 \, S \, L \, 4 \, L \, L \, S \, {\bf (3,0)} \, 3}$~
(${\it stack} = \emptyset$) $\rar$

 ${\rm A \, A \, L \, L \, A \, A \, L \, L \, L \, A \, A \, L \, A \, L \, L \, L \, A \, 2 \, S \, L \, 4 \, L \, L \, {\bf (3,0)} \, S \, 3}$ $\rar$

 ${\rm A \, A \, L \, L \, A \, A \, L \, L \, L \, A \, A \, L \, A \, L \, L \, L \, A \, 2 \, S \, L \, 4 \, L \, {\bf (2,0)} \, L \, S \, 3}$ $\rar$

 ${\rm A \, A \, L \, L \, A \, A \, L \, L \, L \, A \, A \, L \, A \, L \, L \,  L \, A \,  2 \, S \, L \, 4 \, {\bf (1,0)} \, L \, L \, S \, 3}$ ~(${\it push~} {\bf (1,0)}$) $\rar$

 ${\rm A \, A \, L \, L \, A \, A \, L \, L \, L \, A \, A \, L \, A \, L \, L \,  L \, A \,  2 \, S \, L \, {\bf (4,1)} \, 4 \, L \, L \, S \, 3}$ $\rar$

 ${\rm A \, A \, L \, L \, A \, A \, L \, L \, L \, A \, A \, L \, A \, L \, L \,  L \, A \,  2 \, S \, {\bf (3,1)} \, L \, 4 \, L \, L \, S \, 3}$ $\rar$

 ${\rm A \, A \, L \, L \, A \, A \, L \, L \, L \, A \, A \, L \, A \, L \, L \,  L \, A \,  2 \, {\bf (3,1)} \, S \, L \, 4 \, L \, L \, S \, 3}$ ~(${\it push~} {\bf (3,1)}$) $\rar$

 ${\rm A \, A \, L \, L \, A \, A \, L \, L \, L \, A \, A \, L \, A \, L \, L \,  L \, A \,  {\bf (2,1)} \, 2 \, S \, L \, 4 \, L \, L \, S \, 3}$ $\rar$

 ${\rm A \, A \, L \, L \, A \, A \, L \, L \, L \, A \, A \, L \, A \, L \, L \,  L \, {\bf (2,1)} \, A \, 2 \, S \, L \, 4 \, L \, L \, S \, 3}$ $\rar$

 ${\rm A \, A \, L \, L \, A \, A \, L \, L \, L \, A \, A \, L \, A \, L \, L \,  {\bf (1,1)} \, L \, A \, 2 \, S \, L \, 4 \, L \, L \, S \, 3}$ $\rar$

 ${\rm A \, A \, L \, L \, A \, A \, L \, L \, L \, A \, A \, L \, A \, L \, {\bf (0,1)} \, \underline{\rule[-0.1em]{0em}{1em} L \,  L \, A \, 2} \, S \, L \, 4 \, L \, L \, S \, 3}$ $\rar$

 \hspace{4.6cm} {\rm L}-block of $2$

 ${\rm A \, A \, L \, L \, A \, A \, L \, L \, L \, A \, A \, L \, A \, {\bf (0,2)} \, L \, L \,  L \, A \, 2 \, S \, L \, 4 \, L \, L \, S \, 3}$ $\rar$

 ${\rm A \, A \, L \, L \, A \, A \, L \, L \, L \, A \, A \, L \, {\bf (0,1)} \, A \, L \, L \,  L \, A \, 2 \, S \, L \, 4 \, L \, L \, S \, 3}$ $\rar$

 ${\rm A \, A \, L \, L \, A \, A \, L \, L \, L \, A \, A \, {\bf (0,2)} \, L \, A \, L \, L \,  L \, A \, 2 \, S \, L \, 4 \, L \, L \, S \, 3}$ $\rar$

 ${\rm A \, A \, L \, L \, A \, A \, L \, L \, L \, A \, {\bf (0,1)} \, A \, L \, A \, L \, L \,  L \, A \, 2 \, S \, L \, 4 \, L \, L \, S \, 3}$ $\rar$

 ${\rm A \, A \, L \, L \, A \, A \, L \, L \, L \, {\bf (0,0)} \, \underline{\rule[-0.1em]{0em}{1em} A \, A \, L \, A \, L \, L \,  L \, A \, 2} \, S \, L \, 4 \, L \, L \, S \, 3}$ ~(${\it pop~} {\bf (3,1)})$ $\rar$

 \hspace{3.7cm} {\rm A}-block of $2$

 ${\rm A \, A \, L \, L \, A \, A \, L \, L \, L \, {\bf (3,1)} \, A \, A \, L \, A \, L \, L \,  L \, A \, 2 \, S \, L \, 4 \, L \, L \, S \, 3}$ $\rar$

 ${\rm A \, A \, L \, L \, A \, A \, L \, L \, {\bf (2,1)} \, L \, A \, A \, L \, A \, L \, L \,  L \, A \, 2 \, S \, L \, 4 \, L \, L \, S \, 3}$ $\rar$

 ${\rm A \, A \, L \, L \, A \, A \, L \, {\bf (1,1)} \, L \, L \, A \, A \, L \, A \, L \, L \,  L \, A \, 2 \, S \, L \, 4 \, L \, L \, S \, 3}$ $\rar$

%\newpage

 ${\rm A \, A \, L \, L \, A \, A \, {\bf (0,1)} \, \underline{\rule[-0.1em]{0em}{1em} L \, L \, L \, A \, A \, L \, A \, L \, L \,  L \, A \, 2 \, S \, L \, 4} \, L \, L \, S \, 3}$ $\rar$

 \hspace{3.7cm} {\rm L}-block of $4$

 ${\rm A \, A \, L \, L \, A \, {\bf (0,0)} \, \underline{\rule[-0.1em]{0em}{1em} A \, L \, L \, L \, A \, A \, L \, A \, L \, L \,  L \, A \, 2 \, S \, L \, 4} \, L \, L \, S \, 3}$ ~(${\it pop~} {\bf (1,0)})$ $\rar$

 \hspace{3.6cm} {\rm A}-block of $4$

 ${\rm A \, A \, L \, L \, A \, {\bf (1,0)} \, A \, L \, L \, L \, A \, A \, L \, A \, L \, L \,  L \, A \, 2 \, S \, L \, 4 \, L \, L \, S \, 3}$ $\rar$

 ${\rm A \, A \, L \, L \, {\bf (1,0)} \, A \, A \, L \, L \, L \, A \, A \, L \, A \, L \, L \,  L \, A \, 2 \, S \, L \, 4 \, L \, L \, S \, 3}$ $\rar$

 ${\rm A \, A \, L \, {\bf (0,0)} \, \underline{\rule[-0.1em]{0em}{1em} L \, A \, A \, L \, L \, L \, A \, A \, L \, A \, L \, L \,  L \, A \, 2 \, S \, L \, 4 \, L \, L \, S \, 3}}$ ~(${\it stack\/} = \emptyset$, hence {\it stop})

 \hspace{3.4cm} {\rm L}-block of $3$
\end{Exa}


 \begin{Exa}

$(i)$  In the case of Example~\ref{ExaProPat}, we can derive the ${\rm A}$-block of the final num-label~$3$ by starting with $i = 1$. The derivation then needs three more steps.

$(ii)$  Notice that several ${\rm L}$-blocks and ${\rm A}$-blocks arise during the execution of the procedure ${{\cal P}_{fre}}$, and that these can, on their own, be derived by means of a sub-calculation of the one above.
 \end{Exa}

%Example~\ref{ExaProPat} shows that during the execution of the algorithm ${{\cal P}_{fre}}$, several $\rmL$-blocks and $\rmA$-blocks are `generated'. Moreover,
The procedure stops when the application of ${{\cal P}_{fre}}$ to a path $p \, n \inw {\bf u}$ results in the conclusion that a certain sub-path $\rmL \, q \, n$ (ending in the mentioned $n$) is an $\rmL$-block. Then we have found that the displayed $\rmL$ binds the final $n$.

\begin{Lem}
Let ${\bf u} \in \Tfrep$ and assume that ${{\cal P}_{fre}}$ has been applied to $p \, n \inw {\bf u}$ with conclusion that $\rmL \, q \, n$ is an $\rmL$-block.

Then each inner variable of $q$ is the end-variable of exactly one generated $\rmA$-block and exactly one -- corresponding -- generated $\rmL$-block.
\end{Lem}

%The generated $\rmA$-blocks occur inside the path $q$ in a {\em balanced\/} pattern:

%\begin{Lem}
%Let $p \, \rmL \, q \, n \inw {\bf u} \in \Tfrep$, $\rmL \, q \, n$ being an $\rmL$-block, and assume that $a_1$ and $a_2$ are different $\rmA$-blocks inside $q$ generated by ${{\cal P}_{fre}}$.

%$(i)$ Then either $a_1 \subset a_2$, or $a_2 \subset a_1$, or $a_1$ and $a_2$ do not overlap in $q$.

%$(ii)$ If $a_2 \subset a_1$, then also $a_2 \subset l_1$, where $l_1$ is the $\rmL$-block corresponding to~$a_1$.
%\end{Lem}


\subsection{An example of beta-reduction with delayed updating}\label{ExaBetDel}

We consider the following untyped $\lambda$-term (in the usual notation) and three of its delayed, focused $\beta$-reducts. The underlining is meant to mark the redex parts, the dot above the variable (e.g., $\dot y$) marks the focus of the reduction. Transitions $(i) \rar_{f} (ii)$
and $(ii) \rar_f(iii)$ are one step reductions, $(iii) \rrar_f (iv)$ is two-step.

\smallskip

$(i)$ $((\lambda x \lamdot (\underline{\lambda y} \lamdot \lambda z \lamdot \dot y \, z)\underline{\lambda u \lamdot x})x)\lambda v \lamdot v ~ \rar_f$

\smallskip

$(ii)$ $((\lambda x \lamdot (\lambda y \lamdot \underline{\lambda z} \lamdot (\lambda u \lamdot x)\dot z)\lambda u \lamdot x)\underline{x})\lambda v \lamdot v ~ \rar_f $

\smallskip

$(iii)$ $((\underline{\lambda x} \lamdot (\lambda y \lamdot \lambda z \lamdot (\lambda u \lamdot x) \dot x)\lambda u \lamdot \dot x)x)\underline{\lambda v \lamdot v} ~ \rrar_f $

\smallskip

$(iv)$ $((\lambda x \lamdot (\lambda y \lamdot \lambda z \lamdot (\lambda u \lamdot x) \lambda v \lamdot v) \lambda u \lamdot \lambda v \lamdot v) x) \lambda v \lamdot v$.


\medskip

In Figure~\ref{FigDelUpdExa} we represent these four $\lambda$-terms as trees. We mark the $\rmA$-$\rmL$-couples and the arguments in boxes. In the picture, we use a small triangle to mark the focus(ses).

\begin{figure}[h]\label{FigDelUpdExa}

\begin{picture}(300,310)(-40,20)
\put(10,300){$(i)$}
\put(45,300){$\rar_f$}
\put(80,300){$(ii)$}
\put(115,300){$\rar_f$}
\put(150,300){$(iii)$}
\put(185,300){$\rrar_f$}
\put(220,300){$(iv)$}

%XXXXXXXXXXXXXXXXXXXXXXXXXXXXXXX

%\put(10,40){\circle{10}}
\put(10,100){\circle*{4}}
\put(10,120){\circle*{4}}
\put(10,140){\circle*{4}}
\put(10,160){\circle*{4}}
\put(10,180){\circle*{4}}
\put(10,140){\circle*{4}}
\put(10,220){\circle*{4}}
\put(10,260){\circle*{4}}
\put(10,280){\circle*{4}}
\put(30,100){\circle*{4}}
\put(30,140){\circle*{4}}
\put(30,160){\circle*{4}}
\put(30,200){\circle*{4}}
\put(30,240){\circle*{4}}
\put(30,260){\circle*{4}}


\put(10,120){\line(1,-1){20}}
\put(10,180){\line(1,-1){20}}
\put(10,220){\line(1,-1){20}}
\put(10,280){\line(1,-1){20}}
%\put(20,240){\line(1,-1){20}}
%\put(20,60){\line(1,-1){20}}

\put(10,80){\line(0,1){200}}
\put(30,80){\line(0,1){20}}
\put(30,120){\line(0,1){40}}
\put(30,180){\line(0,1){20}}
\put(30,220){\line(0,1){40}}
%\put(40,160){\line(0,1){20}}
%\put(40,200){\line(0,1){20}}
%\put(60,60){\line(0,1){20}}

\put(22,111){${\rm S}$}
\put(22,171){${\rm S}$}
\put(22,211){${\rm S}$}
\put(22,271){${\rm S}$}


\put(0,106){${\rm A}$}
\put(0,126){${\rm L}$}

%\put(-5,146){${\fbox{\rmL}}$}
%\put(-6,166){${\fbox{\rmA}}$}
\put(2,149){\circle{12}}
\put(-1,146.2){$\rm L$}
\put(2,169){\circle{12}}
\put(-1,166.2){$\rm A$}

\put(0,196){${\rm A}$}
\put(0,236){${\rm L}$}
\put(0,266){${\rm A}$}
\put(34,146){${\rm L}$}
\put(34,246){${\rm L}$}

\put(2,87){$2$}
%\put(-2,90){\vector(1,0){2}}
\put(-8,88){$\rhd$}
%\put(10,44.4){\circle*{2}}
\put(34,87){$1$}
\put(34,127){${2}$}
\put(34,187){$1$}
\put(34,227){$1$}


\put(26,118){\line(1,0){15}}
\put(26,164){\line(1,0){15}}
\put(26,118){\line(0,1){46}}
\put(41,118){\line(0,1){46}}


%\put(30,90){\oval(18,40)}
%\put(39,84){\vector(1,-2){26}}

%XXXXXXXXXXXXXXXXXXXXXXXXXXXXXXXXXXXXXXXXXXX

%\put(10,40){\circle{10}}
\put(80,60){\circle*{4}}
\put(80,80){\circle*{4}}
\put(80,100){\circle*{4}}
\put(80,120){\circle*{4}}
\put(80,140){\circle*{4}}
\put(80,160){\circle*{4}}
\put(80,180){\circle*{4}}
\put(80,140){\circle*{4}}
\put(80,220){\circle*{4}}
\put(80,260){\circle*{4}}
\put(80,280){\circle*{4}}
\put(100,100){\circle*{4}}
\put(100,140){\circle*{4}}
\put(100,160){\circle*{4}}
\put(100,200){\circle*{4}}
\put(100,240){\circle*{4}}
\put(100,260){\circle*{4}}


\put(80,120){\line(1,-1){20}}
\put(80,180){\line(1,-1){20}}
\put(80,220){\line(1,-1){20}}
\put(80,280){\line(1,-1){20}}
%\put(20,240){\line(1,-1){20}}
%\put(20,60){\line(1,-1){20}}

\put(80,40){\line(0,1){240}}
\put(100,80){\line(0,1){20}}
\put(100,120){\line(0,1){40}}
\put(100,180){\line(0,1){20}}
\put(100,220){\line(0,1){40}}
%\put(40,160){\line(0,1){20}}
%\put(40,200){\line(0,1){20}}
%\put(60,60){\line(0,1){20}}

\put(92,111){${\rm S}$}
\put(92,171){${\rm S}$}
\put(92,211){${\rm S}$}
\put(92,271){${\rm S}$}

\put(70,66){${\rm L}$}
\put(70,106){${\rm A}$}
\put(70,146){${\rm L}$}

%\put(65,126){${\fbox{\rmL}}$}
%\put(64,196){${\fbox{\rmA}}$}
\put(72,129){\circle{12}}
\put(69,126.2){$\rm L$}
\put(72,199){\circle{12}}
\put(69,196.2){$\rm A$}

\put(70,166){${\rm A}$}
%\put(64,257){$\fbox{\rm A}$}
\put(70,236){${\rm L}$}
\put(70,266){${\rm A}$}
\put(104,146){${\rm L}$}
\put(104,246){${\rm L}$}

\put(72,87){$2$}
\put(72,47){$2$}
\put(111,88){$\lhd$}
%\put(112,90){\vector(-1,0){2}}
%\put(10,44.4){\circle*{2}}
\put(104,87){$1$}
\put(104,127){${2}$}
\put(104,187){$1$}
\put(104,227){$1$}

\put(96,178){\line(1,0){15}}
\put(96,204){\line(1,0){15}}
\put(96,178){\line(0,1){26}}
\put(111,178){\line(0,1){26}}




%XXXXXXXXXXXXXXXXXXXXXXXXXXXXXXXXXXXXXXXXXXX

%\put(10,40){\circle{10}}
\put(150,60){\circle*{4}}
\put(150,80){\circle*{4}}
\put(150,100){\circle*{4}}
\put(150,120){\circle*{4}}
\put(150,140){\circle*{4}}
\put(150,160){\circle*{4}}
\put(150,180){\circle*{4}}
\put(150,140){\circle*{4}}
\put(150,220){\circle*{4}}
\put(150,260){\circle*{4}}
\put(150,280){\circle*{4}}
\put(170,80){\circle*{4}}
\put(170,100){\circle*{4}}
\put(170,140){\circle*{4}}
\put(170,160){\circle*{4}}
\put(170,200){\circle*{4}}
\put(170,240){\circle*{4}}
\put(170,260){\circle*{4}}


\put(150,120){\line(1,-1){20}}
\put(150,180){\line(1,-1){20}}
\put(150,220){\line(1,-1){20}}
\put(150,280){\line(1,-1){20}}
%\put(20,240){\line(1,-1){20}}
%\put(20,60){\line(1,-1){20}}

\put(150,40){\line(0,1){240}}
\put(170,60){\line(0,1){40}}
\put(170,120){\line(0,1){40}}
\put(170,180){\line(0,1){20}}
\put(170,220){\line(0,1){40}}
%\put(40,160){\line(0,1){20}}
%\put(40,200){\line(0,1){20}}
%\put(60,60){\line(0,1){20}}

\put(162,111){${\rm S}$}
\put(162,171){${\rm S}$}
\put(162,211){${\rm S}$}
\put(162,271){${\rm S}$}

\put(140,66){${\rm L}$}
\put(140,106){${\rm A}$}
\put(140,146){${\rm L}$}

\put(142,239){\circle{12}}
\put(139,236.2){$\rm L$}
\put(142,269){\circle{12}}
\put(139,266.2){$\rm A$}

\put(140,166){${\rm A}$}
%\put(64,257){$\fbox{\rm A}$}
\put(140,126){${\rm L}$}
\put(140,196){${\rm A}$}
\put(174,146){${\rm L}$}
\put(174,246){${\rm L}$}

\put(142,87){$2$}
\put(142,47){$2$}
\put(181,68){$\lhd$}
\put(181,128){$\lhd$}
%\put(112,90){\vector(-1,0){2}}
%\put(10,44.4){\circle*{2}}
\put(174,67){$1$}
\put(174,87){$1$}
\put(174,127){${2}$}
\put(174,187){$1$}
\put(174,227){$1$}

\put(166,218){\line(1,0){15}}
\put(166,264){\line(1,0){15}}
\put(166,218){\line(0,1){46}}
\put(181,218){\line(0,1){46}}




%XXXXXXXXXXXXXXXXXXXXXXXXXXXXXXXXXXXXXXXXXXX

%\put(220,20){\circle*{4}}
%\put(220,40){\circle*{4}}
\put(220,60){\circle*{4}}
\put(220,80){\circle*{4}}
\put(220,100){\circle*{4}}
\put(220,120){\circle*{4}}
\put(220,140){\circle*{4}}
\put(220,160){\circle*{4}}
\put(220,180){\circle*{4}}
\put(220,140){\circle*{4}}
\put(220,220){\circle*{4}}
\put(220,260){\circle*{4}}
\put(220,280){\circle*{4}}
\put(240,40){\circle*{4}}
\put(240,60){\circle*{4}}
\put(240,80){\circle*{4}}
\put(240,100){\circle*{4}}
\put(240,100){\circle*{4}}
\put(240,140){\circle*{4}}
\put(240,160){\circle*{4}}
\put(240,200){\circle*{4}}
\put(240,240){\circle*{4}}
\put(240,260){\circle*{4}}
\put(240,120){\circle*{4}}
\put(260,100){\circle*{4}}


\put(220,120){\line(1,-1){20}}
\put(220,180){\line(1,-1){20}}
\put(220,220){\line(1,-1){20}}
\put(220,280){\line(1,-1){20}}
\put(240,120){\line(1,-1){20}}
%\put(20,240){\line(1,-1){20}}
%\put(20,60){\line(1,-1){20}}

\put(220,40){\line(0,1){240}}
\put(240,20){\line(0,1){80}}
\put(240,120){\line(0,1){40}}
\put(240,180){\line(0,1){20}}
\put(240,220){\line(0,1){40}}
\put(260,80){\line(0,1){20}}
%\put(40,160){\line(0,1){20}}
%\put(40,200){\line(0,1){20}}
%\put(60,60){\line(0,1){20}}

\put(232,111){${\rm S}$}
\put(232,171){${\rm S}$}
\put(232,211){${\rm S}$}
\put(232,271){${\rm S}$}

%\put(210,26){${\rm L}$}
\put(210,66){${\rm L}$}
\put(210,106){${\rm A}$}
\put(210,146){${\rm L}$}

%\put(142,239){\circle{12}}
\put(210,236){$\rm L$}
%\put(142,269){\circle{12}}
\put(210,266){$\rm A$}

\put(210,166){${\rm A}$}
%\put(64,257){$\fbox{\rm A}$}
\put(210,126){${\rm L}$}
\put(210,196){${\rm A}$}
\put(244,46){${\rm L}$}
\put(244,146){${\rm L}$}
\put(244,246){${\rm L}$}
\put(252,111){${\rm L}$}

\put(212,87){$2$}
\put(212,47){$2$}
%\put(212,7){$1$}
%\put(251,68){$\lhd$}
%\put(202,48){$\rhd$}
%\put(112,90){\vector(-1,0){2}}
%\put(10,44.4){\circle*{2}}
\put(244,27){$1$}
\put(244,67){$1$}
\put(244,87){$1$}
\put(244,127){${2}$}
\put(244,187){$1$}
\put(244,227){$1$}
\put(264,87){$1$}

%\put(166,218){\line(1,0){15}}
%\put(166,264){\line(1,0){15}}
%\put(166,218){\line(0,1){46}}
%\put(181,218){\line(0,1){46}}





%XXXXXXXXXXXXXXXXXXXXXXXXXXXXXXXXXXXXXXXXXXX
\end{picture}

\caption{Examples of delayed, focused $\beta$-reduction}
\label{ExaLamDel}
\end{figure}

\begin{Note} Figure~\ref{ExaLamDel}, $(iii)$, shows that it may occur that two (or more) num-labels occur adjacently. See the right-most part of the figure, at the end of the complete path $\rmA \, \rmL \, \rm A \, \rmA \, \rmL \, \rmL \, \rmS \, 1 \, 1$. The first (or top-) label $1$ is an inner num-variable, the other $1$ is outer. Since the ${\rm A}$-block of the first $1$ is $\rmA \, \rmA \, \, \rmL \, \rmL \, \rmS \, 1$, the $\rmL$-block of the second $1$ is $\rmL \, \rmA \, \rmA \, \rmL \, \rmL \, \rmS \, 1 \, 1$, so the second $1$ is bound to the leftmost, uppermost~$\rmL$. %(In this example case, there is also an $\rmA$-block for the bottom-$1$.)

In Figure~\ref{ExaLamDel}, $(iv)$, both mentioned labels have become inner ones.
%This has its effect on the notation of paths, in which num-labels may occur adjacently. For example, in Figure~\ref{ExaLamDel}, $(iv)$, we have the

%Multiple labels are a natural consequence of algorithm ${{\cal P}_{fre}}$. More than two labels for one vertex are possible, as well. We emphasize that we consider the mentioned cases each time as labels for only{\em one\/} vertex, the labels obeying a {\em linear order\/} (`top-to-bottom').


\end{Note}

\subsection{From name-carrying to name-free lambda-terms, and vice versa}




In this section we compare ${\cal T}^{\it car}$ with ${\cal T}^{\it fre+}$ and define mappings between them. It turns out that there is a natural relation between ${\rm L}$-blocks in $\Tcar$ and ${\rm L}$-blocks in $\Tfrep$, and between ${\rm A}$-blocks in $\Tcar$ and in $\Tfrep$ (cf.\ Definitions~\ref{DefLBlo}~$(i)$, \ref{DefABlo}~$(i)$ and \ref{DefALBlo}).

We only consider {\it closed\/} $\lambda$-trees, i.e., $\lambda$-trees in which all variables have been bound. We only consider $\lambda$-abstraction, so there are no ${\rm P}_x$'s or ${\rm P}$'s in the sets of terms studied below. (Hence, we limit ourselves to the set $\lambda_{\underline{\omega}}$ in Barendregt's cube; cf.\ \citealp{Bar92}.) As to the mappings to be described, $\Pi$-abstraction behaves similarly to $\lambda$-abstraction.



\begin{Def}
$(i)$ Let ${\bf t} \in {\cal T}^{\it car}$. The {\em structure\/} of ${\bf t}$, or ${\it str}({\bf t})$, is ${\bf t}$ in which all variables have been omitted. This concerns the variables at the leaves, but also the subscripts of labels ${\rm L}_x$, for any $x$: all ${\rm L}_x$'s have been stripped of their~ $x$. The other labels remain as they are.

$(ii)$ Let ${\bf u} \in {\cal T}^{\it fre+}$. The {\em structure\/} of ${\bf u}$, or ${\it str}({\bf u})$, is (again) ${\bf u}$ in which all num-variables have been omitted. This concerns both the outer and the inner variables. The labels are preserved.

Moreover, in ${\it str}({\bf u})$ all edges which did belong to inner labels have been erased (i.e., every path $p \, m \, q \in^\wedge_\vee {\bf u}$, where $q \not = \epsilon$, has been replaced by $p \, q$; this may, of course, need more than one round).
\end{Def}

%Note that, in both cases, the final edges (the leaves) remain in the structures, albeit without labels.

\begin{Def}\label{DefMapStr} We define a mapping from $\Tcar$ to $\Tfre$. Let ${\bf t} \in \Tcar$. Then $[{\bf t}] \in \Tfre$ is the tree with the same structure as ${\bf t}$ and with numbers $n$ instead of the variable names at the leaves of ${\bf t}$. These numbers are chosen such that they {\em respect} the bindings. That is, if $p \, \rmL_x \, q \, x$ is a path in ${\bf t}$, then the corresponding path $p' \, \rmL \, q' \, n$ in $[{\bf t}]$ has the property that $n =  \lVert q' \lVert  + 1$.
\end{Def}

The reverse mapping is slightly more complicated. We include $\lambda$-trees with inner variables, so we describe a mapping from $\Tfrep$ to $\Tcar$. This mapping consists of three stages.

\smallskip

Let ${\bf u}$ be a $\lambda$-tree in $\Tfrep$.

\smallskip

$(i)$ {\it Add names\/} to ${\rm L}$-labels: provide all ${\rm L}$-labels in ${\bf u}$ with a name, such that

\smallskip

$p \, {\rm L}_x,~ q \, {\rm L}_y \inw {\bf u} \Rar (x \equiv y \Rar p \equiv q)$.

\smallskip

By this process, the various ${\rm L}$-labels in ${\bf u}$ get names that are {\it different\/} from each other. We call the paths thus obtained {\it enriched paths\/}.

$(ii)$ {\it Replace\/} the outer num-labels by appropriate names: consider an arbitrary enriched complete path $p \, {\rm L}_x \, q \, n$ in ${\bf u}$, where ${\rm L} \, q \, n$ is an ${\rm L}$-block in the original tree ${\bf u}$. (Since such a path is considered to be complete, $n$ is an outer num-label.) Replace $n$ by $x$. Do so in all enriched complete paths.

$(iii)$ {\it Erase\/} all inner num-labels and their edges. (So the tree changes, it becomes more `compact'.)

\begin{Def} Let ${\bf u} \in \Tfrep$. The combined procedure $(i)$, followed by $(ii)$, followed by $(iii)$, applied to ${\bf u}$, gives a mapping from $\Tfrep$ to $\Tcar$. We denote the resulting tree as $\langle {\bf u} \rangle$.
\end{Def}

\begin{Note} A possibility for speeding up the procedure in stage~$(ii)$, is to use the method of {\em backtracking\/}, applied to the whole tree, instead of doing the job path by path.
\end{Note}

\begin{Lem}\label{LemIde} $(i)$ Let ${\bf t} \in \Tcar$. Then $\langle [ \bft ] \rangle \equiv_{\alpha} \bft$.

$(ii)$ Let ${\bf u} \in \Tfre$. Then $[ \langle {\bf u} \rangle ] \equiv_{\alpha} {\bf u}$.

$(iii)$ Let ${\bf u} \in \Tfrep$. Then $\langle [ \langle {\bf u} \rangle ] \rangle \equiv_{\alpha} \langle {\bf u} \rangle$.
\end{Lem}



%\subsection{The binding relation in name-free lambda-terms}\label{SecBinNam}



\subsection{The behaviour of name-free lambda-trees under beta-reduction}

In this section, we compare delayed, focused reduction on $\lambda$-terms in $\Tfrep$ with focused reduction for $\Tcar$.  We choose to study {\it focused\/} $\beta$-reduction only, since the behaviour of any non-focused $\beta$-reduction can be described in terms of the focused one. \marginpar{v5. \ref{SecDelUpd}--3}
%We discuss erasing reduction later in this chapter.


%YYYYYYYY

For that purpose, it is necessary that we can refer to redexes in both $\Tcar$ and $\Tfrep$.

\begin{Def} $(i)$ Let $r \equiv p \, {\rm A} \, b \, {\rm L}_x \, q \, x\inwv {\bf t} \in \Tcar$, with balanced path $b$; then
%${\it tree}(r)$ together with ${\it tree}(p \, \rmS)$ {\it mark\/} a redex in ${\bf t} \in \Tcar$. The string
$p$ identifies a redex and the pair $\rho \equiv (p,q)$ identifies a {\em focused\/} reduction, called $(p,q)$-reduction. (Cf.\ Definition~\ref{DefBetFreFoc}.)

We write ${\bf t} \rar_f^\rho {\bf t}'$ for the focused $\beta$-reduction generated by $\rho$.


Now look at $\Tfrep$. Assume that ${\bf u} \in \Tfrep$ has the same structure as ${\bf t}$. There exists a path $r' \equiv p' \, \rmA \, b' \, \rmL \, q' \, k \inwv {\bf u}$ with the same structure as $r$. That is, adding appropriate variable names and striking out all inner variables in $r'$, results in $r$.  Then the pair $\rho' \equiv (p',q')$ identifies a focused reduction in $\Tfrep$.



We say that the pairs $\rho$ and $\rho'$ {\em correspond\/} to each other and we write $[\rho]$ for~$\rho'$. We write ${\bf u} \rar_{\it df}^{[\rho]} {\bf u}'$ for the delayed, focused $\beta$-reduction generated by $[\rho]$.

$(ii)$ This correspondence is symmetric. Starting from $r' \equiv  p' \, {\rm A} \, b' \, {\rm L} \, q' \, k \inwv {\bf u} \in \Tfrep$ with balanced path $b'$, and assuming that ${\bf t} \in \Tcar$ has the same structure as ${\bf u}$, there is a path $r \equiv p \, \rmA \, b \, {\rmL}_x \, q \, x \inwv {\bf t}$ with the same structure as $r'$. Again, $\rho$ and $\rho'$ correspond to each other. We write $\langle \rho \rangle$ for~$\rho'$ and ${\bf t} \rar_{\it f}^{\langle \rho \rangle} {\bf t}'$ for the focused $\beta$-reduction generated by $\langle \rho \rangle$.
\end{Def}

\begin{Lem}\label{LemStrRed}
Let ${\bf t} \in \Tcar$ and assume that $\rho$ identifies a focused $\beta$-reduction such that ${\bf t} \rar_f^\rho {\bf t}'$.

Further, let ${\bf u} \in \Tfrep$ and ${\it str}({\bf u}) \equiv {\it str}({\bf t})$. Then ${\bf u} \rar_{\it df}^{[\rho]} {\bf u}'$ and ${\it str}({\bf u}') \equiv {\it str}({\bf t}')$.
\end{Lem}


%\smallskip

In the following lemma, we compare a sequence of focused $\beta$-reductions in $\Tcar$ with a corresponding sequence of delayed, focused $\beta$-reductions in $\Tfrep$.

\begin{Lem}\label{LemVarLes} Let ${\bf t}_1 \in \Tcar$ and assume that ${\bf t}_1 \rar_{f}^{\rho_1} {\bf t}_2 \rar_{f}^{\rho_2} \ldots \rar_{f}^{\rho_{n-1}} {\bf t}_n$, where $\rho_1 \ldots \rho_{n-1}$ identify  the chosen one-step focused reductions.

Let ${\bf u}_1 \equiv [{\bf t}_1] \in \Tfre$. Then there is a delayed, focused reduction:

${\bf u}_1 \rar_{\it df}^{[\rho_1]} {\bf u}_2 \rar_{\it df}^{[\rho_2]} \ldots \rar_{\it df}^{[\rho_{n-1}]} {\bf u}_n$, where ${\bf u}_i \in \Tfrep (2 \leq i \leq n)$.

Moreover, ${\it str}({\bf t}_i) \equiv {\it str}({\bf u}_i)$ for all $1 \leq i \leq n$.

%\noindent %$(i)$,
 %$(i)$ ${\bf t}'_1 \equiv [{\bf t}_1]$, and $(ii)$ for all $1\leq i \leq n$, ${\bf t}_i$ and ${\bf t}'_i$ have the same var-less skeleton.
\end{Lem}

{\it Proof\/} Induction. ${\bf t}_1$ and ${\bf u}_1$ have the same structure, by Definition~\ref{DefMapStr}. Use Lemma~\ref{LemStrRed}. $\Box$

\smallskip

Note that, although ${\bf u}_i \in \Tfrep$, it will in general {\em not\/} be the case that ${\bf u}_i \equiv [{\bf t}_i]$ for $2 \leq i \leq n$, since these ${\bf u}_i$ may contain inner variables, contrary to $[{\bf t}_i]$.


%\newpage

\begin{Lem}\label{LemAblBou}
$(i)$ Let ${\bf u} \in \Tfrep$ and $r \equiv p \, \rmS \, q \, n \inwv {\bf u}$. Then the leaf $n$ is bound by a certain $\rmL$ in either $p$ or $q$. I.e., $p \equiv p_1 \, \rmL \, p_2$ or $q \equiv q_1 \, \rmL \, q_2$ for this $\rmL$.

$(ii)$ Let $a$ be any ${\rmA}$-block. Let $r'$ be $r$ of $(i)$ in which $\rmS$ has been replaced by $a$. Then leaf $n$ is bound by an $\rmL$ in either $p$ or $q$ at a position corresponding to the one in $(i)$.
\end{Lem}

{\it Proof\/} If $\rmL \in q$, then trivial.

Assume $\rmL \in p$, say $p \equiv p_1 \, \rmL \, p_2$. Apply algorithm ${{\cal P}_{fre}}$:

$(i)$ $p \, \rmS \, q \, (n,0) \, n \rrar^{(1)} p \, \rmS \, (m,0) \, q \, n \rrar p \, (m,0) \, \rmS \, q \, n \equiv p_1 \, \rmL \, p_2 \, (m,0) \, \rmS \, q \, n \rrar^{(2)} p_1 \, (0,0) \, \rmL \, p_2 \, \rmS \, q \, n$.

$(ii)$ Let $a \equiv a' \, k$. Then: $p \, a \, q \, (n,0) \, n \rrar^{{\rm see\,}(1)} p \, a' \, k \, (m,0) \, q \, n \rrar^{{\rm push}\, (m,0)} p \, a' \, (k,1) \, k \, q \, n \rrar^{{\rm Def. ~\ref{DefALBlo}},(ii)} p \, (0,0) \, a' \, k \, q \, n \rrar^{{\rm pop}\,(m,0)} p \, (m,0) \, a \, q \, n \equiv$

\noindent $p_1 \, \rmL \, p_2 \, (m,0) \, a \, q \, n \rrar^{{\rm see\,}(2)} p_1 \, (0,0) \, \rmL \, p_2 \, a \, q \, n$.

So $\rmL \, p_2 \, \rmS \, q \, n$ and  $\rmL \, p_2 \, a \, q \, n$ are ${\rmL}$-blocks by Definition~\ref{DefALBlo},$(i)$, and $n$ is bound by an $\rmL$ at corresponding positions in both cases. $\Box$

\begin{Note}\label{NotThe} At transition~{\rm (1)} in $(i)$, the reached state must be $(m,0)$ for some $m$, and not $(m,l)$ for some $l > 0$, since the latter case only occurs when the displayed $\rmS$ is part of an $r$-block. This would contradict what we said in Note~\ref{ExpActPro}, $(v)$.

%$v_3 \rar v_4$: Since $m$ is bound by $\rmL$ (Lemma~\ref{-});

%$w_1 \rar w_2$: The action of ${\cal P}_{fre}$ on $w_1 \rar w_2$ is similar to the action on $v_1 \rar v_2$;

%$w_2 \rar w_3$: The path $\rmA \, b' \, \rmL \, q' \, n$ is an $\rmA$-block, since $n$ is bound by the $\rmL$ in that path; hence: $(m',0)$ is pushed into the stack in $w_2$ and pops out again in $w_3$ (Lemma~\ref{_}).

\end{Note}


%Now we have the following theorem, establishing the correspondence between name-carrying and name-free reductions.

\begin{The}\label{TheRedSeq} Let ${\bf t}_1 \in \Tcar$ and assume that ${\bf t}_1 \rar_{f}^{\rho_1}  \ldots \rar_{f}^{\rho_{n-1}} {\bf t}_n$.

Let ${\bf u}_1 \equiv [{\bf t}_1] \in \Tfre$ and
${\bf u}_1 \rar_{\it df}^{[\rho_1]} \ldots \rar_{\it df}^{[\rho_{n-1}]} {\bf u}_n$ (cf.\ Lemma~\ref{LemVarLes}).

Then for all $1 \leq i \leq n$: $\langle {\bf u}_i \rangle \equiv_{\alpha} {\bf t}_i$.

%Let ${\bf t}'_n$ and ${\bf t}'_{n + 1}$ be in $\Tfrep$ such that ${\bf t}'_n \rar_{f}^{[\rho_{n}]} {\bf t}'_{n + 1}$. Assume that $\langle {\bf t}'_n \rangle \equiv {\bf t}_n$. Then $\langle {\bf t}'_{n + 1} \rangle \equiv {\bf t}_{n + 1}$.
\end{The}


{\it Proof\/} Induction.

$(i)$ Let $i = 1$. Then $\langle {\bf u}_1 \rangle \equiv \langle [{\bf t}_1] \rangle \equiv_\alpha \bft_1$ by Lemma~\ref{LemIde}.

$(ii)$ %Let $\bft'_i \rar_{\it df}^{[\rho_{i-1}]} \bft'_{1 + 1}$ and $\bft_i \rar_f^{\rho_{i-1}} \bft_{1 + 1}$.
Assume that $\langle {\bf u}_i \rangle \equiv_{\alpha} {\bf t}_i$. Now ${\it str}({\bf u}_i) \equiv {\it str}({\bf t}_i)$ by Lemma~\ref{LemVarLes}.

Since ${\bf t}_i \rar_f^{\rho_i} {\bf t}_{i+1}$ and ${\bf u}_i \rar_{\it df}^{[\rho_i]} {\bf u}_{i+1}$, also ${\it str}({\bf u}_{i+1}) \equiv {\it str}({\bf t}_{i+1})$ (Lemma~\ref{LemStrRed}).

Let $\rho_{i}$ be an $(p,q)$-reduction, so there is a path $p \, \rmA \, b \, \rmL_x \, q \, x \in {\bf t}_i$, with $b$ a balanced path. The corresponding path in ${\bf u}_i$ is $p' \, \rmA \, b' \, \rmL \, q' \, n$, with ${\it str}(p') \equiv {\it str}(p)$, and similarly for $b'$ and $q'$. By the reduction $[\rho_i]$, this path is transformed into the grafted tree $p' \, \rmA \, b' \, \rmL \, q' \, n \, {\it tree}(p' \, \rmS)$ in ${\bf u}_{i+1}$.

Note that the paths in the mentioned grafted tree are the only paths that we have to consider, since all other paths remain unchanged, both in the transition from ${\bf t}_i$ to ${\bf t}_{i+1}$ as from ${\bf u}_i$ to ${\bf u}_{i+1}$.

So let $p' \, \rmA \, b' \, \rmL \, q' \, n \, r' \, m$ be a complete path in $p' \, \rmA \, b' \, \rmL \, q' \, n \, {\it tree}(p' \, \rmS)$, so $r' \, m \inwv {\it tree}(p' \, \rmS)$. We have to check whether $m$ is bound and if so, to locate the binding $\rmL$ and compare this with the situation in ${\bf t}_{i+1}$.

Now $r' \, m$ also occurs in the path $p'\, \rmS \, r' \, m$ occurring in both ${\bf u}_{i}$ and ${\bf u}_{i+1}$. Since by induction $\langle {\bf u}_i \rangle \equiv_{\alpha} {\bf t}_i$, we have that there is a path $p \, \rmS \, r \, y \in {\bf t}_i$ for some~$r$ with ${\it str}(r) \equiv {\it str}(r')$.

The final $y$ in this path $p \, \rmS \, r \, y$ is bound by an $\rmL_y$ in either $p$ or $r$.

\smallskip

$(i)$ Assume $y$ is bound in $p$. Then $p \equiv p_1 \, \rmL_y \, p_2$. Also, $p' \equiv p'_1 \, \rmL \, p'_2$ and the final $m$ in $p'_1 \, \rmL \, p'_2 \, \rmS \, r' \, m$ is bound by the mentioned $\rmL$.

Now note that the path $a \equiv \rmA \, b' \, \rmL \, q' \, n$ is an $\rmA$-block in ${\bf u}_{i}$, so also in ${\bf u}_{i+1}$. It follows from Lemma~\ref{LemAblBou} that the final $m$ in $p'_1 \, \rmL \, p'_2 \, \rmA \, b' \, \rmL \, q' \, n \, r' \, m$ is bound by the $\rmL$ following $p'_1$.

Correspondingly, in ${\bf t}_{i+1}$ we have that the final variable $z$ in $p_1 \rmL_y \, p_2 \, \rmA \, b \, \rmL_x \, q \, r \, z$ must be $z \equiv y$. So the two last-mentioned paths are matching.



\smallskip

$(ii)$ Assume $y$ is bound in $r$. Then $r \equiv r_1 \, \rmL_y \, r_2$. Also, $r' \equiv r'_1 \, \rmL \, r'_2$ and the final $m$ in $p' \, \rmS \, r'_1 \, \rmL \, r'_2 \, m$ is bound by the mentioned $\rmL$.

So also the final $m$ in $p' \, \rmA \, b' \, \rmL \, q' \, n \, r'_1 \, \rmL \, r'_2 \, m \in {\bf u}_{i+1}$ is bound by the $\rmL$ following~$r'_1$.

In ${\bf t}_{i+1}$ we have that the corresponding path is $p \, \rmA \, b \, \rmL_x \, q \, r_1 \, \rmL_z \, r_2 \, z$ for some~$z$. (Note that all binding variables in ${\bf t}_{i+1}$ must be different, so $\alpha$-reduction should change $\ldots \rmL_y \ldots y$ into something like $\ldots \rmL_z \ldots z$ in the copy of ${\it tree}(p \, \rmS)$ that pops up in ${\bf t}_{i+1}$ by the $\rar_f$-reduction.) Clearly, the bindings of $m$ and $z$ are matching.

\smallskip

It follows that all bindings in ${\bf u}_{i+1}$ correspond one-to-one to the bindings in~${\bf t}_{i+1}$. This is enough to conclude that $\langle {\bf u}_{i+1} \rangle \equiv_{\alpha} {\bf t}_{i+1}$. $\Box$

\medskip

A corresponding theorem holds the other way round.

\begin{The}\label{TheRedSeq2} Let ${\bf u}_1 \in \Tfre$ and assume that ${\bf u}_1 \rar_{df}^{\rho_1}  \ldots \rar_{df}^{\rho_{n-1}} {\bf u}_n$.

Let ${\bf t}_1 \equiv \langle {\bf u}_1 \rangle \in \Tcar$ and
${\bf t}_1 \rar_{\it f}^{\langle \rho_1 \rangle} \ldots \rar_{\it f}^{[\rho_{n-1}]} {\bf t}_n$.

Then for all $1 \leq i \leq n$: $[{\bf t}_i] \equiv_{\alpha} {\bf u}_i$.
\end{The}

The proof is similar to that of Theorem~\ref{TheRedSeq}.

\subsection{Theorems on delayed updating}\label{SecTheDel}

We single out the $\lambda$-trees in $\Tfrep$ that `originate' from $\Tfre$ and give a number of theorems and lemmas concerning these trees.

\begin{Def}\label{DefLegOri}
If ${\bf u} \in \Tfrep$ has the property that there is an ${\bf u_0} \in \Tfre$ with ${\bf u_0} \rar_{\it df} {\bf u}$, then ${\bf u}$ is called {\it legal\/}. Such an ${\bf u_0}$ is called an {\it origin\/} of ${\bf u}$.
\end{Def}

\begin{Lem}\label{LemUniOri}
Each ${\bf u} \in \Tfrep$ has a unique origin.
\end{Lem}

We continue with some lemma's concerning inner and outer variables. (See also Lemma~\ref{LemALblo}.)

\begin{Lem}\label{LemInnOut}

Let ${\bf u} \in \Tfrep$ be legal. Assume that $p \inwv {\bf u}$ such that $p \equiv p_1 \, n \, p_2$.

$(i)$ Let $p_2$ be non-empty (so $n$ is an {\em inner\/} variable of ${\bf u}$). Then $n$ is the final variable of a corresponding $\rmA$-block enclosing a corresponding $\rmL$-block, both being subpaths of $p_1 \, n$. Moreover, the $\rmA$-block and the $\rmL$-block are corresponding, and the front-$\rmL$ of the $\rmL$-block binds $n$.

$(ii)$ Let $p_2$ be empty (so $n$ is an {\em outer\/} variable of ${\bf u}$). Then $n$ is the final variable of a corresponding $\rmL$-block being a subpath of $p_1 \, n$. Moreover, the front-$\rmL$ of the $\rmL$-block binds $n$. (Note: there may also be a corresponding $\rmA$-block, enclosing the $\rmL$-block, but this is not necessarily so.)
\end{Lem}

Next, we give a general theorem for reductions in $\Tfre$ and $\Tfrep$ (cf.\ Theorem~\ref{TheConRed}).

\begin{The}\label{TheConNamFre}
$(i)$ Each of the reductions $\rar_b$, $\rar_f$ and $\rar_e$ is confluent for name-free paths.

$(ii)$ The reduction $\rar_{\it df}$ is confluent.
\end{The}

{\it Proof of $(ii)$\/} Use Theorem~\ref{TheRedSeq}. $\Box$

\medskip


\begin{Lem}
Let ${\bf u} \in \Tfrep$ be legal. Let ${\bf u_0}$ be defined as the set of all paths $p \, n \inw {\bf u}$ such that $p$ has no inner variables. Then ${\bf u_0}$ is the origin of ${\bf u}$.
\end{Lem}

%Legality is decidable, as the following lemma shows.

\begin{Lem}
Let ${\bf u}$ be a legal $\lambda$-tree with origin ${\bf u}_0$. Then there is a procedure to find an $\rar_{\it df}$-reduction sequence such that ${\bf u_0} \rrar_{\it df} {\bf u}$.
\end{Lem}

\begin{Lem}\label{LemDelRed}
Let ${\bf u}$ be a legal $\lambda$-tree and ${\bf u} \rrar_{\it df} {\bf u}'$. Assume that path $p \inw {\bf u}$ and that ${\it tree}(p)$ is a redex in ${\bf u}$, with argument ${\it tree}(p \, S)$. Then:

$(i)$ ${\it tree}(p)$ is also a redex in ${\bf u}'$,

$(ii)$ argument ${\it tree}(p \, S)$ in ${\bf u}$ is a subtree of argument ${\it tree}(p \, S)$ in ${\bf u}'$.
\end{Lem}



%\begin{Def}\label{DefOrdVar}
%Let ${\bf u}' \in \Tfrep$ be legal. We define a linear order $<$ between all {\em inner} variables of ${\bf u}'$ as follows.

%Assume that $n, \, l$ are inner variables in ${\bf u}'$, where $n \inw p \, n$ and $l \inw q \, l$.

%$(i)$ If $p \subseteq q$, then $n < l$.

%$(ii)$  If $q \subseteq p$, then $l < n$.

%$(iii)$ Otherwise, in $p < q$, then $n < l$.
%\end{Def}
We now discuss some important consequences of Theorem~\ref{TheImpInc}.

\begin{Def}\label{DefTra}
Let ${\bf u} \in \Tfrep$ and $p \inw {\bf u}$. The {\em trail\/} of $p$, or ${\it trail\/}(p)$, is the non-empty sequence of edges obtained from $p$ by omitting all labels, but preserving the 'directions' of the edges; a {\em direction\/} being ${\tt l}$ (`left') or ${\tt r}$ (`right'). This is done such that labels $\rmA$ and $\rmL$  are reflected as ${\tt l}$ in ${\it trail\/}(p)$ and label $\rmS$ as ${\tt r}$. For the (unary) num-labels (inner and outer), we choose ${\tt l}$.
\end{Def}

\begin{Exa} Let $p$ be $\rmL \, \rmL \, \rmA \, 2 \, \rmS \, \rmL \, \rmS \, 1 \, \rmL$, then ${\it trail\/}(p) \equiv {\tt l} \, {\tt l} \, {\tt l} \, {\tt l} \, {\tt r} \, {\tt l} \, {\tt r} \, {\tt l} \, {\tt l}$.
\end{Exa}

The following lemma is a consequence of the construction principles for trees.

\begin{Lem}\label{LemTraIde}  Let ${\bf u} \in \Tfrep$ and $p_1, p_2 \inw {\bf u}$ such that $|p_1| = |p_2|$. If ${\it trail\/}(p_1) \equiv {\it trail\/}(p_2)$, then $p_1 \equiv p_2$.\marginpar{v5. \ref{SecDelUpd}--4}
\end{Lem}

{\em Proof}~~ First we show that, if $p_1 \equiv q \, \ttl_1 \, q'$ and $p_2 \equiv q \, \ttl_2 \, q''$, where $\ttl_1$ and $\ttl_2$ are labels, then $\ttl_1 \equiv \ttl_2$. This is done by distinguishing the different possibilities for $\ttl_1$ and using construction principles of $\lambda$-trees.

Next, apply induction on $|q|$. Since $|p_1| = |p_2|$, also $p_1 \equiv p_2$. $\Box$

\begin{The}\label{The}\label{TheTraTra}
Let ${\bf u} \in \Tfre$ and assume ${\bf u} \rrar_{\it df} {\bf u_1}$ and ${\bf u} \rrar_{\it df} {\bf u_2}$. Assume that $p_1 \inw {\bf u_1}$ and $p_2 \inw {\bf u_2}$ such that ${\it trail\/}(p_1) \equiv {\it trail\/}(p_2)$.
Then $p_1 \equiv p_2$.
\end{The}

{\it Proof\/} By `confluence' (Theorem~\ref{TheConNamFre}), there exists ${\bf u_3} \in \Tfrep$ such that ${\bf u_1} \rrar_{\it df} {\bf u_3}$ and ${\bf u_2} \rrar_{\it df} {\bf u_3}$. Then by Theorem~\ref{TheImpInc}, ${\bf u_1} \subset {\bf u_3}$ and ${\bf u_2} \subset {\bf u_3}$. Hence $p_1, p_2 \inw {\bf u_3}$, and by Lemma~\ref{LemTraIde}, $p_1 \equiv p_2$. $\Box$

\medskip

\begin{Def} $(i)$ By ${\cal T}_{\it bin}$ we denote the infinite binary tree that is the graphic representation of the (infinite) set of all infinite lists composed of ${\tt l}$'s and ${\tt r}$'s.

$(ii)$ Each trail ${\tt t}$ is a non-empty, finite initial part of some element of ${\cal T}_{\it bin}$. We denote this correspondence by ${\tt t} \inw {\cal T}_{\it bin}$.

$(iii)$ An edge in ${\cal T}_{\it bin}$ is called a {\em position\/} in ${\cal T}_{\it bin}$.
\end{Def}

\begin{Def}

$(i)$ Let ${\tt t} \inw {\cal T}_{\it bin}$ and assume that the final element of ${\tt t}$ corresponds to position $\epsilon$ in ${\cal T}_{\it bin}$. Then we say that ${\tt t}$ {\em marks\/} $\epsilon$.

$(ii)$ Let $\bfu \in \Tfrep$ and assume that there is a non-empty $p \inw \bfu$ with final label $\ell$, such that ${\it trail\/} (p)$ marks position $\epsilon$. Then we say that $\bfu$ {\it covers\/} position~$\epsilon$ {\it with label\/} $\ell$. Moreover, we say that position~$\epsilon$ is {\it occupied\/} by label $\ell$.
\end{Def}

Now we can show that, given a $\lambda$-tree $\bfu \in \Tfre$ and some $\rar_{\it df}$-reduct $\bfu'$ of~$\bfu$ (i.e., a $\bfu' \in \Tfrep$ with $\bfu \rrar_{\it df} \bfu'$) that covers position $\epsilon$ with label $\ell$, then this label is {\em unique\/} -- whatever the $\bfu'$ is.

\begin{The}
Let $\bfu \in \Tfre$ and let $\epsilon$ be a position in ${\cal T}_{\it bin}$. Then:

-- {\em either} there is no $\rar_{\it df}$-reduct of $\bfu$ that covers $\epsilon$,

-- {\em or} there is such an $\bfu'$; assume that this $\bfu'$ covers $\epsilon$ with label $\ell$; then {\em all} $\rar_{\it df}$-reducts of $\bfu$ covering $\epsilon$, cover $\epsilon$ with the same label $\ell$.
\end{The}

{\it Proof\/} Use Theorem~\ref{TheTraTra}. $\Box$

\medskip




%Let ${\bf u} \in \Tfrep$ and $p \inw {\bf u}$, where $p \not \equiv \emptyset$. Let $\ell$ be the final label in $p$. Then ${\it trail\/}(p)$ describes the {\em position\/} of that label $\ell$ in the tree.





The following theorem states that weak normalization (WN) implies strong normalization (SN) in $\Tfrep$. We firstly give a definition.

\begin{Def} Let ${\bf u} \in \Tfre$ and ${\bf u} \rrar_{\it df} {\bf u'}$.

$(i)$ The {\em border\/} of ${\bf u'}$, or ${\it border}({\bf u'})$, is the set of all positions of outer num-variables of~${\bf u'}$.

$(ii)$ Let also ${\bf u} \rrar_{\it df} {\bf u"}$. A path $p \inw {\bf u"}$ {\em crosses\/} ${\it border}({\bf u'})$ {\it at position $p_1 \, n$} if $p \equiv p_1 \, n \, p_2$ where $n$ occupies a position in
${\it border}({\bf u'})$ and $p_2$ is not-empty.
\end{Def}

\begin{The}\label{TheWnoSno}
Let ${\bf u} \in \Tfre$ be weakly normalizing. Then ${\bf u}$ is also strongly normalizing.
\end{The}

{\it Proof\/} By WN, there exists a ${\bf u'} \in \Tfrep$ and an $\rrar_{\it df}$-reduction path such that ${\bf u} \rrar_{\it df} {\bf u'}$, while there is no ${\bf u''}$ such that ${\bf u'} \rar_{\it df} {\bf u''}$. (So ${\bf u'}$ is a {\em normal form}.)

Assume that there also exists an {\em infinite} reduction path ${\bf u} \rar_{\it df} {\bf u_1} \rar_{\it df} {\bf u_2} \rar_{\it df} \ldots$. Since ${\bf u} \subset {\bf u_1} \subset {\bf u_2} \subset \ldots$, there must be an ${\bf u_i}$ that is not included in ${\bf u'}$. Hence, there is a path $p \inw {\bf u_i}$ that crosses ${\it border\/}({\bf u'})$ at position, say, $p_1 \, n$. So $p \equiv p_1 \, n \, p_2$ for some non-empty $p_2$.

This $n$ is an inner variable of $p$, so by Lemma~\ref{LemInnOut}, $(i)$, num-variable $n$ corresponds to an $\rmA$-$\rmL$-couple positioned on $p_1$. This $\rmA$-$\rmL$-couple generates a $(p_1,p_2)$-reduction in ${\bf u'}$ that extends ${\bf u'}$ outside its borders. So ${\bf u'}$ is not normal. Contradiction. $\Box$

\section{Concluding Remarks}\label{SecConRem} \marginpar{v5. \ref{SecConRem}--1}


In this article, we proposed for the first time the system $\Tfrep$ of
$\lambda$-calculus featuring the simplest support for delaying
$\alpha$-coversion of function arguments in $\beta$-reduction.
To that aim, our system needs an extension of $\beta$-reduction known
as balanced or distant $\beta$-reduction.

The system comes in two versions.
The main version $\Tfrep$ features depth-indexed variable references (Section~\ref{SecDelUpd}),
while the alternative version $\Tcar$ features named variable references (Appendix B).

As we saw, $\Tfrep$ is confluent
and weak normalization implies strong normalization.
Moreover, we can map each term of $\Tfrep$ to a term of
traditional $\lambda K$\marginpar{v5. \ref{SecConRem}--2} so that each $\beta$-reduction step in $\Tfrep$
corresponds to a $\beta$-reduction step in $\lambda K$ and vice versa.

The first author, being experienced in computer-assisted proof verification,
is checking the present theory of $\Tfrep$ with Matita (\citealp{Asp,Gui14})
and plans to report on this activity in future work.
Moreover, we plan to relate $\Tfrep$ to
other systems featuring delayed $\alpha$-conversion
like the ones we mentioned in Section~\ref{SecShoHis}.


\begin{thebibliography}{8}
\bibitem[Abadi {\it et al.}, 1991]{Aba91}
Abadi, M., Cardelli, L., Curien, P.-L.\ and L\'evy, J.-J.,
Explicit Substitutions, {\em J. of Functional Programming\/}, Vol.\ 1, 375--416,
Cambridge University Press, 1991.

\bibitem[Accattoli and Kesner, 2010]{AccKes}
Accattoli, B.\ and Kesner, D., The structural lambda-calculus. In Dawar, A.\ and
Veith, H.\, eds, {\em CSL 2010\/}, Vol.\ 6247 of LNCS, 381-–395, Springer, 2010.

\bibitem[Accattoli and Kesner, 2012]{AccKes12}
Accattoli, B.\ and Kesner, D., The permutative lambda calculus, {\em 18th International Conference on Logic for Programming, Artificial Intelligence, and Reasoning - LPAR-18\/}, Merida,
Venezuela, 2012

\bibitem[Asperti {\it et al.}, 2011]{Asp}
Asperti, A., Ricciotti, W., Sacerdoti Coen, C.\ and Tassi, E.,
The Matita Interactive Theorem Prover. In Bj{\o}rner, N.\ and Sofronie-Stokkermans, V., eds,
{\em Proceedings of the 23rd International Conference on Automated Deduction (CADE-2011)}, Vol.\ 6803 of LNCS, 64--69, Springer, 2011.

\bibitem[Barenbaum and Bonelli, 2017]{BarBon}
Barenbaum, P.\ and Bonelli, E., Optimality and the Linear Substitution Calculus. In: Miller, D., ed., {\em 2nd International Conference on Formal Structures for Computation and Deduction (FSCD 2017\/)}, 9:1–-9:16,
Leibniz International Proceedings in Informatics, 2017.

\bibitem[Barendregt, 1992]{Bar92}
Barendregt, H.P., Lambda calculi with types. In Abramsky, S., Gabbay, D.M.\ and Maibaum, T., eds, {\em Handbook of Logic in Computer Science\/}, Vol.\ 2, 117--309, Oxford, 1992.

\bibitem[de Bruijn, 1972]{deB72}
de Bruijn, N.G., Lambda calculus notation with nameless dummies, a tool for automatic formula manipulation, with application to the Church-Rosser theorem, {\em Indagationes Math.\/} 34 (1972), 381--392. Also in \cite{NedGeuDeV}.

\bibitem[de Bruijn, 1977]{deB77}
de Bruijn, N.G., {\em A namefree lambda calculus with formulas involving symbols that represent reference transforming mappings\/}, Eindhoven University of Technology, Dept.\ of Math., Memorandum 1977-10, 1977. (See also The Automath Archive AUT 050,
www.win.tue.nl$>$Automath.)\marginpar{v5. Ref--1}

\bibitem[de Bruijn, 1978a]{deB78a}
de Bruijn, N.G.\ (1978a), {\em A namefree lambda calculus with facilities for internal definitions of expressions and segments\/}, Eindhoven University of Technology, EUT-report 78-WSK-03, 1978. (See also The Automath Archive AUT 059,
www.win.tue.nl$>$Automath.)\marginpar{v5. Ref--2}

\bibitem[de Bruijn, 1978b]{deB78b}
de Bruijn, N.G.\ (1978b), Lambda calculus notation with namefree formulas involving symbols that represent reference transforming mappings, {\em Indagationes Math.\/} 81, 348--356, 1978. (See also The Automath Archive AUT 055, www.win.tue.nl$>$Automath.)

\bibitem[Curien, Hardin and L\'evy, 1996]{CHL96}
Curien, P.-L., Hardin, Th.\ and L\'evy, J.-J., Confluence Properties of Weak and Strong Calculi of Explicit Substitutions, {\em Journal of the ACM\/}, 43(2), 362--397, New York, 1996.

\bibitem[Guidi, 2009]{Gui09}
Guidi, F., {\em Landau's ``Grundlagen der Analysis'' from Automath to lambda-delta\/}, University of Bologna, Technical Report UBLCS 2009-16, 2009.

\bibitem[Guidi, 2014]{Gui14}
Guidi, F., {\em Formal specification for the interactive prover Matita 0.99.2}, 2014. ({\tt http://lambdadelta.info/})

\bibitem[Kamareddine and Nederpelt, 1993]{KN93}
Kamareddine, F.D.\ and Nederpelt, R.P.,
On stepwise explicit substitution,
{\em Int. J. of Foundations of Computer Science\/}, 4(3), 197--240, World Scientific Publishing Co, Singapore, 1993.

\bibitem[Kluge, 2005]{Klu05}
Kluge, W., {Abstract Computing Machines --- A Lambda Calculus Perspective}, {\em Texts in Theoretical Computer Science\/}, EATCS Series, Springer-Verlag, 2005.

\bibitem[Nederpelt, 1973]{Ned73}
Nederpelt, R.P., {\em Strong normalisation in a typed lambda-calculus with lambda-structured types}. Ph.D.\ thesis, Eindhoven University of Technology, 1973. Also in \cite{NedGeuDeV}.

\bibitem[Nederpelt, 1979]{Ned79}
Nederpelt, R.P., {\em A system of lambda-calculus possessing facilities for typing and abbreviating, Part I: Informal introduction\/},
Memorandum 1979-02, Department of Mathematics, Eindhoven University of Technology,
The Automath Archive AUT 068,
www.win.tue.nl$>$Automath, 1979.
 	
\bibitem[Nederpelt, 1980]{Ned80}
Nederpelt, R.P., {\em A system of lambda-calculus possessing facilities for typing and abbreviating, Part II: Formal description\/},
Memorandum 1980-11, Department of Mathematics, Eindhoven University of Technology,
The Automath Archive AUT 075,
www.win.tue.nl$>$Automath, 1980.

\bibitem[Nederpelt {\it et al.}, 1994]{NedGeuDeV}
Nederpelt, R.P., Geuvers, J.H.\ and de Vrijer, R.C., eds:
{\em Selected Papers on Automath\/},
North-Holland, Elsevier, 1994.

\bibitem[Ventura {\it et al.}, 2015]{Ven}
Ventura, D.\ L., Kamareddine, F.\ and Ayala-Rincon, M., Explicit substitution calculi with de Bruijn indices and intersection type systems, {\em The Logic Journal of the Interest Group of Pure and Applied Logic\/}, Vol.\ 23, issue 2, 295--340, Oxford, 2015.



\end{thebibliography}


\appendix

\section{An elementary variant of beta-reduction in the case of delayed renaming}\label{SecEleVar}%\marginpar{\ref{SecEleVar}--1}

In this section, we introduce a variant of beta-reduction that `minimizes' the argument of such a reduction: only the `necessary' argument of a reduction counts, not its possible reducts.

In order to specify what we mean by this, we firstly need some definitions and lemmas.

\begin{Lem} A legal ${\bf u} \in \Tfrep$ has precisely one origin.\marginpar{v5. \ref{SecEleVar}--1}

%$(i)$ If ${\bf u'} \in \Tfrep$ such that there is an ${\bf u} \in \Tfre$ with ${\bf u} \rar_{\it df} {\bf u}'$, then ${\bf u}'$ is called {\it legal\/}.

%$(ii)$ Such a ${\bf u}$ is unique and is called the {\it origin\/} of ${\bf u}'$.
\end{Lem}

We now define, for given non-complete path $p$ in a legal $\bfu$, the {\em primal tree\/} belonging to $p$, which is the greatest subtree of ${\it tree}(p)$ with the same root as ${\it tree}(p)$, but having no inner variables.

\begin{Def}
Let $\bfu$ be legal and let $p \inw \bfu$ be a path that is not complete.  Let ${\bf v}$ be the set of all paths $q \, n \inw {\it tree}(p)$ such that $q$ contains no inner variables. Then ${\bf v}$ is called the {\em primal tree\/} of $p$. Notation: ${\bf v} \equiv {\it prim}(p)$.
\end{Def}

\begin{Note} $(i)$ A primal tree is always a $\lambda$-tree.

$(ii)$ All num-variables in a primal tree ${\bf v}$ are {\it outer\/} variables with respect to ${\bf v}$, so ${\bf v} \in \Tfre$. Each of these variables may originate from {\em either outer or inner\/} num-variables in the tree ${\bf v}$ it is coming from.
\end{Note}

For the following  lemma, it is convenient to consider the empty path $\epsilon$ as a path in $\bfu$, although this is in contradiction with the definition of $\lambda$-tree.

\begin{Lem}
Let $\bfu$ be legal and $\epsilon$ be the empty path. Then ${\it prim}(\epsilon)$ is the origin of ${\bfu}$.
\end{Lem}

\medskip

We now define what the {\it primal argument\/} of a redex is, as a refinement of Definition~\ref{DefRedExp}.

\begin{Def}\label{DefPriArg}
Let ${\bf t}$ be a $\lambda$-tree, let $b \in {\bf t}$ be a balanced path and assume that $p \, \rmA \, b \, \rmL_x \in^\wedge {\bf t}$ and there is at least one path $p \, \rmA \, b \, \rmL_x \, q \, x \inwv {\bf t}$.

Then ${\it prim}(p \, S)$ is the {\em primal argument\/} of the redex ${\it tree}(p)$.
\end{Def}

We define an elementary kind of $\beta$-reduction for legal terms, that confines the argument of a $\rar_{\it df}$-redex to the primal argument. This {\em primal reduction\/}, with notation $\rar_{\it pdf}$, refines $\rar_{df}$, as we shall prove (Theorem~\ref{ThePdfDf}).

Note that the only difference between Definition~\ref{DefBetEle} (below), and Definition~\ref{DefBetDel} is, that the {\it primal\/} tree of $p \, S$ is taken instead of the `full' tree of $p \, S$.

\begin{Def}\label{DefBetEle}
Let ${\bf u} \in \Tfrep$, let $b \in {\bf u}$ be a balanced path, assume that $p \, \rmA \, b \, \rmL \in^\wedge {\bf u}$ and that $q \, n$ is a fixed, complete path in ${\it tree}(p \, \rmA \, b \, \rmL)$, where $n$ is bound by $\rmL$.

\smallskip

Then ${\bf u} \rar_{\it pdf} {\bf u} \, [{\it tree}(p \, \rmA \, b \, \rmL) := {\it tree}(p \, \rmA \, b \, \rmL)[q \, n := q \, n \, \, {\it prim}(p \, \rmS)]]$.
\end{Def}

Hence, the argument of a certain $\rmA$-$\rmL$-pair in $\bfu$ determining a redex, does not change under $\rar_{\it pdf}$-reduction, even if the argument itself has been subject to other $\rar_{\it pdf}$-reductions.

The following theorem is the counterpart of Theorem~\ref{TheImpInc}.

\begin{The}\label{TheImpIncPri}
Let ${\bf u}, {\bf u'} \in \Tfrep$. Then ${\bf u} \rar_{\it pdf} {\bf u'}$ implies ${\bf u} \subset {\bf u'}$.
\end{The}

Comparing the following lemma with Lemma~\ref{LemDelRed}, we see that primal reduction does not only preserve the $\rmA$-$\rmL$-pairs of the {\em redexes\/} of legal terms, but also the complete {\em arguments\/} of redexes.

\begin{Lem}\label{LemPriRed}
Let ${\bf u}$ be a legal $\lambda$-tree and ${\bf u} \rrar_{\it pdf} {\bf v}$. Let path $p \inw {\bf u}$ and assume that ${\it tree}(p)$ is a redex in ${\bf u}$, with argument ${\it tree}(p \, \rmS)$. Then:

$(i)$ ${\it tree}(p)$ is also a redex in ${\bf v}$,

$(ii)$ argument ${\it tree}(p \, \rmS)$ of  the redex in ${\bf u}$ is also the argument of the corresponding redex in ${\bf v}$.
\end{Lem}

\begin{The}\label{ThePdfDf}
Let ${\bf u}$ be legal and $\bfu \rrar_{\it df} \bfv$. Then also $\bfu \rrar_{\it pdf} \bfv$.
\end{The}

{\it Proof\/} Let $\bfu \rar_{\it df} \bfv$ be a one-step reduction. There is a path $p \, \rmA \, b \, \rmL \, q \, n \inwv \bfu$ such that $(p, q)$ identifies this reduction step. Hence, part of $\bfv$ is the grafted tree ${\bf g} := p \, \rmA \, b \, \rmL \, q \, n \, {\it tree\/}(p \, \rmS)$.

When we apply on $\bfu$ a (primal) ${\it pdf}$-reduction step identified by the same pair $(p, q)$, we obtain ${\bf g_1} : = p \, \rmA \, b \, \rmL \, q \, n \, {\it prim\/}(p \, \rmS)$ as part of, say, ${\bfv}_1$. If ${\it prim\/}(p \, \rmS) = {\it tree\/}(p \, \rmS)$, we are ready.

So assume  ${\it prim\/}(p \, \rmS) \subset {\it tree\/}(p \, \rmS)$.  Then there is at least one position $\epsilon$ covered by num-variable $k$, such that $k$ is {\em outer\/} variable of ${\it tree\/}(p \, \rmA \, b \, \rmL \, q \, n)$ in ${\bf g_1}$
and at the same time {\em inner\/} variable of ${\it tree\/}(p \, \rmA \, b \, \rmL \, q \, n)$ in ${\bf g}$.
Determine the $\rmA$-block $a$ of this $k$, in either one of the grafted trees and consider the leading label $\rmA$.
%Assume that the leading $\rmA$ in $a$ is preceded by path $p'$.
Then this $\rmA$ together with the matching $\rmL$ generate a one-step ${\it pdf}$-reduction of ${\bf v}_1$ such that ${\bf v_1} \rar_{pdf} {\bf v_2}$. By this action, ${\bf g}_1$ transforms into ${\bf g}_2$.

Continue this process, each time choosing a position covered by a num-variable -- if there is such a position -- that is outer variable of ${\it tree\/}(p \, \rmA \, b \, \rmL \, q \, n)$ in ${\bf v_i}$ and inner variable of ${\it tree\/}(p \, \rmA \, b \, \rmL \, q \, n)$ in ${\bf v}$. Thus we obtain the sequence ${\bf u} \rar_{pdf} {\bf v}_1 \rar_{pdf} {\bf v}_2 \ldots$. This process stops at ${\bf v}_n$, when there is no longer a position with the property mentioned above. Then ${\bf v}_n = {\bf v}$. It follows that ${\bf u} \rar_{pdf} {\bf v}_1 \rar_{pdf} {\bf v}_2 \rar_{pdf} \ldots \rar_{pdf} {\bf v}_n = {\bf v}$.

The theorem is an immediate consequence. $\Box$


%$\{$ To be added: theorems on legality$\}$

%\section{Delayed renaming}
\section{Focused beta-reduction with delayed renaming}\label{SecDelRed}
%\marginpar{v4. \ref{SecDelRed}--1}

%In the previous Section~\ref{SecDelUpd} we considered beta-reduction in the {\em namefree\/} notation and discussed the possibility to delay the updating of the num-variables that is needed because the count to find the binder of a num-variable has been disturbed in some instances by the beta-reduction. We introduced an {\it update procedure\/} in order to enable the recovery of the bond between variable and binder.

%The present section deals with similar observations for the {\em name-carrying\/} notation.
Also in the {\em name-carrying\/} notation, the renaming of variables -- often a cumbersome task accompanying the beta-reduction -- may be postponed. And there is a similar procedure to determine the binding between a variable and its binder. We describe these matters below.

As before, we concentrate on {\em focused\/}, {\em balanced\/} beta-reduction.

\medskip

As a start, we rephrase Definition~\ref{DefBetDel} for this case. We use the same symbol $\rar_{df}$ as before for delayed, focused $\beta$-reduction, but now in $\Tcarp$, being the the set of name-carrying, closed trees {\it with\/} inner variables. (Similarly to Section~\ref{SubSecDel}, we distinguish {\em inner\/} and {\em outer\/} variables.)
%The set of the mentioned trees where only outer variables are permitted, we denote by $\Tcar$.

The relation $\rar_{df}$ in $\Tcarp$ is defined as follows.

\begin{Def}\label{DefBetDelRen}
Let ${\bf t} \in \Tcarp$, let $b \in {\bf t}$ be a balanced path, assume that $p \, \rmA \, b \, \rmL_x \in^\wedge {\bf t}$ and that $q \, x$ is a fixed, complete path in ${\it tree}(p \, \rmA \, b \, \rmL_x)$, so $x$ is bound by the final $\rmL_x$ in $p \, \rmA \, b \, \rmL_x$.

\smallskip

Then ${\bf t} \rar_{df} {\bf t} \, [{\it tree}(p \, \rmA \, b \, \rmL_x) := {\it tree}(p \, \rmA \, b \, \rmL_x)[q \, x := q \, x \, {\it tree}(p \, \rmS)]]$.
\end{Def}

Hence, we leave the variable $x$ in the tree and copy ${\it tree}(p \, \rmS)$ right behind it.

(The $x$ in $q \, x$ is clearly an outer variable, the $x$ in $q \, x \, {\it tree}(p \, \rmS)$ an inner one.)

%\smallskip The following definition reflects Definition~\ref{DefTreFre}.

%\begin{Def}\label{DefTreCar}
%The symbol ${\cal T}^{\it car}$ concerns
%\end{Def}

%\smallskip

Also for name-carrying trees and $\rar_{df}$-reduction, there is a procedure ${\cal P}_{car}$ to locate a binder. This procedure is similar to the one for name-free trees (cf.\ Definition~\ref{DefAlgBin}), although there is one major difference: the first element of the pair representing a state is not a natural number, but either a variable or the symbol $\#$. This $\#$ pops up when the binding $\rmL_x$ for $x$ has been found and the name of variable $x$ does no longer play a role.  The second element of a state is a natural number, as earlier.

\smallskip

The {\bf use of the algorithm} ${\cal P}_{car}$ is as follows. Let ${\bf t} \in {\cal T}^{car+}$ and let $p \, x \inw {\bf t}$. Assume that we desire to apply algorithm ${\cal P}_{car}$. Then transform $p \, x$ into $p \, (x,k) \, x$ for $k$ either $0$ or $1$, and {\it start\/} the algorithm. The $k$ decides whether it delivers an ${\rm L}$-block or an ${\rm A}$-block (see Definition~\ref{DefALBloCal}).

\begin{Def}\label{DefAlgBinCar}
The algorithm ${\cal P}_{car}$ is specified by the following rules:

$1.$ ~ {\it first step:} ${\it stack} = \emptyset$

%$ \left\{ \begin{array}{l}
%{\rm in~search~of~\mbox{L-block}:} ~p \, m \rar p \, (m,0) \, m \\
%{\rm in~search~of~\mbox{A-block}:} ~p \, m \rar p \, (m,1) \, m
%\end{array} \right. $

$2a.$ \, $p ~ {\rm L}_y ~ (x,k) ~ q ~ \rar ~ p ~ (x,k) ~ {\rm L}_y ~ q$, if $x \not \equiv y$

$2b.$ \, $p ~ {\rm L}_x ~ (x,k) ~ q ~ \rar ~ p ~ (\#,k) ~ {\rm L}_x ~ q$

$3.$ ~ $p ~ {\rm A} ~ (x,k) ~ q ~ \rar ~ p ~ (x,k) ~ {\rm A} ~ q$

$4.$ ~ $p ~ {\rm S} ~ (x,k) ~ q ~ \rar ~ p ~ (x,k) ~ {\rm S} ~ q$

$5.$ ~ $p ~ y ~ (x,k) ~ q ~ \rar ~ p ~ (y,1) ~ y ~ q$; {\it push} $(x,k)$

$6.$ ~ $p ~ {\rm L}_y ~ (\#,k) ~ q ~ \rar ~ p ~ (\#,k \makebox{+} 1) ~ {\rm L}_y ~ q$, if $k > 0$

$7.$ ~ $p ~ {\rm A} ~ (\#,k) ~ q ~ \rar ~ p ~ (\#,k \makebox{--} 1) ~ {\rm A} ~ q$, if $k > 0$

$8.$ ~ $p ~ y ~ (\#,k) ~ q ~ \rar ~ p ~ (\#,k) ~ y ~ q$, if $k > 0$

$9a.$ \, $p ~ (\#,0)  ~ q ~ \rar ~ p ~ {\it pop}  ~ q$, if  ${\it stack} \not = \emptyset$

$9b.$ \, $p ~ (\#,0) ~ q ~ \rar {\it stop}$, if ${\it stack} = \emptyset$.

\end{Def}



\begin{Def}\label{DefALBloCal}
Let ${\bf t} \in {\cal T}^{car+}$ and $p \, x \inw {\bf t}$.

$(i)$ If ${\cal P}_{car}$ is applied to $p \, (x,0) \, x$ and stops in $p' \, (\#,0) \, q \, x$, then $q \, x$ is called the ${\rm L}$-{\it block\/} of $x$.

$(ii)$ If ${\cal P}_{car}$ is applied to $p \, (x,1) \, x$ and stops in $p' \, (\#,0) \, q \, x$, then $q \, x$ is called the ${\rm A}$-{\it block\/} of $x$.
\end{Def}

This algorithm for the name-carrying version has a similar behaviour as that for the name-free case (cf.\ Lemma's~\ref{LemVarLes} and \ref{LemAblBou} and Theorem~\ref{TheRedSeq}). There are also theorems similar to the ones in Section~\ref{SecTheDel} that hold in the name-carrying situation.




\end{document}

\begin{Exa}\label{ExaRedWay}

Consider the following $\lambda$-term and two of its reducts:

\smallskip

$(\lambda y : L \lamdot ((\lambda x : K \lamdot M) P)) Q \red (\lambda y : L \lamdot (M [x := P]))Q \red$

$(M[x := P])[y := Q]$.

When maintaining the reducts

\end{Exa}
A grafted tree is a rooted path $p$ with a $\lambda$-tree $t$ connected to its leaf. Notation: $p \, t$

\subsection{Extended $\beta$-reduction}

There is a variant of $\beta$-reduction that is interesting for certain purposes. It is  called {\em extended $\beta$-reduction\/} and we use the symbol $\rar_{B}$ for it. (In the literature, it is known under the names $\beta_1$ (\cite{_}) or $\theta$ (\cite{_}).

In first instance, extended $\beta$-reduction follows the pattern as defined in Definition~\ref{DefBetTre}, but part $(i)$ is omitted and part~$(ii)$ is rephrased as:

\smallskip

\ldots {\em replacing\/} $p \, A \, L \, {\bf u}$ by $p \, A \, L \, ({\bf u}[x := {\bf w}])$.

\smallskip

A consequence is that both the mentioned $A$ and the coupled $L$, together with their subtrees $S \, {\bf w}$ and $S \, {\bf v}$, are not erased in the original tree ${\bf t}$.

\smallskip

There appears, however, a complication in applying this form of non-erasing reduction. For example, $\beta$-reduction  transforms a path $p \, \underline{\underline{A}} \, \underline{A} \, \underline{L} \, \underline{\underline{L}} \ldots$ firstly into $p \, \underline{\underline{A}} \, \underline{\underline{L}} \ldots$, and secondly into $p \ldots$. But when the pair $\underline{A} \, \underline{L}$ is not erased in the first step, the second step is not possible, because $\underline{\underline{A}}$ and $\underline{\underline{L}}$ remain separated.

A well-known solution is to {\em extend\/} $\beta$-reduction in such a manner that this drawback is overcome. Then $A$ and $T$ form an $A$-$T$-{\em couple\/} if they are separated by a {\em balanced string\/} of $A$'s and $T$'s. Examples of such strings are:

\smallskip

the empty string, $A \, T$, $A \, A \, T \, T$, $A \, T \, A \, T$, $A \, A \, T \, A \, A \, T \, T \, T$.

\smallskip

If non-erasing $\beta$-reduction is defined for a {\em coupled\/} $A$-$T$-pair, one keeps a strong relation between sequences of $\beta$-reduction and sequences of extended $\beta$-reduction, as we shall see below.

XXX

\subsection{Name-free lambda-trees}



\begin{Exa}\label{NamFreExa}
In Example~\ref{ExaLamTre}~(2) we represent the $\lambda$-tree of the term given in Example~\ref{NamCarExa}, but now in the {\em name-free}\/ version.
\end{Exa}

The {\em numbers\/} at the leaves (called {\em num-labels}\/) are meant to be references to the binding label (the {\em binder\/}), being an L or a P. One finds the binder of a num-label by following backwards (or, in the picture: upwards) the complete path ending in that num-label, counting down along the L's and P's, until 0 is attained. Note that S's do not influence the down-count.

\begin{Exa}\label{NumLabExa}
Consider the complete path $p \equiv {\rm L \, L \, L \, S \, S \, 2}$ in Figure~\ref{ExaLamTre}, (2). Counting backwards, from 2 to 0, we obtain:

${\rm L ~~~~\, L ~~~~\, L~~~~\, S ~~~~\, S ~~~~\,2}$

${\rm ~~~~~~~0 \leftarrow 1 \leftarrow 2 \leftarrow 2 \leftarrow 2}$

So the var-label 2 in the path $p$ is bound to the second L in the path. (This complies with the correspondence between leaf $\beta$ and label ${\rm L}_\beta$ in Figure~\ref{ExaLamTre}, (1).)
\end{Exa}

The list of all complete paths in this name-free tree is the following:

\smallskip

${\rm L \, L \, L \, L \, A \, 2}$,
${\rm L \, L \, L \, L \, S \, 1}$,
${\rm L \, L \, L \, S \, P \, 4}$,
${\rm L \, L \, L \, S \, S \, 2}$,
${\rm L \, L \, S \, 2}$,
${\rm L \, S \, }\ast$,
${\rm S} \, \ast$.


The list in this example is {\em ordered\/} according to the order relation generated by ${\rm A}, \, {\rm L}, \, {\rm P}$ all $ < {\rm S}$.

\begin{Def}\label{DefFulBun} A set of complete paths of a lambda tree ${\bf t}$ is called a {\em bundle\/} in ${\bf t}$. The set of {\em all\/} complete paths of ${\bf t}$ is the {\em complete bundle\/} of ${\bf t}$.
\end{Def}


\vspace{0.4cm}

{\bf Untyped lambda calculus}

\vspace{0.3cm}

 Example of an untyped lambda tree:

\begin{picture}(180,90)(-60,0)
\put(-50,70){$\langle \lambda z\,v \lamdot (\lambda x \lamdot (\lambda y \lamdot y \; x)z)v \rangle ~~=$}
\put(-50,42){\line(1,0){214}}

\put(-20,42){\circle*{4}}
\put(-50,42){\line(0,-1){10}}
\put(-53.8,30){$\times$}
\put(-39,47){${\rm L}_z$}

\put(10,42){\circle*{4}}
\put(-20,42){\line(0,-1){10}}
\put(-23.8,30){$\times$}
\put(-9,47){${\rm L}_v$}


\put(10,42){\circle*{4}}
\put(10,42){\line(0,-1){18}}
\put(25,27){\oval(30,30)[bl]}
\put(24,12){\line(1,0){17}}
\put(21,47){${\rm A}$}
\put(15,23){${\rm R}$}

\put(40,42){\circle*{4}}
\put(40,42){\line(0,-1){10}}
\put(36.2,30){$\times$}
\put(51,47){${\rm L}_x$}

\put(70,42){\circle*{4}}
\put(70,42){\line(0,-1){18}}
\put(85,27){\oval(30,30)[bl]}
\put(84,12){\line(1,0){17}}
\put(81,47){${\rm A}$}
\put(75,23){${\rm R}$}

\put(100,42){\circle*{4}}
\put(100,42){\line(0,-1){10}}
\put(96.2,30){$\times$}
\put(111,47){${\rm L}_y$}

\put(130,42){\circle*{4}}
\put(130,42){\line(0,-1){18}}
\put(145,27){\oval(30,30)[bl]}
\put(144,12){\line(1,0){17}}
\put(141,47){${\rm A}$}
\put(135,23){${\rm R}$}

\put(40,12){\circle*{4}}
\put(40,12){\line(1,0){28}}
\put(51,17){$v$}

\put(100,12){\circle*{4}}
\put(100,12){\line(1,0){28}}
\put(111,17){$z$}

\put(160,12){\circle*{4}}
\put(160,12){\line(1,0){28}}
\put(171,17){$x$}

\put(160,42){\circle*{4}}
\put(160,42){\line(1,0){28}}
\put(171,47){$y$}
\end{picture}

\vspace{0.2cm}


Bundle of branches:

\smallskip

$\{{\rm L}_z \, {\rm L}_v \, {\rm A} \, {\rm L}_x \, {\rm A} \, {\rm L}_y \, {\rm A} \, y,~~
{\rm L}_z \, {\rm L}_v \, {\rm A} \, {\rm L}_x \, {\rm A} \, {\rm L}_y \, {\rm R} \, x,~~
{\rm L}_z \, {\rm L}_v \, {\rm A} \, {\rm L}_x \, {\rm R}  \, z,~~
{\rm L}_z \, {\rm L}_v \, {\rm R} \, v\}$.

%\newpage

{\bf Typed lambda calculus:}

n-fork is also called a {\em nat-label\/}

\vspace{0.3cm}

 Example of a typed lambda tree:

%$\langle \lambda z : (\Pi x : S \lamdot \Pi y : S \lamdot Q \; x \; y) \lamdot \lambda u : S \lamdot z \; u \; u \rangle ~~=$

\bigskip

\begin{picture}(300,110)(-60,0)
%\put(10,100){$\langle \lambda z : (\Pi x : S \lamdot \Pi y : S \lamdot Q \; x \; y) \lamdot \lambda u : S \lamdot z \; u \; u \rangle ~~=$}
\put(-50,100){$\langle \lambda z : (\Pi x : S \lamdot \Pi y : S \lamdot Q \; x \; y) \lamdot \lambda u : S \lamdot z \; u \; u \rangle ~~=$}

\put(10,72){\line(1,0){247}}
\put(40,42){\line(1,0){111}}

\multiput(10,72)(-13.25,0){5}{\line(-1,0){7}}
\put(-50,72){\circle*{4}}
\put(-20,72){\circle*{4}}
\put(-43,76){${\rm L}_S$}
\put(-13,76){${\rm L}_Q$}

\put(10,72){\circle*{4}}
\put(10,72){\line(0,-1){18}}
\put(25,57){\oval(30,30)[bl]}
\put(24,42){\line(1,0){14}}
\put(13,55){${\rm R}$}
\put(17,76){${\rm L_z}$}

\put(40,42){\circle*{4}}
\put(40,42){\line(0,-1){18}}
\put(55,27){\oval(30,30)[bl]}
\put(54,12){\line(1,0){5}}
\put(43,25){${\rm R}$}
\put(47,46){${\rm P}_x$}

\put(70,42){\circle*{4}}
\put(70,42){\line(0,-1){18}}
\put(85,27){\oval(30,30)[bl]}
\put(84,12){\line(1,0){5}}
\put(73,25){${\rm R}$}
\put(77,46){${\rm P}_y$}

\put(100,42){\circle*{4}}
\put(100,42){\line(0,-1){18}}
\put(115,27){\oval(30,30)[bl]}
\put(114,12){\line(1,0){5}}
\put(103,25){${\rm R}$}
\put(107,46){${\rm A}$}

\put(130,42){\circle*{4}}
\put(130,42){\line(0,-1){18}}
\put(145,27){\oval(30,30)[bl]}
\put(144,12){\line(1,0){5}}
\put(133,25){${\rm R}$}
\put(137,46){${\rm A}$}

\put(175,72){\circle*{4}}
\put(175,72){\line(0,-1){18}}
\put(190,57){\oval(30,30)[bl]}
\put(189,42){\line(1,0){5}}
\put(178,55){${\rm R}$}
\put(182,76){${\rm L}_u$}

\put(205,72){\circle*{4}}
\put(205,72){\line(0,-1){18}}
\put(220,57){\oval(30,30)[bl]}
\put(219,42){\line(1,0){5}}
\put(208,55){${\rm R}$}
\put(212,76){${\rm A}$}

\put(235,72){\circle*{4}}
\put(235,72){\line(0,-1){18}}
\put(250,57){\oval(30,30)[bl]}
\put(249,42){\line(1,0){5}}
\put(238,55){${\rm R}$}
\put(242,76){${\rm A}$}

\put(262,72){\circle{10}}
\put(259,70){$z$}

\put(199,42){\circle{10}}
\put(195,39){$S$}

\put(229,42){\circle{10}}
\put(226,40){$u$}

\put(259,42){\circle{10}}
\put(256,40){$u$}

\put(156,42){\circle{10}}
\put(152,39){$Q$}

\put(64,12){\circle{10}}
\put(60.3,9){$S$}

\put(94,12){\circle{10}}
\put(90.3,9){$S$}

\put(124,12){\circle{10}}
\put(121,10){$y$}

\put(154,12){\circle{10}}
\put(151,10){$x$}


\end{picture}


\vspace{0.5cm}

Free variables: $S$, $Q$. Bound variables: $x$, $y$, $z$, $u$.

The tree of a lambda term $t$, constructed as in the above examples, is called a {\em lambda tree\/}, or {\em the lambda tree of $t$\/}.

\medskip

{\em Note:}

In the tree we can represent the free variables as follows.

(1) De Bruijn's solution (see above): add dashed line in front with two vertices and two labels: ${\rm L}_S$ and ${\rm L}_Q$, respectively.

(2) In the spirit of type theory: add in front $\lambda S : \ast \lamdot \lambda Q : S \rar S \rar \ast$, where $S \rar S \rar \ast ~\equiv~ \Pi v : S \lamdot \Pi w : S \lamdot \ast$. Then $\ast$ can be considered to be the only free `variable' ($\ast$ is often called a `constant' of Type Theory.

\vspace{0.5cm}

{\bf Definitions and examples:}

\medskip

A {\em (lambda) branch\/} consists of the list of labels along a path from the root node to an end node. Since there is a one-to-one correspondence between branches and their end nodes, one may use the branch to identify this end node.


\vspace{0.5cm}

Example (see above): the branches (and the nodes they determine) are:

\smallskip

$t_1 =$ ${\rm L}_x$ ${\rm L}_u$ ${\rm A}$ ${\rm A}$ $z$

$t_2 =$ ${\rm L}_x$ ${\rm L}_u$ ${\rm A}$ ${\rm R}$ $u$

$t_3 =$ ${\rm L}_x$ ${\rm L}_u$ ${\rm R}$ $u$

$t_4 =$ ${\rm L}_x$ ${\rm R}$ $S$

$t_5 =$ ${\rm R}$ ${\rm P}_x$ ${\rm P}_y$ ${\rm A}$ ${\rm A}$ $Q$

$t_6 =$ ${\rm R}$ ${\rm P}_x$ ${\rm P}_y$ ${\rm A}$ ${\rm R}$ $x$

$t_7 =$ ${\rm R}$ ${\rm P}_x$ ${\rm P}_y$ ${\rm R}$ $y$

$t_8 =$ ${\rm R}$ ${\rm P}_x$ ${\rm R}$ $S$

$t_9 =$ ${\rm R}$ ${\rm R}$ $S$

\vspace{0.5cm}

A {\em bundle\/} $B$ is a complete set of branches representing a binary rooted tree as above. E.g., $B = \{ t_1, t_2, \ldots, t_9\}$.

%\newpage

%{\bf Alternative trees}

%\medskip

%\begin{picture}(350,75)(30,0)
%\put(10,50){$\langle\lambda x : M \lamdot N \rangle ~~=$}
%\put(120,52){\circle*{4}}
%\put(120,18){\circle*{4}}
%\put(90,52){\line(1,0){30}}
%\put(90,52){\line(0,-1){22}}
%\put(105,32){\oval(30,30)[bl]}
%\put(104,17){\line(1,0){14}}
%\put(118,58){${\rm L}_x$}
%\put(118,24){${\rm R}$}
%\put(130,49){$\langle N \rangle$}
%\put(130,14){$\langle M \rangle$}

%\put(12, 0){$\lambda$-abstraction}

%\put(170,50){$\langle M  N \rangle ~~=$}
%\put(255,52){\circle*{4}}
%\put(255,18){\circle*{4}}
%\put(225,52){\line(1,0){30}}
%\put(225,52){\line(0,-1){22}}
%\put(240,32){\oval(30,30)[bl]}
%\put(239,17){\line(1,0){14}}
%\put(253,58){${\rm A}$}
%\put(253,24){${\rm R}$}
%%\put(231,34){${\ell}$}
%\put(265,49){$\langle M \rangle$}
%\put(265,14){$\langle N \rangle$}

%\put(172, 0){application}

%\put(305,50){$\langle x \rangle ~=$}
%\put(375,52){\circle*{4}}
%\put(372,58){$x$}
%\put(342,52){\line(1,0){32}}
%\put(307,35){variable}

%\put(320,18){\circle*{4}}
%\put(320,18){\circle{6}}
%\put(310,0){root}
%\end{picture}


%\vspace{0.4cm}

{\bf Definitions}

\medskip


An initial part of a branch $p$, starting from the root, is a {\em initial segment of~$p$}.

A final part of a branch, including its end node, is a {\em final segment of $p$}.

A {\em segment\/} $s$ of branch $p$ is a string forming a connected part of $p$, so we have $p = q' \; s \, q''$ for some initial and final segments $q'$ and $q''$.

A segment of branch $p$ consisting of one label only (so being $x$, $R$, $A$, $L_x$ or $P_x$ for some $x$) is called an {\em element\/} of $p$. If $e$ is the final element of lambda string $p$, then we say that $p$ {\em ends in\/} $e$.

(So each branch is an initial segment, a final segment and a segment of itself, and each element is a segment.)


\medskip


Each node (not only a leaf) in a bundle can be named with the initial segment up to this node of any branch passing `through' this node. For example: in Fig.\ref{-} the node between the labels $P_x$ and $P_y$ has the name $R$ $P_x$.


\medskip

{\bf Lemma}

\smallskip

Let $B$ be a bundle of branches. Each branch is composed of elements $R$, $A$, $L_x$ or $x$ for some $x$. In higher order calculi, also $P_x$ is allowed as an element of a branch.

\medskip

{\bf Binary tree}

\smallskip

The binary rooted tree-character can be characterised by the following properties, which form a definition of the notion `bundle'.

Let $q \, e$ be an initial segment of a branch in $B$, where $e$ is an element. Then $e$ may be $x$, and consequently $q \, e$ is a (complete) branch. In that case there is no other branch $q \, y$ with $y \not = x$ in $B$.

If $e$ is not $x$, there are two distinct possibilities:

(1) $e$ is $R$,

(2) $e$ is either $A$, or $L_x$, or $P_x$.
(Note that `either $\ldots$ or $\ldots$' means `exclusive or' here.)
Then for {\em all} branches with initial segment $q$, the following element, if not $R$, is identical to $e$.

Moreover, both (1) and (2) must actually occur if $e$ is not $x$.

\smallskip

{\bf Note} Is there a shorter and more intuitive description than this one?


\vspace{0.5cm}

{\bf Definitions:}

\smallskip

(1) Let $B$ be a bundle and $p \, L_x \, q \, x$ be a branch in $B$. Then this $x$ {\em is bound\/} by the mentioned label $L_x$. Note that there can be zero or more branches ending in an $x$ bound by $L_x$. The label $L_x$ is the {\em binder\/} of $x$, also called the {\em L-binder\/}.

On the other hand, each element $L_x$ can bind several (zero or more) $x$'s. The set of all branches $r$ ending in $x$ is the {\em reach\/} of the binder $L_x$. Note that we assume all binding variables to be distinct, so $L_x$ is unique in $B$, and in particular in every branch in which it occurs; moreover, all the mentioned branches, containing $L_x$ and ending in $x$, begin with the same initial segment, up to $L_x$; the final segments of these branches following $L_x$, however, are different. The lengths of these final segments may differ, as well.

\smallskip

Similarly, in the branch $p \, P_x \, q \, x$ of a bundle $B$, element $x$ is bound by $P_x$. This $P_x$ is called the {\em $P$-binder\/} of $x$.

\medskip

(2) A bundle $B$ in which all variables are bound is called a {\em bound bundle\/}.

\medskip

(3) Let $q$ be an initial segment of some branch in $B$. Consider all branches $q \, r$ in $B$, i.e., the branches starting with $q$. Then all final segments $r$ of these branches form a bundle $B'$ themselves, called the {\em sub-bundle\/} belonging to $q$. We use the notation $q \, B'$ to denote the set of all $q \, r$ (quantification ranging over $r$).

%When $q = \varepsilon$, so when bundle $B$ starts in the same root as sub-bundle $B'$, we say that $B'$ is a {\em root-sharing\/} sub-bundle of $B$. Notation: $B' \subseteq B$.

\vspace{0.5cm}

Now assume that a particular branch of $B$ has the form $p \, A \, L_x \, q \, x$, so an $A$ immediately precedes $L_x$ in this branch, and the branch has end note $x$ bound by $L_x$. Then $\beta$-reduction is possible. Their are two main variants:

\smallskip

(1) {\em One-step $\beta$-reduction\/}, denoted $\beta_1$:

Let $B'$ be the sub-bundle belonging to $p \, R$. Then $B  \rar_{\beta_1}  B_1$, where $B_1$ is the bundle obtained from $B$ by replacing the branch $p \, A \, L_x \, q \, x$ by the set of branches $p \, A \, L_x \, q \, B'$ (i.e., $x$ has been replaced by $B'$).

\smallskip

(2) {\em (Multiple) $\beta$-reduction\/}, denoted $\beta$:

By applying one-step $\beta$-reduction to {\em all} branches with head segment $p \, A \, L_x$ and ending in $x$, we obtain $B''$, for which $B \rar_{\beta} B''$.


\vspace{0.5cm}
{\bf Example:}

\smallskip

\begin{picture}(300,110)(-60,0)
\put(10,100){$\langle \; (\lambda x \lamdot x \; x) \; (\lambda y \lamdot y \; y) \; \rangle ~~=$}
\put(10,72){\line(1,0){180}}
\put(40,42){\line(1,0){75}}

%\multiput(10,72)(-13.25,0){5}{\line(-1,0){7}}
%\put(-20,72){\circle*{4}}
%\put(-50,72){\circle*{4}}
%\put(-13,76){${\rm L}_Q$}
%\put(-43,76){${\rm L}_S$}

\put(10,72){\circle*{4}}
\put(10,72){\line(0,-1){18}}
\put(25,57){\oval(30,30)[bl]}
\put(24,42){\line(1,0){14}}
\put(13,55){${\rm R}$}
\put(17,76){${\rm A}$}

\put(40,42){\circle*{4}}
\put(40,42){\line(0,-1){10}}
%\put(55,27){\oval(30,30)[bl]}
%\put(54,12){\line(1,0){5}}
%\put(43,25){${\rm R}$}
\put(47,46){${\rm L}_y$}
%\put(-20,42){\line(0,-1){10}}
\put(36.2,30){$\times$}


\put(70,42){\circle*{4}}
\put(70,42){\line(0,-1){18}}
\put(85,27){\oval(30,30)[bl]}
%\put(84,12){\line(1,0){5}}
\put(73,25){${\rm R}$}
\put(77,46){${\rm A}$}

\put(115,72){\circle*{4}}
\put(115,72){\line(0,-1){10}}
\put(122,76){${\rm L}_x$}
\put(111.2,60){$\times$}


\put(145,72){\circle*{4}}
\put(145,72){\line(0,-1){18}}
\put(160,57){\oval(30,30)[bl]}

\put(148,55){${\rm R}$}
\put(152,76){${\rm A}$}

%\put(172,72){\circle{10}}
%\put(169.3,69.8){$x$}
\put(175,72){\circle*{4}}
\put(179,77){$x$}

%\put(169,42){\circle{10}}
%\put(166.4,39.8){$x$}
\put(160,42){\circle*{4}}
\put(164,47){$x$}
\put(160,42){\line(1,0){15}}

%\put(97,42){\circle{10}}
%\put(94.5,40.5){$y$}
\put(100,42){\circle*{4}}
\put(104,47){$y$}

%\put(94,12){\circle{10}}
%\put(91.4,10.5){$y$}
\put(85,12){\circle*{4}}
\put(89,17){$y$}
\put(85,12){\line(1,0){15}}

\put(200,40){$\rar_{\beta_1}$}


\end{picture}

%\vspace{0.5cm}

%\vspace{0.4cm}

\begin{picture}(300,110)(-60,0)
\put(10,100){$\langle \; (\lambda x \lamdot x \; (\lambda v \lamdot v \; v)) \; (\lambda y \lamdot y \; y) \; \rangle ~~=$}
%\put(10,72){\line(1,0){157}}
%\put(40,42){\line(1,0){52}}
\put(10,72){\line(1,0){180}}
\put(40,42){\line(1,0){75}}

%\multiput(10,72)(-13.25,0){5}{\line(-1,0){7}}
%\put(-20,72){\circle*{4}}
%\put(-50,72){\circle*{4}}
%\put(-13,76){${\rm L}_Q$}
%\put(-43,76){${\rm L}_S$}

\put(10,72){\circle*{4}}
\put(10,72){\line(0,-1){18}}
\put(25,57){\oval(30,30)[bl]}
\put(24,42){\line(1,0){14}}
\put(13,55){${\rm R}$}
\put(17,76){${\rm A}$}

\put(40,42){\circle*{4}}
\put(40,42){\line(0,-1){10}}
%\put(55,27){\oval(30,30)[bl]}
%\put(54,12){\line(1,0){5}}
%\put(43,25){${\rm R}$}
\put(47,46){${\rm L}_y$}
%\put(-20,42){\line(0,-1){10}}
\put(36.2,30){$\times$}


\put(70,42){\circle*{4}}
\put(70,42){\line(0,-1){18}}
\put(85,27){\oval(30,30)[bl]}
\put(84,12){\line(1,0){5}}
\put(73,25){${\rm R}$}
\put(77,46){${\rm A}$}

\put(115,72){\circle*{4}}
\put(115,72){\line(0,-1){10}}
\put(122,76){${\rm L}_x$}
\put(111.2,60){$\times$}


\put(145,72){\circle*{4}}
\put(145,72){\line(0,-1){18}}
\put(160,57){\oval(30,30)[bl]}
\put(159,42){\line(1,0){91}}
\put(148,55){${\rm R}$}
\put(152,76){${\rm A}$}

%\put(172,72){\circle{10}}
%\put(169.3,69.8){$x$}
%\put(172,72){\circle{10}}
%\put(169.3,69.8){$x$}
\put(175,72){\circle*{4}}
\put(179,77){$x$}

%\put(169,42){\circle{10}}
%\put(166.4,39.8){$x$}

%\put(97,42){\circle{10}}
%\put(94.5,40.5){$y$}
%\put(97,42){\circle{10}}
%\put(94.5,40.5){$y$}
\put(100,42){\circle*{4}}
\put(104,47){$y$}

%\put(94,12){\circle{10}}
%\put(91.4,10.5){$y$}
%\put(94,12){\circle{10}}
%\put(91.4,10.5){$y$}
\put(85,12){\circle*{4}}
\put(89,17){$y$}
\put(85,12){\line(1,0){15}}

%\put(200,40){$\rar_{\beta_1}$}

\put(175,42){\circle*{4}}

\put(175,42){\circle*{4}}
\put(175,42){\line(0,-1){10}}
%\put(55,27){\oval(30,30)[bl]}
%\put(54,12){\line(1,0){5}}
%\put(43,25){${\rm R}$}
\put(182,46){${\rm L}_v$}
%\put(-20,42){\line(0,-1){10}}
\put(171.2,30){$\times$}


\put(205,42){\circle*{4}}
\put(205,42){\line(0,-1){18}}
\put(220,27){\oval(30,30)[bl]}
\put(219,12){\line(1,0){15}}
\put(208,25){${\rm R}$}
\put(212,46){${\rm A}$}

%\put(232,42){\circle{10}}
%\put(229.5,40.03){$v$}
\put(235,42){\circle*{4}}
\put(239,47){$v$}

%\put(229,12){\circle{10}}
%\put(226.4,10.03){$v$}
\put(219,12){\circle*{4}}
\put(223,17){$v$}

\end{picture}



\newpage


{\bf Definitions:}

\smallskip

(1) Lexicographic ordering $<$ of branches and segments is based on the following order: $R < A,\, L_x\, {\rm or} \, P_x$.

Hence the following. Let $p$ and $q$ be segments. Then $p < q$ if $p$ is an initial segment of $q$, or there are initial segments $p' \, v$ of $p$ and $q' \, w$ of $q$, where $v$ and $w$ are labels, for which $p'$ is identical to $q'$ and $v < w$.

\smallskip

(2) Let $B$ be a bundle. Then there is precisely one branch $p$ in $B$ which does not contain the label $R$. This branch is called the {\em main branch\/} of $p$. Note: the main branch of a bundel $B$ is lexicographically the minimum of all branches of $B$.

\medskip

(3) Let $p$ be a branch (not necessarily the main branch) in a bundle of branches $B$. Then all information relevant for relations connected to $p$, is present in the subset $C_p$ consisting of $p$ and all branches that are `below' $p$ in the lexicographical ordering. We call $C_p$ the {\em information subset\/} of $B$ relative to $p$. (ZIE OOK LATER BIJ `TO BE INVESTIGATED', subterm)

We elucidate the word `information' by the following remarks:

\smallskip

($i$) An `argument' for a function playing a role in $\beta$-reduction, can be found below the branch in which it will be substituted (see the first tree in the example above: the tree of the argument $\lambda y \lamdot y \, y$ can be found {\em below\/} the branch ${\rm A} \, {\rm L}_x \, {\rm R} \, x$ in which it will become substituted).

($ii$) In order to prove `correctness' for a term in {\em typed\/} lambda calculus, we need to inspect types of variables. In the tree, a type of $x$ occurs below the branch ending in $x$.

\medskip

{\bf Example:}

\smallskip

See the second tree in the previous example.

Consider the branch $p \, = \, {\rm A} \, {\rm L}_x \, {\rm R} \, {\rm L}_v \,{\rm A} \, v$. The collection $C_p$ consists of:

\smallskip

%${\rm A} \, {\rm L}_x \, {\rm R} \, {\rm L}_v \,{\rm A} \, v$,

${\rm A} \, {\rm L}_x \, {\rm R} \, {\rm L}_v \,{\rm R} \, v$,

${\rm A} \, {\rm L}_x \, {\rm R} \, \times$,

${\rm A} \, {\rm R} \, {\rm L}_y \, {\rm A} \, y$,

${\rm A} \, {\rm R} \, {\rm L}_y \, {\rm R} \, y$,

${\rm A} \, {\rm R} \, \times$.

\medskip

{\bf Extended beta reduction}

\smallskip

There is a special kind of segments that only consist of ${\rm A}$'s and ${\rm L}$'s, the {\em ${\rm A} \, {\rm L}$-segments\/}. We focus on those ${\rm A} \, {\rm L}$-segments in a branch which have the special property that, counting from left to right, the number of ${\rm A}$'s is always {\em greater\/} than the number of ${\rm L}$'s --- except for the segment as a whole, in which the number of ${\rm A}$'s is {\em equal\/} to the number of ${\rm L}$'s.

\smallskip

Examples: ${\rm A} \, {\rm L}$, ${\rm A} \, {\rm A} \, {\rm L} \, {\rm L}$, ${\rm A} \, {\rm A} \, {\rm L} \, {\rm A} \, {\rm L} \, {\rm L}$, ${\rm A} \, {\rm A} \, {\rm A} \, {\rm L} \, {\rm L} \, {\rm A} \, {\rm L} \, {\rm L}$, ${\rm A} \, {\rm A} \, {\rm L} \, {\rm A} \, {\rm L} \, {\rm A} \, {\rm L} \, {\rm L}$.

We call such segments {\em embraced\/}. Compare these segments with a string of nested parentheses having an embracing pair: $( \, )$,~  $( \, ( \, ) \, )$,~ $( \, ( \, ) \, ( \, ) \, )$,~ \ldots.

\smallskip

When dropping the requirement of embracing, such segments, are called {\em well-balanced\/}. So all embraced segments are well-balanced. Examples: the previous ones, plus $\varepsilon$, ${\rm A} \, {\rm L}$, ${\rm A} \, {\rm L} \, {\rm {\rm A}} \, {\rm L}$, ${\rm A} \, {\rm A} \, {\rm L} \, {\rm L} \, {\rm A} \, {\rm L}$, ${\rm A} \, {\rm L} \, {\rm A} \, {\rm L} \, {\rm A} \, {\rm L}$.

\medskip

Note that each ${\rm A}$ in an ${\rm A} \, {\rm L}$-segment is coupled to exactly one ${\rm L}$, and vice versa, just as with opening and closing parentheses.

If an ${\rm A}$ is immediately followed by an ${\rm L}$ in such a segment, then we speak (after de Bruijn) of an ${\rm A \, L}$-{\em pair\/} and obviously (one- or more-step) $\beta$-reduction is possible. In general, such `matching' ${\rm A}$ and ${\rm L}$ are called an ${\rm A \, L}$-{\em couple\/}. Also ${\rm A \, L}$-couples can give rise to $\beta$-reduction, as has firstly been worked out in Ne\-derpelt~73.

\medskip

{\bf Definition}

\smallskip

Let $p \, {\rm A} \, q \, {\rm L}_x \, r \, x$ be a branch in a bundle $B$ such that the A and ${\rm L}_x$ form an ${\rm A} \, {\rm L}$-couple. (Then $q$ is well-balanced and ${\rm L}_x$ is the {\em $L$-binder\/} of $x$.) We also say that the mentioned ${\rm A}$ is the {\em $A$-binder\/} of $x$.

\smallskip

{\bf Note} In a bound bundle $B$, every element $x$ in a branch has {\em exactly one\/} $L$-binder or $P$-binder and {\em at most one\/} $A$-binder.

\medskip

{\bf Example}

\medskip

Consider the typed lambda term

%\smallskip

(1) $\lambda v : \alpha \lamdot ((\underline {\rule[-0.5ex]{0ex}{3ex}\lambda y : \alpha} \lamdot \underline{\underline{\rule[-0.5ex]{0ex}{3ex}\lambda x : \alpha \rar \alpha}} \lamdot x \, (x \, y)) \, \underline{\rule[-0.5ex]{0ex}{3ex}v} \, (\underline{\underline{\rule[-0.5ex]{0ex}{3ex} \lambda u : \alpha \lamdot u}}))$

\smallskip

Clearly, the underlined parts $\lambda y : \alpha$ and $v$ fit together, they form an ${\rm A} \, {\rm L}$-pair. When applying local $\beta$-reduction at this ${\rm A} \, {\rm L}$-pair, one obtains (substituting $v$ for $y$):

\smallskip

(2) $\lambda v : \alpha \lamdot ((\lambda y : \alpha \lamdot \lambda x : \alpha \rar \alpha \lamdot x \, (x \, v)) \, v \, ( \lambda u : \alpha \lamdot u))$

\smallskip

But also the doubly underlined parts belong together, they are an ${\rm A} \, {\rm L}$-{\em couple\/}. Although the latter pair does not give rise to a $\beta$-reduction in the conventional sense (that is to say: not immediately, but only after `removal' of the singly underlined pair), they are ready for {\em extended $\beta$-reduction\/}, denoted $\rar_{\beta_1}$. There are two $x$'s in the body $x(x \, y)$, so there are two possible $\rar_{\beta_1}$-reductions. One of them leads from (1) directly to:

\smallskip


(3) $\lambda v : \alpha \lamdot ((\lambda y : \alpha \lamdot \lambda x : \alpha \rar \alpha \lamdot x \, ((\lambda u : \alpha \lamdot u)y)) \, v \, (\lambda u : \alpha \lamdot u))$

\smallskip

A further $\beta_1$-reduction gives:

(4) $\lambda v : \alpha \lamdot ((\lambda y : \alpha \lamdot \lambda x : \alpha \rar \alpha \lamdot (\lambda u : \alpha \lamdot u) \, ((\lambda u : \alpha \lamdot u) y))v( \lambda u : \alpha \lamdot u))$

\smallskip

So (1) $\rar_{\beta_1}$ (2) by means of the singly underlined reduction, and
(1) $\rar_{\beta_1}$ (3) $\rar_{\beta_1}$ (4) by the doubly underlined reduction.

%The ${\beta_2}$-normal form of (4) is (5):

%\smallskip

%(5) $\lambda v : \alpha v $

\medskip

{\bf The erasing reduction}

\smallskip

This example shows that extended $\beta$-reduction does {\em not\/} erase either the ${\rm A} \, {\rm L}$-pair or the ${\rm A} \, {\rm L}$-couple that gave rise to the reduction. We need a so-called $\beta_2$-reduction to get rid of these pairs or couples. This is only allowed when they are `void', i.e., if the `binding' $x$ of the ${\rm L}_x$ does not bind any $x$.

For example, $\beta_2$-reduction can be applied to (4), by omitting the ${\rm A} \, {\rm L}$-pair ${\rm L} \equiv \lambda x : \alpha \rar \alpha$ and ${\rm A} \equiv \lambda u : \alpha \lamdot u$ (we mean the final occurrence of this subterm). So we have that (4) $\rar_{\beta_2}$ (5):

\smallskip

(5) $\lambda v : \alpha \lamdot ((\lambda y : \alpha \lamdot (\lambda u : \alpha \lamdot u)((\lambda u : \alpha \lamdot u) y))v)$

\smallskip

Note that we cannot apply $\beta_2$-reduction to (3), since both $x$ and $y$ are still available for $\beta_1$-reduction in the body $x((\lambda u : \alpha \lamdot u) \, y)$.

Applying $\beta_1$-reduction to (5) results in substituting $v$ for $y$. Consequently, two $\beta_2$-reductions lead to

\smallskip

(6) $\lambda v : \alpha \lamdot v$,

\smallskip

\noindent which is also the ${\beta}$-normal form.


%\newpage
%\vspace{0.4cm}

\section{Namefree lambda calculus}\label{AutFor}


\vspace{0.4cm}

In the so-called {\em name-free\/} lambda calculus, bound variables do not refer to their binding ${\rm L}$ or ${\rm P}$ by {\em names\/}, such as $x$ and $y$, but by {\em numbers\/}. Each leaf label is replaced by a natural number, which represents the number of ${\rm L}$'s and ${\rm P}$'s that separate the leaf label from its binding ${\rm L}$ or ${\rm P}$, measured on the path to the root. The binding ${\rm L}$'s and ${\rm P}$'s then need not to be suffixed with a variable.

\bigskip

{\bf Example 1}

\medskip

{\em Name-carrying lambda tree\/}

\smallskip

$\langle \lambda z : (\Pi x : S \lamdot \Pi y : S \lamdot Q \; x \; y) \lamdot \lambda u : S \lamdot z \; u \; u \rangle$

\medskip

See Figure\ref{--}.


\bigskip

{\em Name-free lambda tree\/}

\smallskip


%\vspace{0.4cm}
\begin{picture}(300,110)(-50,0)
%\put(10,100){$\langle \lambda z : (\Pi x : S \lamdot \Pi y : S \lamdot Q \; x \; y) \lamdot \lambda u : S \lamdot z \; u \; u \rangle ~~=$}
\put(10,100){$\langle \lambda \lamdot \lambda \lamdot \lambda : (\Pi : 2 \lamdot \Pi  : 3 \lamdot 3 ~ 2 ~ 1) \lamdot \lambda : 3 \lamdot 2 ~ 1 ~ 1 \rangle ~~=$}
\put(10,72){\line(1,0){270}}
\put(40,42){\line(1,0){135}}

\multiput(10,72)(-13.25,0){5}{\line(-1,0){7}}
\put(-20,72){\circle*{4}}
\put(-50,72){\circle*{4}}
\put(-13,76){${\rm L}$}
\put(-43,76){${\rm L}$}

\put(10,72){\circle*{4}}
\put(10,72){\line(0,-1){18}}
\put(25,57){\oval(30,30)[bl]}
\put(24,42){\line(1,0){14}}
\put(13,55){${\rm R}$}
\put(17,76){${\rm L}$}

\put(40,42){\circle*{4}}
\put(40,42){\line(0,-1){18}}
\put(55,27){\oval(30,30)[bl]}
\put(54,12){\line(1,0){15}}
\put(52,12){\circle*{4}}
\put(56,17){$2$}
\put(43,25){${\rm R}$}
\put(47,46){${\rm P}$}

\put(70,42){\circle*{4}}
\put(70,42){\line(0,-1){18}}
\put(85,27){\oval(30,30)[bl]}
\put(84,12){\line(1,0){15}}
\put(82,12){\circle*{4}}
\put(86,17){$3$}
\put(73,25){${\rm R}$}
\put(77,46){${\rm P}$}

\put(100,42){\circle*{4}}
\put(100,42){\line(0,-1){18}}
\put(115,27){\oval(30,30)[bl]}
\put(114,12){\line(1,0){15}}
\put(112,12){\circle*{4}}
\put(116,17){$1$}
\put(103,25){${\rm R}$}
\put(107,46){${\rm A}$}

\put(130,42){\circle*{4}}
\put(130,42){\line(0,-1){18}}
\put(145,27){\oval(30,30)[bl]}
\put(144,12){\line(1,0){15}}
\put(142,12){\circle*{4}}
\put(146,17){$2$}
\put(133,25){${\rm R}$}
\put(137,46){${\rm A}$}

\put(175,72){\circle*{4}}
\put(175,72){\line(0,-1){18}}
\put(190,57){\oval(30,30)[bl]}
\put(189,42){\line(1,0){15}}
\put(187,42){\circle*{4}}
\put(191,47){$3$}
\put(178,55){${\rm R}$}
\put(182,76){${\rm L}$}

\put(205,72){\circle*{4}}
\put(205,72){\line(0,-1){18}}
\put(220,57){\oval(30,30)[bl]}
\put(219,42){\line(1,0){15}}
\put(217,42){\circle*{4}}
\put(221,47){$1$}
\put(208,55){${\rm R}$}
\put(212,76){${\rm A}$}

\put(235,72){\circle*{4}}
\put(235,72){\line(0,-1){18}}
\put(250,57){\oval(30,30)[bl]}
\put(249,42){\line(1,0){15}}
\put(247,42){\circle*{4}}
\put(251,47){$1$}
\put(238,55){${\rm R}$}
\put(242,76){${\rm A}$}

\put(263,72){\circle*{4}}
\put(267,77){$2$}

%\put(199,42){\circle{10}}
%\put(195,39){$3$}

%\put(229,42){\circle{10}}
%\put(226,40){$1$}

%\put(259,42){\circle{10}}
%\put(256,40){$1$}

\put(158,42){\circle*{4}}
\put(162,47){$3$}

%\put(64,12){\circle{10}}
%\put(60.3,9){$2$}

%\put(94,12){\circle{10}}
%\put(90.3,9){$3$}

%\put(124,12){\circle{10}}
%\put(121,10){$1$}

%\put(154,12){\circle{10}}
%\put(151,10){$2$}


\end{picture}

\smallskip

{\bf Example 2}

\medskip

Consider (see before) the lambda term

\smallskip

(1) $\lambda v : \alpha \lamdot ((\lambda y : \alpha \lamdot \lambda x : \alpha \rar \alpha \lamdot x \, (x \, y)) \, v \, ( \lambda u : \alpha \lamdot u))$

\smallskip

We employ a slightly different notation for trees: variables are depicted as labelled lines instead of encircled variable names. See the examples below.

We add as binder of $\alpha$, the head $\lambda \alpha : \ast$ in front of (1), where $\ast$ is a `constant' that can be considered to denote `the type of all types'.

\newpage

\bigskip

{\em Name-carrying lambda tree\/}

\smallskip

\begin{picture}(240,130)(0,10)

\put(20,120){\circle*{4}}
\put(40,120){\circle*{4}}
\put(60,120){\circle*{4}}
\put(80,120){\circle*{4}}
\put(100,120){\circle*{4}}
%\put(120,120){\circle*{4}}
\put(140,120){\circle*{4}}
\put(160,120){\circle*{4}}
\put(180,120){\circle*{4}}
\put(20,120){\line(1,0){175}}

\put(25,125){${\rm L}_\alpha$}
\put(45,125){${\rm L}_v$}
\put(65,125){${\rm A}$}
\put(85,125){${\rm A}$}
\put(105,125){${\rm L}_y$}
\put(145,125){${\rm L}_x$}
\put(165,125){${\rm A}$}
\put(185,125){$x$}

\put(35,35){\oval(30,30)[bl]}
\put(40,20){\circle*{4}}
\put(35,20){\line(1,0){20}}
\put(20,120){\line(0,-1){85}}
\put(45,25){$\ast$}
\put(26,26){${\rm R}$}

\put(55,55){\oval(30,30)[bl]}
\put(60,40){\circle*{4}}
\put(55,40){\line(1,0){20}}
\put(40,120){\line(0,-1){65}}
\put(65,45){$\alpha$}
\put(46,46){${\rm R}$}

\put(75,75){\oval(30,30)[bl]}
\put(80,60){\circle*{4}}
\put(100,60){\circle*{4}}
\put(75,60){\line(1,0){40}}
\put(60,120){\line(0,-1){45}}
\put(66,66){${\rm R}$}
\put(86,65){${\rm L}_u$}
\put(105,65){$u$}
\put(95,40){\line(1,0){20}}
\put(95,55){\oval(30,30)[bl]}
\put(100,40){\circle*{4}}
\put(80,60){\line(0,-1){5}}
\put(86,46){${\rm R}$}
\put(105,45){$\alpha$}

\put(95,95){\oval(30,30)[bl]}
\put(100,80){\circle*{4}}
\put(95,80){\line(1,0){20}}
\put(80,120){\line(0,-1){25}}
\put(105,85){$v$}
\put(86,86){${\rm R}$}

\put(115,115){\oval(30,30)[bl]}
\put(120,100){\circle*{4}}
\put(115,100){\line(1,0){20}}
\put(100,120){\line(0,-1){5}}
\put(125,105){$\alpha$}
\put(106,106){${\rm R}$}

\put(155,75){\oval(30,30)[bl]}
\put(160,60){\circle*{4}}
\put(180,60){\circle*{4}}
\put(155,60){\line(1,0){40}}
\put(140,120){\line(0,-1){45}}
\put(146,66){${\rm R}$}
\put(166,65){${\rm P}_z$}
\put(185,65){$\alpha$}
\put(175,40){\line(1,0){20}}
\put(175,55){\oval(30,30)[bl]}
\put(180,40){\circle*{4}}
\put(160,60){\line(0,-1){5}}
\put(166,46){${\rm R}$}
\put(185,45){$\alpha$}

\put(175,115){\oval(30,30)[bl]}
\put(180,100){\circle*{4}}
\put(200,100){\circle*{4}}
\put(200,80){\circle*{4}}
\put(175,100){\line(1,0){40}}
\put(160,120){\line(0,-1){5}}
\put(166,106){${\rm R}$}
\put(186,105){${\rm A}$}
\put(205,105){$x$}
\put(195,80){\line(1,0){20}}
\put(195,95){\oval(30,30)[bl]}
\put(200,80){\circle*{4}}
\put(180,100){\line(0,-1){5}}
\put(186,86){${\rm R}$}
\put(205,85){$y$}

\end{picture}

\medskip


{\em Name-free lambda tree\/}

\begin{picture}(240,130)(0,10)

\put(20,120){\circle*{4}}
\put(40,120){\circle*{4}}
\put(60,120){\circle*{4}}
\put(80,120){\circle*{4}}
\put(100,120){\circle*{4}}
%\put(120,120){\circle*{4}}
\put(140,120){\circle*{4}}
\put(160,120){\circle*{4}}
\put(180,120){\circle*{4}}
\put(20,120){\line(1,0){175}}

\put(25,125){${\rm L}$}
\put(45,125){${\rm L}$}
\put(65,125){${\rm A}$}
\put(85,125){${\rm A}$}
\put(105,125){${\rm L}$}
\put(145,125){${\rm L}$}
\put(165,125){${\rm A}$}
\put(185,125){$1$}

\put(35,35){\oval(30,30)[bl]}
\put(40,20){\circle*{4}}
\put(35,20){\line(1,0){20}}
\put(20,120){\line(0,-1){85}}
\put(45,25){$\ast$}
\put(26,26){${\rm R}$}

\put(55,55){\oval(30,30)[bl]}
\put(60,40){\circle*{4}}
\put(55,40){\line(1,0){20}}
\put(40,120){\line(0,-1){65}}
\put(65,45){$1$}
\put(46,46){${\rm R}$}

\put(75,75){\oval(30,30)[bl]}
\put(80,60){\circle*{4}}
\put(100,60){\circle*{4}}
\put(75,60){\line(1,0){40}}
\put(60,120){\line(0,-1){45}}
\put(66,66){${\rm R}$}
\put(86,65){${\rm L}$}
\put(105,65){$1$}
\put(95,40){\line(1,0){20}}
\put(95,55){\oval(30,30)[bl]}
\put(100,40){\circle*{4}}
\put(80,60){\line(0,-1){5}}
\put(86,46){${\rm R}$}
\put(105,45){$2$}

\put(95,95){\oval(30,30)[bl]}
\put(100,80){\circle*{4}}
\put(95,80){\line(1,0){20}}
\put(80,120){\line(0,-1){25}}
\put(105,85){$1$}
\put(86,86){${\rm R}$}

\put(115,115){\oval(30,30)[bl]}
\put(120,100){\circle*{4}}
\put(115,100){\line(1,0){20}}
\put(100,120){\line(0,-1){5}}
\put(125,105){$2$}
\put(106,106){${\rm R}$}

\put(155,75){\oval(30,30)[bl]}
\put(160,60){\circle*{4}}
\put(180,60){\circle*{4}}
\put(155,60){\line(1,0){40}}
\put(140,120){\line(0,-1){45}}
\put(146,66){${\rm R}$}
\put(166,65){${\rm P}$}
\put(185,65){$4$}
\put(175,40){\line(1,0){20}}
\put(175,55){\oval(30,30)[bl]}
\put(180,40){\circle*{4}}
\put(160,60){\line(0,-1){5}}
\put(166,46){${\rm R}$}
\put(185,45){$3$}

\put(175,115){\oval(30,30)[bl]}
\put(180,100){\circle*{4}}
\put(200,100){\circle*{4}}
\put(200,80){\circle*{4}}
\put(175,100){\line(1,0){40}}
\put(160,120){\line(0,-1){5}}
\put(166,106){${\rm R}$}
\put(186,105){${\rm A}$}
\put(205,105){$1$}
\put(195,80){\line(1,0){20}}
\put(195,95){\oval(30,30)[bl]}
\put(200,80){\circle*{1}}
\put(180,100){\line(0,-1){5}}
\put(186,86){${\rm R}$}
\put(205,85){$2$}

\end{picture}

\smallskip

{\bf Example 3}

\smallskip

A one-step $\beta_1$-reduct of (1) is

\smallskip

(2) $\lambda v : \alpha \lamdot ((\lambda y : \alpha \lamdot \lambda x : \alpha \rar \alpha \lamdot x \, (x \, v)) \, v \, ( \lambda u : \alpha \lamdot u))$

\medskip

Its name-free lambda tree is

\bigskip

\begin{picture}(240,130)(0,10)

\put(20,120){\circle*{4}}
\put(40,120){\circle*{4}}
\put(60,120){\circle*{4}}
\put(80,120){\circle*{4}}
\put(100,120){\circle*{4}}
%\put(120,120){\circle*{4}}
\put(140,120){\circle*{4}}
\put(160,120){\circle*{4}}
\put(180,120){\circle*{4}}
\put(20,120){\line(1,0){175}}

\put(25,125){${\rm L}$}
\put(45,125){${\rm L}$}
\put(65,125){${\rm A}$}
\put(85,124){\framebox(10,10){${\bf A}$}}
%\put(88,128){\circle{12}}
\put(105,124){\framebox(10,10){${\bf L}$}}
%\put(108,128){\circle{12}}
\put(145,125){${\rm L}$}
\put(165,125){${\rm A}$}
\put(185,125){$1$}

\put(35,35){\oval(30,30)[bl]}
\put(40,20){\circle*{4}}
\put(35,20){\line(1,0){20}}
\put(20,120){\line(0,-1){85}}
\put(45,25){$0$}
\put(26,26){${\rm R}$}

\put(55,55){\oval(30,30)[bl]}
\put(60,40){\circle*{4}}
\put(55,40){\line(1,0){20}}
\put(40,120){\line(0,-1){65}}
\put(65,45){$1$}

\put(46,46){${\rm R}$}

\put(75,75){\oval(30,30)[bl]}
\put(80,60){\circle*{4}}
\put(100,60){\circle*{4}}
\put(75,60){\line(1,0){40}}
\put(60,120){\line(0,-1){45}}
\put(66,66){${\rm R}$}
\put(86,65){${\rm L}$}
\put(105,65){$1$}
\put(95,40){\line(1,0){20}}
\put(95,55){\oval(30,30)[bl]}
\put(100,40){\circle*{4}}
\put(80,60){\line(0,-1){5}}
\put(86,46){${\rm R}$}
\put(105,45){$2$}

\put(95,95){\oval(30,30)[bl]}
\put(100,80){\circle*{4}}
\put(95,80){\line(1,0){20}}
\put(80,120){\line(0,-1){25}}
\put(105,85){\bf 1}
%\put(108,88){\circle{10}}
\put(105,86){\circle{20}}
\put(86,86){${\rm R}$}

\put(115,115){\oval(30,30)[bl]}
\put(120,100){\circle*{4}}
\put(115,100){\line(1,0){20}}
\put(100,120){\line(0,-1){5}}
\put(125,105){$2$}
\put(106,106){${\rm R}$}

\put(155,75){\oval(30,30)[bl]}
\put(160,60){\circle*{4}}
\put(180,60){\circle*{4}}
\put(155,60){\line(1,0){40}}
\put(140,120){\line(0,-1){45}}
\put(146,66){${\rm R}$}
\put(166,65){${\rm P}$}
\put(185,65){$4$}
\put(175,40){\line(1,0){20}}
\put(175,55){\oval(30,30)[bl]}
\put(180,40){\circle*{4}}
\put(160,60){\line(0,-1){5}}
\put(166,46){${\rm R}$}
\put(185,45){$3$}

\put(175,115){\oval(30,30)[bl]}
\put(180,100){\circle*{4}}
\put(200,100){\circle*{4}}
\put(200,80){\circle*{4}}
\put(220,80){\circle*{4}}
\put(175,100){\line(1,0){40}}
\put(160,120){\line(0,-1){5}}
\put(166,106){${\rm R}$}
\put(186,105){${\rm A}$}
\put(205,105){$1$}

\put(195,80){\line(1,0){40}}
\put(195,95){\oval(30,30)[bl]}
\put(200,80){\circle*{4}}
\put(180,100){\line(0,-1){5}}
\put(186,86){${\rm R}$}
%\put(205,85){\bf 2}
\put(202,83){\framebox(10,10){${\bf 2}$}}
%\put(207,88){\circle{12}}
\put(225,85){\bf 1}
%\put(228,88){\circle{10}}
\put(225,86){\circle{20}}

\end{picture}

\newpage

{\bf Comment}

\medskip

-- The added boxes and circles do not belong to the tree; they are only meant to ease the discussion below.

-- The $\beta_1$-reduction is based on the boxed ${\rm A}$-${\rm L}$-pair.

-- The `argument' of the function is the left-most encircled number~1. As a consequence of the described one-step $\beta_1$-reduction, this argument becomes reproduced behind the boxed variable 2. This boxed 2 is bound by the  boxed~L.

-- Note that this leads, in this simple case, to a succession of {\em two unary edges\/}.

-- We see that the syntax for trees must be extended, in order to enable continuation of a tree behind a variable

(VERTALEN:) -- Definitie: Laat p A w L q n een beginstuk van een tak van een lambdaboom zijn, met p en q strings (van A, L en numerieke variabelen), w een well-balanced segment (bijv. A7LRARAL3L) en n een numerieke variabele.
(1) Als n door L gebonden is, is L de  >L-binder<  van n en (in dit geval) A de  >A-binder<  van n. (In een gesloten term heeft elke n een L-binder, maar niet elke n een A-binder, natuurlijk; ik praat even niet over P-binders voor $\Pi$-termen.)
(2) De string A w L q n heet in dit geval (als A de A-binder van n is) de  >definer<  van n.


%-- In order to find the binding place of a variable: start with the value $n$ of this variable; then follow the path towards the root; {\em add\/} $k$ when passing a number $k$ and {\em subtract\/} 1 when passing an L or P. The first L or P on this route that brings the count to 0, is the binding place.

-- In order to find the binding place of a variable: start with the value $n$ of this variable; then follow the path towards the root; when passing a number $k$, {\em jump leftwards\/} over the definer of $k$ and continue the search; when passing an L or P,  {\em subtract\/} 1 and continue the search. When passing an A or R, do nothing. The first L or P on this route that brings the count to 0, is the binding place.
(Note: in order to determine definer(k), one obviously has to do a recursive call to this definition.)


(AANPASSEN:) In the example case, e.g.: start with the one in the right-most encircled~$1$; add~$2$ since one passes the boxed variable with this value; this adds up to $3$; following the path to the root, one encounters two ${\rm R}$'s (which do not influence the procedure); then two ${\rm L}$'s follow (the box around the ${\rm L}$ doe not matter), so subtract $2$, giving $1$ as result; next one passes two ${\rm A}$'s, with no influence; finally, the second upper-most ${\rm L}$ brings the count down to $0$. So this ${\rm L}$ is the binder we are looking for.

-- There is a slight complication in this finding of the binding place, as we explain after the following example. A general procedure will follow later.

-- Note that, technically, $\beta_1$-reduction for name-free lambda calculus amounts to an {\em exact reproduction\/} of the argument behind the targeted variable, represented by the boxed end node bound by the ${\rm L}$ of the ${\rm A}$-${\rm L}$-pair. There is {\em no update needed\/} for the labels of the end nodes concerned (see also the following example). This absence of updating is a major difference with the usual ways of dealing with $\beta$-reduction in name-free lambda calculi.

-- The trees depicted above are no longer binary. This property can be restored by depicting each variable in a `carpenter's square'. For example, the right-most encircled number $1$ can be replaced by
\begin{picture}(30,16)(5,17)
\put(10,20){\circle*{4}}
\put(10,20){\line(1,0){18}}
\put(10,20){\line(0,-1){10}}
\put(18,25){$1$}
\put(6.4,6.9){$\times$}
\end{picture}

\vspace{0.4cm}

When a branch ends in the symbol $\times$, we use the new label ${\rm X}$ to represent the $\times$ in the branch. Hence, the branch in the previous tree corresponding to~$\times$, is ${\rm L \, L \, A \, A \, L \, L \, R \, R \, 2 \, X}$.


\bigskip

{\bf Definition}

\smallskip

A number acting as a label in a name-free lambda tree is called a {\em numerical label\/} or a {\em num-label\/}. All other labels, such as ${\rm A}$, ${\rm L}$ or ${\rm R}$, are called non-numerical labels.

%\newpage

{\bf Example 4}

\medskip

Finally, we depict the lambda tree of another one-step $\beta_1$-reduct of (1) that has been given earlier, viz.\

%\newpage


(3) $\lambda v : \alpha \lamdot ((\lambda y : \alpha \lamdot \lambda x : \alpha \rar \alpha \lamdot x \, ((\lambda u : \alpha \lamdot u)y) \, v \, (\lambda u : \alpha \lamdot u))$

\smallskip

\begin{picture}(240,130)(0,10)


\put(20,120){\circle*{4}}
\put(40,120){\circle*{4}}
\put(60,120){\circle*{4}}
\put(80,120){\circle*{4}}
\put(100,120){\circle*{4}}
%\put(120,120){\circle*{4}}
\put(140,120){\circle*{4}}
\put(160,120){\circle*{4}}
\put(180,120){\circle*{4}}
\put(20,120){\line(1,0){175}}

\put(25,125){${\rm L}$}
\put(45,125){${\rm L}$}
%\put(65,125){${\rm A}$}
\put(85,125){${\rm A}$}
\put(105,125){${\rm L}$}
%\put(145,125){${\rm L}$}
\put(165,125){${\rm A}$}
\put(185,125){$1$}

\put(35,35){\oval(30,30)[bl]}
\put(40,20){\circle*{4}}
\put(35,20){\line(1,0){20}}
\put(20,120){\line(0,-1){85}}
\put(45,25){$0$}
\put(26,26){${\rm R}$}

\put(55,55){\oval(30,30)[bl]}
\put(60,40){\circle*{4}}
\put(55,40){\line(1,0){20}}
\put(40,120){\line(0,-1){65}}
\put(65,45){$1$}
\put(46,46){${\rm R}$}

\put(75,75){\oval(30,30)[bl]}
\put(80,60){\circle*{4}}
\put(100,60){\circle*{4}}
\put(75,60){\line(1,0){41.5}}
\put(60,120){\line(0,-1){45}}
\put(66,66){${\rm R}$}
\put(86,65){${\rm L}$}
\put(105,65){$1$}
\put(95,40){\line(1,0){14}}
\put(95,55){\oval(30,30)[bl]}
\put(100,40){\circle*{4}}
\put(80,60){\line(0,-1){5}}
\put(86,46){${\rm R}$}
\put(105,45){$2$}

\put(95,95){\oval(30,30)[bl]}
\put(100,80){\circle*{4}}
\put(95,80){\line(1,0){20}}
\put(80,120){\line(0,-1){25}}
\put(105,85){$1$}
\put(86,86){${\rm R}$}

\put(115,115){\oval(30,30)[bl]}
\put(120,100){\circle*{4}}
\put(115,100){\line(1,0){20}}
\put(100,120){\line(0,-1){5}}
\put(125,105){$2$}
\put(106,106){${\rm R}$}

\put(155,75){\oval(30,30)[bl]}
\put(160,60){\circle*{4}}
\put(180,60){\circle*{4}}
\put(155,60){\line(1,0){40}}
\put(140,120){\line(0,-1){45}}
\put(146,66){${\rm R}$}
\put(166,65){${\rm P}$}
\put(185,65){$4$}
\put(175,40){\line(1,0){20}}
\put(175,55){\oval(30,30)[bl]}
\put(180,40){\circle*{4}}
\put(160,60){\line(0,-1){5}}
\put(166,46){${\rm R}$}
\put(185,45){$3$}

\put(175,115){\oval(30,30)[bl]}
\put(180,100){\circle*{4}}
\put(200,100){\circle*{4}}
\put(200,80){\circle*{4}}
\put(175,100){\line(1,0){81.5}}
\put(160,120){\line(0,-1){5}}
\put(166,106){${\rm R}$}
\put(186,105){${\rm A}$}
%\put(205,105){\bf 1}
\put(195,80){\line(1,0){20}}
\put(195,95){\oval(30,30)[bl]}
\put(200,80){\circle*{1}}
\put(180,100){\line(0,-1){5}}
\put(186,86){${\rm R}$}
\put(205,85){$2$}

\put(220,100){\circle*{4}}
\put(240,100){\circle*{4}}
%\put(255,100){\line(1,0){40}}
%\put(200,160){\line(0,-1){45}}
%\put(206,106){${\rm R}$}
\put(226,105){${\rm L}$}
\put(245,105){$2$}
\put(235,80){\line(1,0){14}}
\put(235,95){\oval(30,30)[bl]}
\put(240,80){\circle*{4}}
\put(220,100){\line(0,-1){5}}
\put(226,86){${\rm R}$}
\put(245,85){$2$}

\put(65,124){\framebox(10,10){${\bf A}$}}
%\put(88,128){\circle{12}}
\put(145,124){\framebox(10,10){${\bf L}$}}
%\put(208,108){\circle{12}}
\put(202,103){\framebox(10,10){${\bf 1}$}}

\put(97,56){\circle{60}}
\put(237,96){\circle{60}}
%\put(235,96){\circle{37}}


\end{picture}

\smallskip

{\bf Comment}

\medskip

-- Here we start with an ${\rm A}$-${\rm L}$-{\em couple\/}, namely the pair of boxed labels. The `argument' in the $\beta_1$-reduction now is the left-most encircled subtree. After ${\beta_1}$-reduction, this argument is copied after the boxed num-label $1$, for clarity marked by the right-most big circle.

-- But there is a complication here. We show this by searching the binder of the upper-most, right-most num-label $2$ in the example tree, in the right-hand circle. Following the path in the direction of the root, one firstly encounters an ${\rm L}$, so the counter $2$ changes into $1$. Then we meet the boxed $1$, so the counter goes back to $2$. Continuing, the ${\rm A}$ and ${\rm R}$ can be bypassed. Next, the (boxed) ${\rm L}$ brings the counter back to $1$, again.

-- Now one has to {\em disregard\/} all labels between the boxed ${\rm L}$ and the matching ${\rm A}$ to the left of it, for reasons to be explained underneath. So the ${\rm L}$ between the boxed ${\rm A}$ and the boxed ${\rm L}$ does not count. Continuing the counting process, we attain $0$ at the left-most, upper-most ${\rm L}$. This ${\rm L}$ is the binder we aim at.

\bigskip

\newpage

{\bf Explanation}

\smallskip

\begin{picture}(260,130)(0,10)
\multiput(20,130)(5,0){8}{\circle*{1.5}}
\multiput(45,94)(5,-5){3}{\circle*{1.5}}
\multiput(20,120)(5,-5){4}{\circle*{1.5}}
\put(40,100){\circle*{4}}
\put(40,100){\line(1,0){20}}
\put(60,100){\circle*{4}}


\multiput(60,100)(5,-5){4}{\circle*{1.5}}
\put(80,80){\circle*{4}}
\put(80,80){\line(1,0){20}}
\put(100,80){\circle*{4}}
\put(46,105){${\rm L}$}

\put(95,60){\line(1,0){5}}
\put(95,75){\oval(30,30)[bl]}
\put(100,110){\circle*{1}}
\put(80,80){\line(0,-1){5}}
\put(86,66){${\rm R}$}
\put(100,60){\circle*{4}}
\put(86,85){\framebox(10,10){${\bf A}$}}
\put(116,85){${\bf w}$}

\multiput(200,60)(5,-5){4}{\circle*{1.5}}
\put(220,40){\circle*{4}}
\put(220,40){\line(1,0){13}}
%\put(240,40){\circle*{4}}
\put(223,45){${\rm m}$}


\multiput(100,80)(5,0){8}{\circle*{1.5}}
\put(140,80){\circle*{4}}
\put(140,80){\line(1,0){20}}
\put(160,80){\circle*{4}}
\put(146,85){\framebox(10,10){${\bf L}$}}

\multiput(160,80)(5,-5){4}{\circle*{1.5}}
\put(180,60){\circle*{4}}
\put(180,60){\line(1,0){20}}
\put(200,60){\circle*{4}}
\put(186,65){\framebox(10,10){${\rm n}$}}

\multiput(100,60)(5,-5){4}{\circle*{1.5}}
\put(120,40){\circle*{4}}
\put(120,40){\line(1,0){13}}
%\put(140,40){\circle*{4}}
\put(123,45){${\rm m}$}

\put(115,49){\circle{60}}
\put(215,49){\circle{60}}

%\multiput(160,80)(5,0){8}{\circle*{1.5}}
\multiput(145,74)(5,-5){3}{\circle*{1.5}}

\multiput(100,60)(5,0){4}{\circle*{1.5}}
\multiput(200,60)(5,0){4}{\circle*{1.5}}
\end{picture}

\medskip

The picture schematizes the case of an single-step $\beta_1$-reduction, that has been executed on the basis of the boxed ${\rm A}$-${\rm L}$-couple. We allow a well-balanced segment {\bf w} in between the ${\rm A}$ and the ${\rm L}$. The `argument' of the $\beta_1$-reduction is the tree inside the left-most circle. It is copied after the `target' -- the boxed num-label $n$ -- in the right-most circle. This $n$ is bound by the boxed ${\rm L}$.

\medskip

We allow internal num-labels inside a well-balanced segment:

\smallskip

{\bf Definition}

\smallskip

Let $A \, w \, L$ be a segment in a branch $p$ of a bundle $B$.

\noindent (1) $w$ is called {\em well-balanced\/} if

(i) $w$ consists of A-labels, L-labels and num-labels only,

(ii) for each initial segment $w'$ of $w$, the number of A-labels is greater than or equal to the number of L-labels.

\noindent (2) The {\em measure\/} $\mu(w)$ of a well-balanced segment $w$ is the number of L-labels in $w$.

\medskip

Now consider a num-label $m$ in the left-most circle, and assume that it is bound by the un-boxed label ${\rm L}$ at the left. Earlier, we described how to count num-labels, ${\rm L}$'s and ${\rm P}$'s when following the root path from $m$ to the un-boxed ${\rm L}$. Because $m$ is bound by this ${\rm L}$, the count results in $0$ exactly at this ${\rm L}$ (and this is the first time it becomes $0$).

When looking at the `image' of $m$ in the right hand circle, it is clear that the intention is that this num-label $m$ has the same binding label ${\rm L}$ as the original $m$. But the path from the {\em right\/}-hand $m$ to the un-boxed ${\rm L}$ obviously contains other num-labels, ${\rm L}$'s and ${\rm P}$'s, than the one from the {\em left\/}-hand $m$ to the un-boxed ${\rm L}$. So one might wonder whether the right-most num-label $m$ has the correct number to start with.

Now note that the boxed $n$ has been purposely maintained as a `remnant' of the original target of the $\beta_1$-reduction. It links to the `boxed-L'. Moreover, the counting procedure prescribes that the well-balanced segment {\bf w} must be skipped during the count. It is not hard to see that these two provisions ensure that also the right-most $m$ is bound by the un-boxed ${\rm L}$.

\medskip

\newpage

{\bf Definition of binding} (ZIE OOK BLZ 9)

\smallskip

Consider a bound bundle $B$ and $p$ a branch in $B$. Let $n$ be a num-label in $p$, so $p = p_1 \, n \, p_2$.

In order to find the L-binder of $n$, we define a parial function $\varphi$ on initial branches $q \, e$ of $p_1 \, n$. This definition is recursive, going from the complete $p_1 \, n$ downwards to smaller {\em initial\/} branches.

A second induction is on the length of $p$: assume that for all num-labels $m$ in $p_1$, the L-binder already has been established.

\smallskip

(1) {\bf Case $e = k$}. Then $q \, e = q \, k$.

(1a) {\bf Start} If $k = n$ (so $q \, e$ coincides with $p_1 \, n$), $\varphi(q \, e) = n$.

(1b) If $k \not = n$ (so $n$ is an internal label of $p_1$), then $q \, k = q' \, A \, w \, L \, q'' \, k$ for some well-balanced segment $w$, where $L$ is the L-binder of $k$ (already known by induction). Now $\varphi(q') = \varphi(q \, k)$.

(2) {\bf Case $e = L$}. Then $\varphi(q) = \varphi(q \, e) - 1$.

(3) {\bf Case $e = P$ or $R$ or $X$}. Then $\varphi(q) = \varphi(q \, e)$.

\smallskip

Now find the {\em longest\/} initial segment $p' \, L$ of $p_1$ for which $\varphi(p' \, L) = 0$. Then $p' \, L$ is the L-binder of $n$.


\bigskip

{\bf Example}

\smallskip

\begin{picture}(300,60)(-15,0)
\put(-7,50){$\ldots$}
\put(10,50){L}
\put(20,50){L}
\put(30,50){R}
\put(38,48){\framebox(10,10){A}}
%\put(40,40){A}
\put(48,48){\framebox(10,10){L}}
%\put(50,40){L}
\put(60,50){A}
\put(70,50){L}
\put(80,50){L}
\put(90,50){R}
\put(100,50){L}
\put(108,48){\framebox(10,10){A}}
%\put(110,40){A}
\put(120,50){A}
\put(130,50){L}
\put(140,50){A}
\put(150,50){L}
\put(158,48){\framebox(10,10){L}}
%\put(160,40){L}
\put(170,50){R}
\put(180,50){L}
\put(190,50){A}
\put(200,50){R}
\put(210,50){2}
\put(220,50){R}
\put(230,50){L}
\put(240,50){5}
\put(250,50){L}
\put(260,50){L}
\put(270,50){R}
\put(280,50){4}
\put(290,50){$\ldots$}

\put(282,10){4}
\put(272,10){4}
\put(262,10){3}
\put(252,10){2}

\put(242,20){5}
\put(232,20){4}
\put(222,20){4}

\put(211.5,30){2}
\put(201.5,30){2}
\put(191.5,30){2}
\put(182,30){1}
\put(172,30){1}
\put(162,30){0}

\put(102,20){3}
\put(92,20){3}
\put(82,20){2}
\put(72,20){1}
\put(62,20){1}
\put(52,20){0}

\put(32,10){2}
\put(22,10){1}
\put(12,10){0}

\put(163,38){\vector(0,1){8}}
\put(54,28){\vector(0,1){18}}
\put(14,19){\vector(0,1){28}}


\put(213,47){\vector(0,-1){9}}
\put(244,47){\vector(0,-1){19}}
\put(284,47){\vector(0,-1){28}}

\put(216,22){\vector(-1,0){105}}
\put(246,12){\vector(-1,0){205}}

\put(38,60){\line(1,0){20}}
\put(108,60){\line(1,0){60}}




\end{picture}

\medskip

One can also find the binding $L$ by a non-recursive procedure, in a similar manner as the recursive one above. We assume that the picture speaks for itself, in comparison with the previous one.

\bigskip

{\bf Example 2}

\begin{picture}(300,75)(-15,-10)

\put(-7,50){$\ldots$}
\put(10,50){L}
\put(20,50){L}
\put(30,50){R}
\put(38,48){\framebox(10,10){A}}
%\put(40,40){A}
\put(48,48){\framebox(10,10){L}}
%\put(50,40){L}
\put(60,50){A}
\put(70,50){L}
\put(80,50){L}
\put(90,50){R}
\put(100,50){L}
\put(108,48){\framebox(10,10){A}}
%\put(110,40){A}
\put(120,50){A}
\put(130,50){L}
\put(140,50){A}
\put(150,50){L}
\put(158,48){\framebox(10,10){L}}
%\put(160,40){L}
\put(170,50){R}
\put(180,50){L}
\put(190,50){A}
\put(200,50){R}
\put(210,50){2}
\put(220,50){R}
\put(230,50){L}
\put(240,50){5}
\put(250,50){L}
\put(260,50){L}
\put(270,50){R}
\put(280,50){4}
\put(290,50){$\ldots$}

%\put(-10,-5){\line(1,0){320}}

\put(282,30){4}
\put(272,30){4}
\put(262,30){3}
\put(252,30){2}

\put(245,28){\line(0,-1){28}}
\put(242,-10){5\!+\!2}
\put(253,28){\line(0,-1){18}}
\put(255,28){\line(0,-1){20}}
\put(53,28){\line(0,-1){20}}
\put(55,28){\line(0,-1){18}}
\put(253,10){\line(-1,0){198}}
\put(255,8){\line(-1,0){202}}
\put(242,30){7}
\put(232,30){6}
\put(222,30){6}

\put(215,28){\line(0,-1){28}}
\put(212,-10){2\!+\!6}
\put(223,28){\line(0,-1){8}}
\put(225,28){\line(0,-1){10}}
\put(163,28){\line(0,-1){10}}
\put(165,28){\line(0,-1){8}}
\put(223,20){\line(-1,0){58}}
\put(225,18){\line(-1,0){62}}
\put(211.5,30){8}
\put(201.5,30){8}
\put(191.5,30){8}
\put(182,30){7}
\put(172,30){7}
\put(162,30){6}

\put(102,30){5}
\put(92,30){5}
\put(82,30){4}
\put(72,30){3}
\put(62,30){3}
\put(52,30){2}

\put(32,30){2}
\put(22,30){1}
\put(12,30){0}

\put(156,32){\vector(-1,0){45}}
\put(47,32){\vector(-1,0){6}}

\put(38,60){\line(1,0){20}}
\put(108,60){\line(1,0){60}}


\end{picture}

\medskip

{\bf Formal definition of $\beta_1$-reduction for bundles}

\smallskip

Let $B$ be a bundle and $p$ a branch in $B$. Assume that $p$ has the form $q \, {\rm A} \, {\bf w} \, {\rm L} \, r \, m$, where $q$ and $r$ are segments, ${\bf w}$ a well-balanced segment and $m$ a num-label bound by ${\rm L}$. Then a (single-step) $\beta_1$-reduction can be performed.
{\em See the procedure described before.}

\medskip

\newpage

{\bf Definition}

\medskip

The {\em grid\/} is the infinite rooted binary tree, with labels $l$ and $r$ along the edges, denoting whether one is `going' {\em left\/} or {\em right\/}. Each subtree of the grid can be described by the set of its {\em paths\/}, the $l$-$r$-{\em bundle}. A subtree can be either infinite or finite.

Each lambda tree $T$ has a natural injective mapping $g(T)$ into the grid, that is obtained by replacing the labels of $T$ as follows:

\smallskip

$\varepsilon \rar \varepsilon$,

${\rm A} \rar l$,

${\rm L}_x \rar l$, for every $x$,

$x \rar l$,

${\rm R} \rar r$,

${\rm X} \rar r$.

A consequence is, that the roots of $T$ and $g(T)$ coincide.

\medskip

Example: the $g$-map of Example~\ref{-}, expressed as an $l$-$r$-bundle, is:

\smallskip

$l \, l \, l \, l$, corresponding to ${\rm A} \, {\rm L}_x \, {\rm A} \, x$,

$l \, l \, r \, l$, for ${\rm A} \, {\rm L}_x \, {\rm R} \, x$,

$l \, r$, for ${\rm A} \, \times$,

$r \, l \, l \, l$, for ${\rm R} \, {\rm L}_y \, {\rm A} \, y$,

$r \, l \, r \, l$, for ${\rm R} \, {\rm L}_y \, {\rm R} \, y$,

$r \, r $, for ${\rm R} \, \times$.

\smallskip

Note: We use the cross $\times$ for the non-given types in untyped lambda calculus.

\medskip

{\bf Lemma}

\smallskip

Let $T$ and $T'$ be bundles with $T \rrar_{\beta_1} T'$.

(1) Then $g(T)$ is a subtree of $g(T')$.

(2) Moreover, let $T' \triangleleft \, g(T)$ be the part of $T'$ corresponding to $g(T)$. Then $T' \triangleleft \, g(T)$ is a subtree of $T'$.

\smallskip

Hence, in $\beta_1$-reduction, not only the grid, but also the lambda-labels such as ${\rm A}$, ${\rm L}_x$, \ldots are `respected'. This can also be formulated as follows:

\medskip

{\bf Lemma}

\smallskip

Let $B$ be a bundle, $B \rrar_{\beta_1} B_1$ and $B \rrar_{\beta_1} B_2$. Let $p$ and $q$ be initial segments of branches $B_1$ and $B_2$, respectively. Assume that $p$ and $q$ have equal length and that $g(p) = g(q)$. Then $p$ and $q$ are identical. $\Box$

\smallskip

Consequently, for every grid point that is `occupied' by both $B_1$ and $B_2$, being reducts of a bundle $B$, corresponding segments of $B_1$ and $B_2$ passing through this point have the same label.

A consequence is that in each step of a $\beta_1$-reduction $T \rar_{\beta_1} T_1 \rar_{\beta_1} T_2 \ldots $, the lambda tree becomes {\em extended\/}. So $T_{n+1}$ is an extension of $T_n$, where extension takes place at one specific end node of $T_n$ being the `target' of the local $\beta_1$-reduction. This extension does not have any effect at all on the `remnants' of the original $T_n$ in $T_{n+1}$.

\medskip

\newpage

{\bf Definition}

\smallskip

Let $T$ be a bundle. The union of all bundles $T'$ such that $T \rrar_{\beta_1} T'$ is called the {\em pattern\/} of $T$. Notation: $\pi(T)$. (The pattern $\pi(T)$ can be infinite.)

\medskip

{\bf Lemma}

\smallskip

If the pattern is finite, then $\pi(T)$ is the $\beta_1$-{\em normal form\/} of $T$.

\medskip



{\bf Note}

\smallskip

Let $T_1$ and $T_2$ be $\beta_1$-reducts of $T$. Then it is possible to reduce $T_1$ and $T_2$ to a common reduct
$T_3$ that is the union of $T_1$ and $T_2$ (Church-Rosser; give a proof?). Both $T_1$ and $T_2$ are subtrees of $\pi(T)$.

\vspace{0.4cm}

{\bf Definition}

\smallskip

From now on, we differentiate between name-carrying and name-free bundles, the sets of which we denote by ${\bf c}$ and ${\bf f}$, respectively. The corresponding relations are denoted with superscripts, for example $\stackrel{\bf c~~}{\rar_{\beta_1}}$ for the one-step $\beta_1$-reduction between name-carrying bundles.

\medskip

Let $B \in {\bf c}$. Let's say that the mapping $\varphi: {\bf c} \rar {\bf f}$ transforms a name-carrying into the corresponding name-free version of a bundle, in the manner described in the previous examples and comments.

\medskip

{\bf Theorem}

\smallskip

(1) Let $B \in {\bf c}$ and $B \stackrel{\bf c~~}{\rar_{\beta_1}} B'$. Then $B' \in {\bf c}$.

(2) Let $B \in {\bf f}$ and $B \stackrel{\bf f~~}{\rar_{\beta_1}} B'$. Then $B' \in {\bf f}$.

\medskip

This means that reducts of bundles in {\bf c} or {\bf f}, respectively, are, again, proper bundles in the same sets.

\medskip

{\bf Theorem}

\smallskip

Let ${\rm B}$ be a bundle in {\bf c}, with ${\rm B} \stackrel{\bf c~~}{\rar_{\beta_1}} {\rm B}'$.
Then $\varphi({\rm B}) \stackrel{\bf f~~}{\rar_{\beta_1}} \varphi({\rm B'})$.

\smallskip

So we have the commutative diagram

\smallskip

\begin{picture}(50,50)(0,0)
\put(16,38){${\rm B}$}
\put(76,38){${\rm B}'$}
\put(11,8){$\varphi({\rm B})$}
\put(71,8){$\varphi({\rm B}')$}
\put(20,32){\vector(0,-1){10}}
\put(80,32){\vector(0,-1){10}}
\put(43,38){$\stackrel{\bf c~~}{\rar_{\beta_1}}$}
\put(44,8){$\stackrel{\bf f~~}{\rar_{\beta_1}}$}
\end{picture}

\medskip



{\bf Corollary}

\smallskip

Let ${B}$ be a bundle in {\bf c}, with ${B} \stackrel{\bf c~~}{\rrar_{\beta_1}} {B}'$.
Then $\varphi({B}) \stackrel{\bf f~~}{\rrar_{\beta_1}} \varphi({B'})$.

\medskip

{\bf Definition}

\smallskip

(1) Let ${B}$ be a bundle in {\bf f}. Num-labels $n$ occurring at the leaves of ${B}$ are called {\em leaf labels\/}. All other num-labels occurring in ${B}$, are {\em internal num-labels\/} of ${B}$.

(2) Let ${B}$ be a bundle in {\bf f} such that {\em all\/} num-labels $n$ in ${B}$ are leaf labels. Then this ${B})$ is called an {\em original\/} bundle. All other bundles are called {\em derived\/} bundles.

\medskip

{\bf Definition}

\smallskip

A bundle ${B'}$ such that there exists an original bundle ${B}$ with ${B} \stackrel{\bf f~~}{\rrar_{\beta_1}} {B'}$, is called {\em regular\/}.

\medskip

{\bf Lemma}

Let ${B}$ be a regular, derived bundle.

(1) Assume ${B} \stackrel{\bf f~~}{\rar_{\beta_1}} {B}'$ (in one step). All internal num-labels of ${B}$ can be found at the same grid positions in ${B}'$, and there is exactly one leaf label in ${B}$ that has been changed into an internal num-label in ${B}'$.

(2) It follows that there is a one-to-one correspondence between the internal num-labels in ${B}$ and the $\stackrel{\bf f~~}{\rrar_{\beta_1}}$-steps leading from the original bundle to ${B}$.

(3) There is at least one internal num-label $n$, final element of an initial segment $p \, n$, such that the sub-bundle belonging to $p \, n$ has no internal num-labels.

%\newpage

{\bf Definition}

\smallskip

A num-label $n$ as mentioned in (3) is called a {\em final\/} internal num-label.


\medskip

{\bf Proof} of (3)

\smallskip

Since $B$ is a derived bundle, there must at least one internal num-label in $B$. Let $p \, n$ be the {\em smallest\/} branch in $B$ that has at least one internal num-label. Let  $m$ be the right-most internal num-label in $p \, n$ and assume $p = q \, m \, q_1$. Let $q \, B'$ be the subbundle of $q \, m$. If $B'$, at its turn, has at least one internal num-label, then repeat the process as follows:

Find the smallest $q \, m \, q'$ in $q \, B'$ that has an internal num-label. Let $m'$ be the right-most internal num-label in $q'$ and $q' = r \, m' \, r_1$. Consider the subbundle of $q \, m \, r$, say $q \, m \, r \, B''$. Repeat the process until we have that $s \, k \, s_1 \in q \, B'$ and the subbundle of $s$ has {\em no\/} internal num-labels, so $k$ is final.

\medskip

{\bf Definition}

\smallskip

The num-label $s$ obtained in the last step of this proof, is called the {\em primal\/} num-label of the regular, derived bundle $B$.


\medskip

{\bf Lemma}

\smallskip

Let ${B}$ be a regular bundle and ${B} \frrar {B}_2$. Assume that $n$ is the primal num-label of ${B}_2$, being the final element of an initial segment $p \, n$. Let $p \, {B}'$ be the sub-bundle belonging to $p$.

Define the bundle ${B}_1$ to be ${B}_2$ in which the subbundle $p \, {B}'$ has been replaced by $p \, n$ only. Then ${B} \frrar {B_1}$. $\Box$

\medskip

{\bf Theorem}

\smallskip

Let ${B}'$ be a bundle in {\bf f}.

(1) Then there is a procedure, based on the internal num-labels of ${B}'$, to find an {\em original\/} bundle ${B}$ such that ${B} \stackrel{\bf f~~}{\rrar_{\beta_1}} {B}'$.

(2) Moreover, such a bundle ${B}$ is {\em unique\/}.

(3) The mentioned procedure produces a list of bundles ${B}_1, \ldots, {B}_m$ such that
${B} \stackrel{\bf f~~}{\rar_{\beta_1}} {B}_n \stackrel{\bf f~~}{\rar_{\beta_1}} \ldots \stackrel{\bf f~~}{\rar_{\beta_1}} {B}_1 \stackrel{\bf f~~}{\rar_{\beta_1}} {B}'$.

\medskip

{\bf Proof}

\smallskip

Use the previous lemma. Induction on the number of internal num-labels in ${B}'$. $\Box$



\vspace{0.4cm}

{\bf Definition: $\beta_2$-reduction}

\smallskip

Let $B$ be a bundle and $q$ an initial segment of a branch in $B$, such that $q = r \, {\rm A} \, w \, {\rm L}_x$ with $w$ a well-balanced segment. Let $q \, B_1$ be the sub-bundle belonging to $q$ and assume that there is no free $x$ in $B_1$ (otherwise said: the final element ${\rm L}_x$ of $q$ does not bind any $x$).

Now we can apply $\beta_2$-reduction to $B$, by omitting the (now `useless') ${\rm A}$-${\rm L}$-couple. This amounts to replacing $q \, B_1$ by $r \, w \, B_1$. The result is a new bundle (lemma!) $B'$ and we denote this reduction by $B \, \beta_2 \, B'$.

Note that $\beta_2$-reduction acts on a {\em bundle\/} of branches, contrary to the $\beta_1$-reduction in this paper that relates single branches.

\vspace{0.4cm}

{\bf To be investigated}

\smallskip

-- Check all definitions, lemma's and theorems and give proofs where necessary. In particular, the {\bf name-free treatment of beta-reduction} needs to be thoroughly checked. It is easy to make mistakes.

\smallskip

-- {\bf Compare the sketched approach with the usual one,} for example as to the common theorem about how to find a normal form (if it exists).

\smallskip

-- Look at work of Danos \& Regnier, van Oostrom. Maybe L\'{e}vy, Lamping, Asperti; Abadi, Curien.

\smallskip

-- Find out whether this approach has more advantages than the well-known ones (about the absence of update functions). In short: {\bf judge the value of all this}.

\smallskip

-- Try to find more useful theorems about the name-free version.

\smallskip

-- Mention the paper in which four different reductions are introduced with Greek letters in front. I have forgotten the reference. One of these reductions corresponds to the $A$-$L$-reduction for an $A$-$L$-couple.

\smallskip

-- Give more references to papers on name-free calculus.

\smallskip

-- Find out whether more results described in my thesis are of use (weak normalization, strong normalization, ...). Church-Rosser?

\smallskip

-- Maybe: Define a refined $\beta_1$-reduction such that -- starting from a lambda term $M$ -- in a reduction $(\lambda x \lamdot M)N$ every previously obtained (legal) $N'$ with $M \rrar_{\beta_1} C[N']$ and $N' \rrar_{\beta_1} N$ can be taken for $N$, so  $(\lambda x \lamdot M)N \rar_{\beta_0} M[x := N']$ for every earlier obtained $N' \rrar_{\beta_1} N$. Demonstrate this in the namefree version. Discuss its properties and use.

\smallskip

-- Let $B_1$ and $B_2$ be bundles. Define the union $B_1 \cup B_2$ as the smallest bundle containing $B_1$ and $B_2$. Also define the intersection of $B_1$ and $B_2$.

Let $B$ be a bundle and assume $B \rred B_1$ and $B \rred B_2$. Investigate whether $B \rred B_1 \cup B_2$. If not, define the smallest bundle $B'$ including $B_1$ and $B_2$ such that $B \rred B'$. Investigate $B_1 \cap B_2$.

-- Laat ook zien dat een equivalente boomrepresentatie bestaat door de R weg te laten. Dan zijn de bomen niet meer binair, maar verder blijft alles gelijk.

-- Geef aan hoe een subterm van een lambdaterm t te vinden is door in een deelboom van de boom B van t een aantal R's weg te laten. Die deelboom bevat alles onder en gelijk aan een tak in B.

-- Bewijs `zwakke normalisatie => sterke normalisatie' door te laten zien dat de boom van een normaalvorm van een term een bovengrens is van alle reductiebomen van die term.

-- Voeg een sectie toe die in het kort de belangrijkste resultaten over dit onderwerp opsomt (Vincent vragen?).


\bigskip


\begin{thebibliography}{8}

\bibitem[de Bruijn, 1972]{deB72}
de Bruijn, N.G., Lambda calculus notation with nameless dummies, a tool for automatic formula manipulation, with application to the Church-Rosser theorem, {\em Indagationes Math.\/} 34 (1972), 381--392. Also in \cite{NedGeuDeV}.

\bibitem[de Bruijn, 1978a]{deB78a}
de Bruijn, N.G., {\em A namefree lambda calculus with facilities for internal definitions of expressions and segments\/}, Eindhoven University of Technology, EUT-report 78-WSK-03, 1978. (See also The Automath Archive AUT 050,
www.win.tue.nl>Automath.)

\bibitem[de Bruijn, 1978b]{deB78b}
de Bruijn, N.G., Lambda calculus notation with namefree formulas involving symbols that represent reference transforming mappings, {\em Indagationes Math.\/} 40 (1978), 348--356. Also in \cite{NedGeuDeV}. (See also The Automath Archive AUT 055, www.win.tue.nl>Automath.)

\bibitem[Nederpelt, 1973]{Ned73}
Nederpelt, R.P., {\em Strong normalisation in a $\lambda$-typed $\lambda$-calculus}. Ph.D.\ thesis, Eindhoven University of Technology, 1973. Also in \cite{NedGeuDeV}.

\bibitem[Nederpelt {\it et al.}, 1994]{NedGeuDeV}
Nederpelt, R.P., Geuvers, J.H.\ and de Vrijer, R.C., eds, 1994:
{\em Selected Papers on Automath\/},
North-Holland, Elsevier.



\end{thebibliography}
%XXXXX Subbundle of $p$ (is a bundle, indeed). Beta reduction with `substitution' of the subbundle of $p \; L$.

%Notation $p \; B$ invoeren

\end{document}

\begin{picture}(300,140)(0,-30)
\put(10,80){$\langle \lambda z : (\Pi x : S \lamdot \Pi y : S \lamdot Q \; x \; y) \lamdot \lambda u : S \lamdot z \; u \; u \rangle ~~=$}

\put(10,12){\line(1,0){247}}

\put(10,12){\circle*{4}}
\put(10,12){\line(0,1){18}}
\put(25,27){\oval(30,30)[tl]}
\put(24,42){\line(1,0){14}}
\put(14,17){${\rm U}$}


\put(40,42){\circle*{4}}
\put(40,42){\line(0,1){8}}
\put(55,47){\oval(30,30)[tl]}
\put(54,62){\line(1,0){5}}
\put(44,48){${\rm U}$}

\put(70,42){\circle*{4}}
\put(70,42){\line(0,1){8}}
\put(85,47){\oval(30,30)[tl]}
\put(84,62){\line(1,0){5}}
\put(74,48){${\rm U}$}

\put(100,42){\circle*{4}}
\put(100,42){\line(0,-1){8}}
\put(115,37){\oval(30,30)[bl]}
\put(114,22){\line(1,0){5}}
\put(104,32){${\rm D}$}

\put(130,42){\circle*{4}}
\put(130,42){\line(0,-1){8}}
\put(145,37){\oval(30,30)[bl]}
\put(144,22){\line(1,0){5}}
\put(134,32){${\rm D}$}

\put(175,12){\circle*{4}}
\put(175,12){\line(0,1){8}}
\put(190,17){\oval(30,30)[tl]}
\put(189,32){\line(1,0){5}}
\put(179,18){${\rm U}$}

\put(205,12){\circle*{4}}
\put(205,12){\line(0,-1){8}}
\put(220,7){\oval(30,30)[bl]}
\put(219,-8){\line(1,0){5}}
\put(209,2){${\rm D}$}

\put(235,12){\circle*{4}}
\put(235,12){\line(0,-1){8}}
\put(250,7){\oval(30,30)[bl]}
\put(249,-8){\line(1,0){5}}
\put(239,2){${\rm D}$}

\put(40,42){\line(1,0){112}}


\end{picture}

${\rm A}~{\rm T}_x ~ \parallel ~ {\rm A}~ \parallel ~ \parallel ~ {\rm T}_x$

${\rm A}~{\rm T}_x ~ \slash ~{\rm A}~ \slash ~ \slash ~ {\rm T}_x$

${\rm A}~{\rm T}_x ~ \ell~ {\rm A}~ \ell ~ \ell ~ {\rm T}_x$

${\rm A}~{\rm T}_x ~ \mid ~{\rm A}~ \mid ~ \mid ~ {\rm T}_x$

${\rm A} \, {\rm T}_x  \mid {\rm A} \mid  \mid  {\rm T}_x$

${\rm A} \, {\rm T}_x \, {\rm I} \, {\rm A} \, {\rm I} \, {\rm I} \, {\rm T}_x$

${\rm A} \, {\rm T}_x \, \vee \, {\rm A} \, \vee \, \vee \, {\rm T}_x$

${\rm A} \, {\rm T}_x \, \wr \, {\rm A} \, \wr \, \wr \, {\rm T}_x$

${\rm A} \, {\rm T}_x \, {\rm D} \, {\rm A} \, {\rm D} \, {\rm D} \, {\rm T}_x$
%\end{document}

\begin{picture}(300,110)(-60,0)
\put(10,100){$\langle \lambda z : (\Pi x : S \lamdot \Pi y : S \lamdot Q \; x \; y) \lamdot \lambda u : S \lamdot z \; u \; u \rangle ~~=$}

\put(10,72){\line(1,0){247}}
\put(40,42){\line(1,0){111}}

\multiput(10,72)(-13.25,0){5}{\line(-1,0){7}}
\put(-50,72){\circle*{4}}
\put(-20,72){\circle*{4}}
\put(-43,76){${\rm L}_S$}
\put(-13,76){${\rm L}_Q$}

\put(10,72){\circle*{4}}
\put(10,72){\line(0,-1){18}}
\put(25,57){\oval(30,30)[bl]}
\put(24,42){\line(1,0){14}}
\put(13,55){${\rm R}$}
\put(17,76){${\rm L_z}$}

\put(40,42){\circle*{4}}
\put(40,42){\line(0,-1){18}}
\put(55,27){\oval(30,30)[bl]}
\put(54,12){\line(1,0){5}}
\put(43,25){${\rm R}$}
\put(47,46){${\rm P}_x$}

\put(70,42){\circle*{4}}
\put(70,42){\line(0,-1){18}}
\put(85,27){\oval(30,30)[bl]}
\put(84,12){\line(1,0){5}}
\put(73,25){${\rm R}$}
\put(77,46){${\rm P}_y$}

\put(100,42){\circle*{4}}
\put(100,42){\line(0,-1){18}}
\put(115,27){\oval(30,30)[bl]}
\put(114,12){\line(1,0){5}}
\put(103,25){${\rm R}$}
\put(107,46){${\rm A}$}

\put(130,42){\circle*{4}}
\put(130,42){\line(0,-1){18}}
\put(145,27){\oval(30,30)[bl]}
\put(144,12){\line(1,0){5}}
\put(133,25){${\rm R}$}
\put(137,46){${\rm A}$}

\put(175,72){\circle*{4}}
\put(175,72){\line(0,-1){18}}
\put(190,57){\oval(30,30)[bl]}
\put(189,42){\line(1,0){5}}
\put(178,55){${\rm R}$}
\put(182,76){${\rm L}_u$}

\put(205,72){\circle*{4}}
\put(205,72){\line(0,-1){18}}
\put(220,57){\oval(30,30)[bl]}
\put(219,42){\line(1,0){5}}
\put(208,55){${\rm R}$}
\put(212,76){${\rm A}$}

\put(235,72){\circle*{4}}
\put(235,72){\line(0,-1){18}}
\put(250,57){\oval(30,30)[bl]}
\put(249,42){\line(1,0){5}}
\put(238,55){${\rm R}$}
\put(242,76){${\rm A}$}

\put(262,72){\circle{10}}
\put(259,70){$z$}

\put(199,42){\circle{10}}
\put(195,39){$S$}

\put(229,42){\circle{10}}
\put(226,40){$u$}

\put(259,42){\circle{10}}
\put(256,40){$u$}

\put(156,42){\circle{10}}
\put(152,39){$Q$}

\put(64,12){\circle{10}}
\put(60.3,9){$S$}

\put(94,12){\circle{10}}
\put(90.3,9){$S$}

\put(124,12){\circle{10}}
\put(121,10){$y$}

\put(154,12){\circle{10}}
\put(151,10){$x$}


\end{picture}



\begin{picture}(300,120)(0,0)
\put(10,100){$\langle \lambda z : (\Pi x : S \lamdot \Pi y : S \lamdot Q \; x \; y) \lamdot \lambda u : S \lamdot z \; u \; u \rangle ~~=$}
\put(10,72){\line(1,0){265}}
\put(24,42){\line(1,0){141}}

\put(10,72){\circle*{4}}
\put(10,72){\line(0,-1){18}}
\put(25,57){\oval(30,30)[bl]}
\put(24,42){\line(1,0){4}}
\put(13,55){${\rm R}$}
\put(17,76){${\rm L}_z$}

\put(30,42){\circle*{4}}
\put(30,42){\line(0,-1){18}}
\put(45,27){\oval(30,30)[bl]}
\put(44,12){\line(1,0){15}}
\put(33,25){${\rm R}$}
\put(37,46){${\rm P}_x$}

\put(60,42){\circle*{4}}
\put(60,42){\line(0,-1){18}}
\put(75,27){\oval(30,30)[bl]}
\put(74,12){\line(1,0){15}}
\put(63,25){${\rm R}$}
\put(67,46){${\rm P}_y$}

\put(90,42){\circle*{4}}
\put(90,42){\line(0,-1){18}}
\put(105,27){\oval(30,30)[bl]}
%\put(104,12){\line(1,0){5}}
\put(104,12){\line(1,0){15}}
\put(93,25){${\rm R}$}
\put(97,46){${\rm A}$}

\put(120,42){\circle*{4}}
\put(120,42){\line(0,-1){18}}
\put(135,27){\oval(30,30)[bl]}
%\put(144,12){\line(1,0){5}}
\put(134,12){\line(1,0){15}}
\put(123,25){${\rm R}$}
\put(127,46){${\rm A}$}

\put(170,72){\circle*{4}}
\put(170,72){\line(0,-1){18}}
\put(185,57){\oval(30,30)[bl]}
\put(184,42){\line(1,0){15}}
\put(173,55){${\rm R}$}
\put(177,76){${\rm L}_u$}

\put(200,72){\circle*{4}}
\put(200,72){\line(0,-1){18}}
\put(215,57){\oval(30,30)[bl]}
\put(214,42){\line(1,0){15}}
\put(203,55){${\rm R}$}
\put(207,76){${\rm A}$}

\put(230,72){\circle*{4}}
\put(230,72){\line(0,-1){18}}
\put(245,57){\oval(30,30)[bl]}
\put(244,42){\line(1,0){15}}
\put(233,55){${\rm R}$}
\put(237,76){${\rm A}$}

\put(260,72){\circle*{4}}
\put(264,77){$z$}

\put(184,42){\circle*{4}}
\put(188,47){$S$}

\put(214,42){\circle*{4}}
\put(218,47){$u$}

\put(244,42){\circle*{4}}
%\put(258,42){\line(1,0){15}}
\put(248,47){$u$}

\put(150,42){\circle*{4}}
\put(154,46){$Q$}

\put(44,12){\circle*{4}}
\put(48,17){$S$}

\put(74,12){\circle*{4}}
\put(78,17){$S$}

\put(104,12){\circle*{4}}

\put(108,17){$y$}

\put(134,12){\circle*{4}}

\put(138,17){$x$}
\end{picture}

\put(150,20){\circle{10}}
\put(150,40){\circle{10}}
\put(150,60){\circle*{4}}
\put(150,80){\circle*{4}}
\put(150,100){\circle*{4}}
\put(150,120){\circle*{4}}
\put(150,140){\circle*{4}}
\put(150,160){\circle*{4}}
\put(150,180){\circle*{4}}

%\put(170,28){\circle{13.5}}
%\put(170,40){\circle{10}}
\put(170,35){\oval(10,20)}
\put(161,30){\vector(1,0){2}}
\put(141,20){\vector(1,0){2}}

\put(170,80){\circle{10}}
%\put(170,40){\circle{13.5}}
\put(170,100){\circle*{4}}
\put(170,120){\circle{10}}
\put(170,140){\circle{10}}
\put(170,160){\circle*{4}}


\put(150,60){\line(1,-1){16.1}}
\put(150,120){\line(1,-1){20}}
\put(150,140){\line(1,-1){16.3}}
\put(150,180){\line(1,-1){20}}
%\put(20,240){\line(1,-1){20}}
%\put(20,60){\line(1,-1){20}}

\put(150,180){\line(0,-1){135}}
\put(150,35){\line(0,-1){10.5}}
\put(170,100){\line(0,-1){15}}
%\put(30,140){\line(0,-1){15}}
\put(170,160){\line(0,-1){15}}
%\put(40,160){\line(0,1){20}}
%\put(40,200){\line(0,1){20}}
%\put(60,60){\line(0,1){20}}

\put(159,53){${\rm S}$}
\put(155,117){${\rm S}$}
\put(154,138){${\rm S}$}
\put(159,173){${\rm S}$}

\put(140,25){${\rmL}$}
\put(140,49){${\rm A}$}
\put(140,67){${\rm L}$}
\put(140,87){${\rmL}$}
\put(140,107){${\rmA}$}
\put(140,127){${\rm A}$}
\put(135,147){$\fbox{\rm L}$}
\put(134,167){$\fbox{\rm A}$}
\put(160,88){${\rm L}$}
\put(162.5,148.5){${\rm L}$}

%\put(150,22.2){\circle*{2}}
\put(147.5,17){$2$}
\put(147.5,37){$2$}
%\put(170,32.4){\circle*{2}}
\put(167.5,27){$1$}
\put(167.5,37){$1$}
\put(167.5,77){$2$}
\put(167.5,117){$1$}
\put(167.5,137){$1$}

%\put(170,150){\oval(20,18)}
%\put(106.5,113){\line(1,-4){27.25}}
%\put(133.62,4){\line(1,0){30.38}}
%\put(164,4){\vector(1,4){4.3}}
