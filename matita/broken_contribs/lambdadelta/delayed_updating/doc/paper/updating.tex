\section{Updating}
\label{sec:updating}

In name-carrying systems of $\lambda$-calculus
a binder of a term $M$, say $\lambda_x$, and the variable occurrences
that refer to it carry the same name, say $x$.
On the contrary, name-free systems use unnamed binders, say $\lambda$,
and replace a bound variable occurrence $x$
with an index that is a non-negative integer denoting the position of
the corresponding $\lambda_x$ along the path connecting $x$ to the
root of $M$ in the representation of $M$ as an abstract syntax tree.
As we pointed out in the introduction,
the $\beta$-reduction step of the latter systems  
requires updating the indexes occurring in a copied
argument, say $N$, to maintain the relationship between the bound variable
instances in $M$ and the respective binders.
Depending on the particular system,
if immediate updating is in effect, the update occurs by 
applying a so-called update function to the indexes in $N$. On the
contrary, if delayed updating is in effect, the update function
is just stored in the syntax of the copied $N$.

The first name-free systems with immediate updating appear in \cite{Bru72}
with the basic update functions $\tau_{d,h}$ of type $\naturals^+ \to \naturals^+$,
where $d\in\naturals$ and $h\in\naturals$.
\[
\tau_{d,h} \defeq i \mapsto
\left\{\begin{tabular}{ll}
$i$&if $i \le d$\\
$i+h$&if $i > d$\\
\end{tabular}\right.
\]
The systems originating from \cite{Bru78a}
(for instance those in \cite{Bru78b})
are the first to allow delayed updating
by featuring the term node $\phi(f)$
where $f$ is an arbitray function of type $\naturals^+ \to \naturals^+$.
Other systems of the same family like \cite{Ned79,Ned80,KN93}
feature the term node $\mu(d,h)$ or $\phi^{(d,h)}$
that holds the function $\tau_{d,h}$.
Moreover, the systems originating from \cite{ACCL91}
(for instance those in \cite{CHL96})
feature the explicit substitution constructors $\mathit{id}$ and $\uparrow$
that essentially hold the functions $\tau_{0,0}$ (the identity)
and $\tau_{0,1}$ (the successor) respectively.

The system we are going to introduce takes a simpler approach.
Its syntax has a term node $p$ we call \emph{inner numeric label}
where $p \in \naturals^+$.
An active $p$ holds the function $\tau_{0,p}$
while a passive, i.e. present but ignored, $p$ holds the function $\tau_{0,0}$.
In some sense, we want to show that supporting the functions $\tau_{0,h}$
suffices to implement delayed updating in basic name-free $\lambda$-calculus.
